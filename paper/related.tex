% !TEX root = main.tex

\chapter{Related Work}
\label{ch:related}
\noindent


% TODO: Fact check

\textbf{General purpose language framework.}
% Lem
Lem~\cite{lem} is a language and tool for developing large-scale, rigorous
semantics; it supports multiple output formats (e.g., LaTeX, Coq, and OCaml) to
integrate formal specifications with proof tools and implementations.
While Lem itself does not directly generate interpreters, its OCaml backend can
be used to implement one.
% Ott
Ott~\cite{ott} is a lightweight tool for writing and maintaining formal
semantics of programming languages, enabling generation of LaTeX for
documentation and definitions for theorem provers such as Coq, Isabelle, or
Twelf.
It does not directly generate interpreters but serves as a frontend for
creating formal specifications that can be used to derive interpreters manually
or via theorem provers.
% PLT
PLT Redex~\cite{plt} is a Racket-based domain-specific language for specifying,
experimenting with, and testing operational semantics, making it particularly
suited for exploring small-step semantics and reduction systems.
It provides an interactive environment for evaluating terms, which is akin to
generating an interpreter.
% K
K Framework~\cite{k} is a formal language for defining programming languages
and formal analysis tools; it enables the generation of interpreters,
compilers, and formal verification frameworks directly from the semantics,
making it a powerful tool for executable formal semantics.
% Sail
Sail~\cite{sail} is a domain-specific language designed for the formal
specification of hardware architectures, particularly processors, using an
abstract, executable semantics.
While it is primarily used for hardware modeling, its general framework can
also be applied to other systems for language.
However, none of these tools are capable of specifying an algorithmic
specification.


\textbf{Specialized language framework.}
% ESMeta
ESMeta~\cite{esmeta} is a language framework specialized for JavaScript.
It parses the JavaScript specification, ECMAScript, to automatically generate
parser and interpreter ~\cite{jiset}, to generate tests and perform
differential tests ~\cite{jest}, to find meta-level errors in the specification
~\cite{jstar}, and to perform meta-level static analysis for the static
analysis of the defined language ~\cite{jsaver}.
However, it is not capable of specifying declarative specification.
% ASL
ASL~\cite{asl} is a language framework specialized for specifying the behavior
and semantics of ARM instruction sets.
% Only test?
It also parses the algorithmic specification of ARM and test the specification.
% ASL to Sail
ASL to Sail project~\cite{asl2sail} translates ASL into Sail to enable formal
verification, analysis, and multi-platform compatibility of ARM architecture
semantics.
Hence, ASL can specify both declarative and algorithmic specifications;
however, it takes an opposite approach compared to SpecTec, translating the
algorithmic specification into a declarative form.
