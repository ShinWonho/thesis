% !TEX root = main.tex

\chapter{Declarative Specification Language}
\label{ch:dsl}
\noindent


% DSL high-level explanation
SpecTec offers a domain-specific language (DSL) tailored for specification
purposes.
The DSL is a declarative language, providing a compact and user-friendly syntax.
It includes definitions for type, relation, and functions.
% Type definition
Types define the abstract syntax in a mannar analogous to the BNF format, where
terminals are represented by capital letters, while non-terminals are denoted
by lowercase letters.
\cref{ex:dsl-syntax} represents the abstract syntax for \texttt{testop},
\texttt{numtype} and \texttt{instr}.
\\
\begin{example}
\begin{lstlisting}[style=dsl]
syntax testop = EQZ
syntax numtype = I32 | I64 | F32 | F64 | V128
syntax instr =
  | CONST numtype num
  | TESTOP numtype testop
  | ...
\end{lstlisting}
\label{ex:dsl-syntax}
\end{example}


% Relation definition
With relation, user can define the formal semantics of Wasm.
\cref{ex:dsl-testop} illustrates the execution semantics for \texttt{testop}
instruction described in a relation definition.
\\
\begin{example}
\begin{lstlisting}[style=dsl]
relation Step_pure: instr* ~> instr*
rule Step_pure/testop:
(CONST nt c_1) (TESTOP nt testop)  ~>  (CONST I32 c)
-- if c = @\dollarcode@testop_(nt, testop, c_1)
\end{lstlisting}
  \label{ex:dsl-testop}
\end{example}
The signature of the \texttt{Step\_pure} relation $\tildecode$\texttt{>} is
defined first.
A \texttt{Step\_pure} rule is then specified to describe the semantics of the
\texttt{testop} instruction with premises under which the relation holds.
SpecTec performs type checking to reject any incorrect symbols or ill-typed
relations like below codes.
\\
\begin{example}
\begin{lstlisting}[style=dsl]
(* Wrong type *)
rule Step_pure/testop:
(CONST nt c_1) (TESTOP nt testop)  ~>  c
-- if c = @\dollarcode@testop_(nt, testop, c_1)

(* Wrong symbol *)
rule Step_pure/testop:
(CONST nt c_1) (TESTOP nt testop)  =>  (CONST I32 c)
-- if c = @\dollarcode@testop_(nt, testop, c_1)
\end{lstlisting}
  \label{ex:dsl-error}
\end{example}


% Function definition
Functions enable auxiliary definitions that are used for other definitions such
as \texttt{\$testop\_} in the relation for \texttt{testop} in the code example
above.
As with relation definitions, defining a function begins by specifying the
signature of the function.
The function's actual definition is then specified, including output value and
the premises under which the function is applicable.
\cref{ex:dsl-fun} represents the function definition for \texttt{\$testop\_}.
% XXX: Hack
\newpage
\begin{example}
\begin{lstlisting}[style=dsl]
def @\dollarcode@testop_(nt, testop, num) : num
def @\dollarcode@testop_(nt, testop, c) = 1
-- if c = 0
def @\dollarcode@testop_(nt, testop, c) = 1
-- otherwise
\end{lstlisting}
  \label{ex:dsl-fun}
\end{example}
In addition, functions themselves can be used to describe semantics, rather
than merely serving as auxiliary definitions.
Specifically, SpecTec employs function definitions to describe semantics of module
instantiation and function invocation.


% LaTeX and mechanized definitions
SpecTec generates LaTeX code from these DSL definitions.
The DSL allows users to include rendering hints, enabling them to customize how
the definitions are rendered.
The below grey box in \cref{fig:spectec-testop} is rendered result of
LaTeX code generated from the relation definition.
By using rendering hints, \texttt{(CONST nt c)} appears as \texttt{(nt.const
c)}, and \texttt{(TESTOP nt testop)} is displayed as \texttt{(nt.testop)}.
In addition, SpecTec generates mechanized definitions for theorem provers
enabling the formal verification of properties such as type soundness.
