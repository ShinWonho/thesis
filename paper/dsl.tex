% !TEX root = main.tex

\chapter{DSL}
\label{ch:dsl}
\noindent


% DSL high-level explanation
SpecTec offers a domain-specific language (DSL) tailored for specification
purposes.
The DSL is a declarative language, providing a compact and user-friendly syntax.
It includes definitions for type, relation, and functions.
Type definitions define the abstract syntax in a mannar analogous to the BNF
format, where terminals are denoted by capital letters, while non-terminals are
represented by lowercase letters.
The below code represents the abstract syntax for \texttt{testop},
\texttt{numtype} and \texttt{instr}.
\begin{lstlisting}[style=dsl]
syntax testop = EQZ
syntax numtype = I32 | I64 | F32 | F64 | V128
syntax instr =
  | CONST numtype number
  | TESTOP numtype testop
  | ...
\end{lstlisting}

With relation, user can define the formal semantics of Wasm.
The below code illustrates the execution semantics for \texttt{testop}
instruction described in a relation definition.
\begin{lstlisting}[style=dsl]
  relation Step_pure: instr* ~> instr*
  rule Step_pure/testop:
  (CONST numtype c_1) (TESTOP numtype testop)  ~>  (CONST I32 c)
  -- if c = @$\dollarcode$@testop_(numtype, testop, c_1)
\end{lstlisting}
The signature of the step relation $\tildecode$\texttt{>} is defined first.
It then specifies a step rule that describes the semantics of the
\texttt{testop} instruction with premises under which the relation holds.
SpecTec performs type checking to ensure that any incorrect symbols or
ill-typed relations like below code are rejected.
\begin{lstlisting}[style=dsl]
  (* Wrong type *)
  rule Step_pure/testop:
  (CONST numtype c_1) (TESTOP numtype testop)  ~>  c
  -- if c = @$\dollarcode$@testop_(numtype, testop, c_1)

  (* Notation misuse *)
  rule Step_pure/testop:
  (CONST numtype c_1) (TESTOP numtype testop)  =>  (CONST I32 c)
  -- if c = @$\dollarcode$@testop_(numtype, testop, c_1)
\end{lstlisting}


TODO: funciton definition
Function definitions can be utilized for other definitions such as
\texttt{\$testop\_} in the code.
In addition, they can be also used to describe semantics.
Especially, SpecTec uses function definitions to describe module instantiation
and function invocation.

TODO: latex and mechanized definitions
