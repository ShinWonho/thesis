% !TEX root = main.tex

\chapter{DSL}
\label{ch:dsl}
\noindent


% DSL high-level explanation
SpecTec offers a domain-specific language (DSL) tailored for specification
purposes.
It is a declarative language, providing a compact and user-friendly syntax.
It consists of DSL definitions: type, grammar, relation, and functions.
Type definitions are used to define abstract syntax, while grammar definitions
are used to define concrete syntax.
The below code represents the abstract syntax and concrete syntax for
\texttt{numtype}.
\begin{lstlisting}[style=dsl]
syntax numtype = I32 | I64 | F32 | F64 | V128
grammar Bnumtype : numtype =
  | 0x7F => I32
  | 0x7E => I64
  | 0x7D => F32
  | 0x7C => F64
\end{lstlisting}
Relation definitions are used to define semantics, while function definitions
are used auxiliary definitions.

It is type checked to prevent meta-level errors such as notation misuses and
dimension mismatches. The code below is the specification for \texttt{testop}
instruction written in the DSL, which is designed to be similar to the formal
notation when describing the semantics with compact and user-friendly
notation.
\begin{lstlisting}[style=dsl]
  rule Step_pure/testop:
  (CONST nt c_1) (TESTOP nt testop)  @$\tildecode$@>  (CONST I32 c)
  -- if c = @$\dollarcode$@testop_(nt, testop, c_1)
\end{lstlisting}


% DSL
The DSL consists of DSL definitions.
There are mainly 4 kinds of DSL definitions: type, grammar, relation, and
functions.
Type definitions are used to define abstract syntax, while grammar definitions
are used to define concrete syntax.
Relation definitions are used to define semantics; the code above is also
described in relation definition.
Function definitions can be utilized for other definitions such as
\texttt{\$testop\_} in the code.
In addition, they can be also used to describe semantics.
Especially, SpecTec uses function definitions to describe module instantiation
and function invocation.
