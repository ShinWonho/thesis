% !TEX root = main.tex

\chapter{Motivation}
\label{ch:motivation}
\noindent

\newcommand{\officialp}{official prose}
\newcommand{\spectecp}{SpecTec prose}

\red{TODO: remove redundant part in the fig}



% control flow structure in official prose
It might be easy to understand how the \officialp{} explains the control flow
of WebAssembly, if we assume that a WebAssembly code is loaded on a memory and
a pc points to the instruction to execute.
To describe control flow, it uses a structure named \textit{block} which
consitutes of a instruction sequence.
When executing the instructions in the block, pc can be changed to the starting
point of the block or the end of the block.


% an example of Wasm control flow in official prose
\textbf{Example 1}
\begin{verbatim}
  // infinite loop
  (loop (result i32) (i32.const 42) (br 0) (unreachable) end) (unreachable)
\end{verbatim}

Example 1 is a WebAssembly code example that has \texttt{loop} and
\texttt{unreachable}
Here, result type of the \texttt{loop} is \texttt{i32} and the \texttt{loop}
has a block of three instructions: \texttt{i32.const}, \texttt{br}, and
\texttt{unreachable}.
\cref{fig:loop} is the \officialp{} of the \texttt{loop} instruction.
It says that the continuation is the start of the loop, the information is
stored in a label, and it \textit{enters} the block with the label.
The term \textit{enter} means that it pushes the label in the stack and makes
pc points to the first instruction in the block.
The \texttt{i32.const} instruction is executed first, which just pushes the
\texttt{i32} value \texttt{42} in the stack.
Then, The \texttt{br} instruction is executed.
\cref{fig:br} is the \officialp{} of the \texttt{br} instruction.
It pops the label from the stack, and makes pc points to the continuation of
the label, which is the start of the loop instruction.
As a result, the code example above is a infinite loop pushing \texttt{42}
forever without executing \texttt{unreachable} instructions.

\begin{figure}[h!]
    \centerline{\includegraphics[width=15cm]{fig/loop}}
    \caption[Enter the caption title here]{\texttt{loop} instruction} \label{fig:loop}
    \centerline{\includegraphics[width=15cm]{fig/br}}
    \caption[Enter the caption title here]{\texttt{br} instruction} \label{fig:br}
\end{figure}


% exiting label in official prose
There is also a special form of a behavior related to the block: an
\textit{exiting label}.
\cref{fig:exiting-label} is the \officialp{} of the \textit{exiting label}.
It is special because this behavior is performed without an explicit
WebAssembly instruction.
Rather, the behavior is performed when \textbf{the end of a block is reached}
without control instructions or runtime error.
The behavior is that the label is popped from the stack, and the pc changes to
the point after the end of the block.

\begin{figure}[h!]
    \centerline{\includegraphics[width=15cm]{fig/exiting}}
    \caption[Enter the caption title here]{exiting label} \label{fig:exiting-label}
\end{figure}


% an example of exiting label in official prose
\textbf{Example 2}
\begin{verbatim}
  // exiting label
  (loop (result i32) (i32.const 42) end) (f64.const 3.14)
\end{verbatim}

Example 2 is a WebAssembly code example that has \texttt{loop} whose block is
only \texttt{i32.const}, and \texttt{f32.const}
When the \texttt{loop} instruction is executed, a label is pushed in the stack,
and it enters the block.
After \texttt{i32.const} is executed, the end of the block is reached.
As a result, exiting label occurs so that the label is popped from the stack,
pc changes to the point after the end of block: \texttt{f64.const}.
As a result, after executing the code, the value 42 and 3.14 pushed to the
stack.


% general control flow structure of official prose
Similar to the control flow using the block and the label, there are many other
control instructions and structures such as a \texttt{call} instruction that
calls a function with pushing a frame to stack and a \texttt{return}
instruction that returns from the function with popping the frame from the
stack.
Not only that, there are the special behaviors such as a \textit{returning from
a function} that corresponds to the exiting label.


% control flow structure in SpecTec prose
However, the \spectecp{} assumes different a model to explain the control flow
of WebAssembly.
Rather than using the notion of a pc, \spectecp{} assumes that WebAssembly
instructions are just given one by one.
This is because the \spectecp{} is generated automatically from the
\red{SpecTec dsl}.
\red{SpecTec dsl} uses rewrite rule to describe WebAssembly semantics, and it
can be seen as consuming a WebAssembly instruction and values in the stack,
changing the stack and the store, and inputing new WebAssembly instructions.


% an example of Wasm control flow in official prose
\begin{figure}[h!]
    \centerline{\includegraphics[width=15cm]{fig/spectec-loop}}
    \caption[Enter the caption title here]{SpecTec \texttt{loop}} \label{fig:spectec-loop}
\end{figure}
\begin{figure}[h!]
    \centerline{\includegraphics[width=15cm]{fig/spectec-br}}
    \caption[Enter the caption title here]{SpecTec \texttt{br}} \label{fig:spectec-br}
\end{figure}

Consider example 1 above.

\texttt{loop} and \texttt{unreachable} is the input instruction for the model.
\cref{fig:spectec-loop} is the \spectecp{} of the \texttt{loop} instruction.
Rather than storing the point to jump as a continuation in the label, it stores
the \texttt{loop} instruction itself and \textit{enters} the block.
Here, \textit{enter} means that it pushes the label in the stack and inputs the
instructions as inputs to execute.
As a result, after the \texttt{loop} is executed, \texttt{i32.const},
\texttt{br}, \texttt{unreachable}, and \texttt{unreachable} are remained as
inputs to the model.
When \texttt{i32.const} is excuted, it pushes 42 as \officialp{} does.
However, \texttt{br} behaves a bit differently.
\cref{fig:spectec-br} is the \spectecp{} of the \texttt{br} instruction.
It pops the label from the stack, removes the input instructions until the end
of the block, considers the loop instruction in the label as a new input
instruction, and executes it again.
Consequently, remaining inputs are \texttt{loop}, and \texttt{unreachable}.
Therefore, it explains the behavior in different point of view, but actually
describes the same thing.


% exiting label in SpecTec prose
However, this viewpoint is challenged by the exiting label.
As the block structure doesn't remain intact, it is hard to describe
\textbf{the end of the block}.
Furthermore, the exiting label is not performed by a specific WebAssembly
instruction, which makes it hard to model the behavior.
To handle this problem, an administrative instruction \texttt{end} is
introduced in \spectecp.
When entering a block, assume that there is an \texttt{end}




% official spec prose notation control structure
% seems to be pc-based
% AL from DSL ~ formal notation: no pc
% AL: instruction sequence input continously
% device exiting semantics for AL
% Executable spec passes all wasm testcases
% weird, maybe not happened in realworld, but valid wasm code
% wrong!
