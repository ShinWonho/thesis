% !TEX root = main.tex

\chapter{Formalization}
\label{ch:Formalization}
\noindent



\newcommand{\seq}[1]{#1^*}



%% font definitions

\renewcommand{\symbol}[1]{\textbf{#1}}
\newcommand{\helper}[1]{\textsf{#1}}
\newcommand{\premise}[1]{\textsf{\textbf{#1}}}

% Algorithm
\newcommand{\rel}{\symbol{Rel}}
\newcommand{\fun}{\symbol{Fun}}

% Continuation
\newcommand{\mt}{\symbol{Empty}}
\newcommand{\toplevelcall}{\symbol{TopLevelCall}}
\newcommand{\call}{\symbol{Call}}
\newcommand{\wasm}{\symbol{Wasm}}
\newcommand{\exe}{\symbol{Execute}}
\newcommand{\algo}{\symbol{Algo}}
\newcommand{\ret}{\symbol{Return}}

% Instruction
\newcommand{\ifi}{\symbol{IfI}}
\newcommand{\eitheri}{\symbol{EitherI}}
\newcommand{\enteri}{\symbol{EnterI}}
\newcommand{\pushctxi}{\symbol{PushCtxI}}
\newcommand{\pushi}{\symbol{PushI}}
\newcommand{\popctxi}{\symbol{PopCtxI}}
\newcommand{\popi}{\symbol{PopI}}
\newcommand{\popni}{\symbol{PopNI}}
\newcommand{\popalli}{\symbol{PopAllI}}
\newcommand{\leti}{\symbol{LetI}}
\newcommand{\trapi}{\symbol{TrapI}}
\newcommand{\nopi}{\symbol{NopI}}
\newcommand{\returnreli}{\symbol{ReturnRelI}}
\newcommand{\returnfunci}{\symbol{ReturnFuncI}}
\newcommand{\executei}{\symbol{ExecuteI}}
\newcommand{\calli}{\symbol{CallI}}
\newcommand{\replacei}{\symbol{ReplaceI}}

% Expression
\newcommand{\vare}{\symbol{VarE}}
\newcommand{\nume}{\symbol{NumE}}
\newcommand{\boole}{\symbol{BoolE}}
\newcommand{\fnamee}{\symbol{FnameE}}
\newcommand{\une}{\symbol{UnE}}
\newcommand{\bine}{\symbol{BinE}}
\newcommand{\acce}{\symbol{AccE}}
\newcommand{\upde}{\symbol{UpdE}}
\newcommand{\stre}{\symbol{StrE}}
\newcommand{\compe}{\symbol{CompE}}
\newcommand{\cate}{\symbol{CatE}}
\newcommand{\meme}{\symbol{MemE}}
\newcommand{\lene}{\symbol{LenE}}
\newcommand{\tupe}{\symbol{TupE}}
\newcommand{\casee}{\symbol{CaseE}}
\newcommand{\itere}{\symbol{IterE}}
\newcommand{\liste}{\symbol{ListE}}
\newcommand{\getcurctxe}{\symbol{GetCurContextE}}
\newcommand{\choosee}{\symbol{ChooseE}}
\newcommand{\iscaseofe}{\symbol{IsCaseOfE}}
\newcommand{\ctxkinde}{\symbol{CtxKindE}}
\newcommand{\matche}{\symbol{MatchE}}
\newcommand{\hastypee}{\symbol{HasTypeE}}

% Unary operator
\newcommand{\notop}{\symbol{NotOp}}
\newcommand{\minusop}{\symbol{MinusOp}}

% Binary operator
\newcommand{\addop}{\symbol{AddOp}}
\newcommand{\subop}{\symbol{SubOp}}
\newcommand{\mulop}{\symbol{MulOp}}
\newcommand{\divop}{\symbol{DivOp}}
\newcommand{\modop}{\symbol{ModOp}}
\newcommand{\expop}{\symbol{ExpOp}}
\newcommand{\implop}{\symbol{ImplOp}}
\newcommand{\equivop}{\symbol{EquivOp}}
\newcommand{\andop}{\symbol{AndOp}}
\newcommand{\orop}{\symbol{OrOp}}
\newcommand{\eqop}{\symbol{EqOp}}
\newcommand{\neop}{\symbol{NeOp}}
\newcommand{\ltop}{\symbol{LtOp}}
\newcommand{\gtop}{\symbol{GtOp}}
\newcommand{\leop}{\symbol{LeOp}}
\newcommand{\geop}{\symbol{GeOp}}

% Path
\newcommand{\idxp}{\symbol{Idx}}
\newcommand{\slicep}{\symbol{Slice}}
\newcommand{\dotp}{\symbol{Dot}}

% Iter
\newcommand{\listiter}{\symbol{List}}
\newcommand{\listniter}{\symbol{ListN}}
\newcommand{\listidxiter}{\symbol{Index}}

% Value
\newcommand{\numv}{\symbol{NumV}}
\newcommand{\boolv}{\symbol{BoolV}}
\newcommand{\listv}{\symbol{ListV}}
\newcommand{\strv}{\symbol{StrV}}
\newcommand{\casev}{\symbol{CaseV}}
\newcommand{\tupv}{\symbol{TupV}}
\newcommand{\fnamev}{\symbol{FnameV}}
\newcommand{\trapv}{\symbol{TrapV}}
\newcommand{\storev}{\symbol{StoreV}}




% Helper function
\newcommand{\domain}{\helper{domain}}
\newcommand{\getalgoname}{\helper{get\_algo\_name}}
\newcommand{\lookup}{\helper{lookup}}
\newcommand{\createalgo}{\helper{create\_algo}}
\newcommand{\execute}{\helper{execute}}
\newcommand{\exit}{\helper{exit}}
\newcommand{\duplicateenv}{\helper{duplicate\_env}}
\newcommand{\getenv}{\helper{get\_env}}
\newcommand{\setenv}{\helper{set\_env}}
\newcommand{\getctx}{\helper{get\_ctx}}
\newcommand{\addctx}{\helper{add\_ctx}}
\newcommand{\clearctx}{\helper{clear\_ctx}}
\newcommand{\getstore}{\helper{get\_store}}
\newcommand{\setstore}{\helper{set\_store}}
\newcommand{\prependinstr}{\helper{prepend\_instr}}
\newcommand{\popwasminstr}{\helper{pop\_wasm\_instr}}
\newcommand{\push}{\helper{push}}
\newcommand{\pop}{\helper{pop}}
\newcommand{\popn}{\helper{popn}}
\newcommand{\unop}{\helper{unop}}
\newcommand{\splitctx}{\helper{split\_ctx}}
\newcommand{\getcurctx}{\helper{get\_cur\_ctx}}
\newcommand{\getcurframe}{\helper{get\_cur\_frame}}
\newcommand{\setcurframe}{\helper{set\_cur\_frame}}
\newcommand{\access}{\helper{access}}
\newcommand{\update}{\helper{update}}
\newcommand{\updateidx}{\helper{update\_idx}}
\newcommand{\updateslice}{\helper{update\_slice}}
\newcommand{\updatedot}{\helper{update\_dot}}
\newcommand{\getendinstr}{\helper{get\_end\_instr}}
\newcommand{\isendinstr}{\helper{is\_end\_instr}}
\newcommand{\istrue}{\helper{is\_true}}
\newcommand{\isframe}{\helper{is\_frame}}
\newcommand{\isctx}{\helper{is\_ctx}}
\newcommand{\zip}{\helper{zip}}
\newcommand{\fold}{\helper{fold}}
% Hardcode
\newcommand{\assign}{\helper{assign}}
\newcommand{\binop}{\helper{binop}}
% Reference interpreter
\newcommand{\match}{\helper{match}}
\newcommand{\hastype}{\helper{has\_type}}




%% Syntax
\section{Syntax of AL}
\label{syntax}

\begin{align*}
\begin{array}{lcccrlr}
%
% State
  \text{State}\quad& \Sigma &\ni& \sigma & ::=& \seq a, w, k \\
%
% Algorithm
  \text{Algorithm}\quad& A &\ni& a &::=& ~ \rel ~ (s, \seq e, \seq i) &\quad\text{(Relation)} \\
  &&&& | & ~ \fun ~ (s, \seq e, \seq i) &\quad\text{(Function)} \\
%
% Wasm state
  \text{Wasm State}\quad& W &\ni& w &::=& \seq{wv}, \seq{wi}, sto \\
  \text{Wasm Value}\quad& WV &\ni& wv &::=& v \\
  \text{Wasm Instruction}\quad& WI &\ni& wi &::=& v \\
%
% Store
  \text{Store}\quad& Store &\ni& sto &::=& \seq{(s, v)} \\
%
% Continuation
  \text{Continuation}\quad & K &\ni& k &::=& ~ \mt &\quad\text{(Empty)} \\
    &&&& | & ~ \toplevelcall ~ (s, \seq v) &\quad\text{(Top-level Call)} \\
    &&&& | & ~ \call ~ (s, s, \seq v, k) &\quad\text{(Call)} \\
    &&&& | & ~ \exe ~ (\seq{wi}, k) &\quad\text{(Execute)} \\
    &&&& | & ~ \wasm ~ (\seq c, k) &\quad\text{(Wasm)} \\
    &&&& | & ~ \algo ~ (\seq i, \mu, \seq c, k) &\quad\text{(Algorithm)} \\
    &&&& | & ~ \ret ~ (v, k) &\quad\text{(Return)} \\
%
% Environment
  \text{Environment}\quad& M &\ni& \mu &::=& \seq{[s \mapsto v]} \\
%
% Context
  \text{Context}\quad& C &\ni& c &::=& v \\
%
% Instruction
  \text{Instruction}\quad& I &\ni& i &::=& ~ \ifi ~ (e, \seq i, \seq i) &\quad\text{(If)} \\
    &&&& | & ~ \eitheri ~ (\seq i, \seq i) &\quad\text{(Either)} \\
    &&&& | & ~ \enteri ~ (e, e) &\quad\text{(Enter)} \\
    &&&& | & ~ \pushctxi ~ e &\quad\text{(Push Context)} \\
    &&&& | & ~ \pushi ~ e &\quad\text{(Push)} \\
    &&&& | & ~ \popctxi ~ e &\quad\text{(Pop Context)} \\
    &&&& | & ~ \popi ~ e &\quad\text{(Pop)} \\
    &&&& | & ~ \popni ~ (e, e) &\quad\text{(Pop N)} \\
    &&&& | & ~ \popalli ~ e &\quad\text{(Pop All)} \\
    &&&& | & ~ \leti ~ (e, e) &\quad\text{(Let)} \\
    &&&& | & ~ \trapi &\quad\text{(Trap)} \\
    &&&& | & ~ \returnreli &\quad\text{(Return Relation)} \\
    &&&& | & ~ \returnfunci ~ e &\quad\text{(Return Function)} \\
    &&&& | & ~ \executei ~ e &\quad\text{(Execute)} \\
    &&&& | & ~ \calli ~ (s, s, \seq{e}) &\quad\text{(Let Call)} \\
    &&&& | & ~ \replacei ~ (e, \seq{p}, e) &\quad\text{(Replace)} \\
\end{array}
\end{align*}
\newpage
\begin{align*}
\begin{array}{lcccrlr}
%
% Expression
  \text{Expression}\quad& E &\ni& e &::=& ~ \vare ~ s &\quad\text{(Variable)} \\
    &&&& | & ~ \nume ~ n &\quad\text{(Number)} \\
    &&&& | & ~ \boole ~ b &\quad\text{(Boolean)} \\
    &&&& | & ~ \fnamee ~ s &\quad\text{(Function Name)} \\
    &&&& | & ~ \liste ~ \seq e &\quad\text{(List)} \\
    &&&& | & ~ \stre ~ \seq{(s, e)} &\quad\text{(Record)} \\
    &&&& | & ~ \tupe ~ \seq e &\quad\text{(Tuple)} \\
    &&&& | & ~ \casee ~ (s, \seq e) &\quad\text{(Tagged Tuple)} \\
    &&&& | & ~ \une ~ (unop, e) &\quad\text{(Unary Operation)} \\
    &&&& | & ~ \bine ~ (binop, e, e) &\quad\text{(Binary Operation)} \\
    &&&& | & ~ \acce ~ (e, p) &\quad\text{(Access)} \\
    &&&& | & ~ \upde ~ (e, \seq{p}, e) &\quad\text{(Update)} \\
    &&&& | & ~ \cate ~ (e, e) &\quad\text{(Concatenation)} \\
    &&&& | & ~ \compe ~ (e, e) &\quad\text{(Composition)} \\
    &&&& | & ~ \meme ~ (e, e) &\quad\text{(Membership)} \\
    &&&& | & ~ \choosee ~ e &\quad\text{(Choose)} \\
    &&&& | & ~ \lene ~ e &\quad\text{(Length)} \\
    &&&& | & ~ \iscaseofe ~ (e, s) &\quad\text{(Check Tag)} \\
    &&&& | & ~ \getcurctxe ~ s &\quad\text{(Get Current Context)} \\
    &&&& | & ~ \ctxkinde ~ s &\quad\text{(Context Kind)} \\
    &&&& | & ~ \itere ~ (e, iter, \seq{s}) &\quad\text{(Iteration)} \\
    &&&& | & ~ \matche ~ (e, e) &\quad\text{(Match)} \\
    &&&& | & ~ \hastypee ~ (e, s) &\quad\text{(Has Type)} \\
%
% Operator
  \text{Unary Operator}\quad& Unop &\ni& unop &::=& ~ \notop ~ | ~ \minusop \\
  \text{Binary Operator}\quad& Binop &\ni& binop &::=& ~ \addop &\quad\text{(Addition)} \\
    &&&& | & ~ \subop &\quad\text{(Subtraction)} \\
    &&&& | & ~ \mulop &\quad\text{(Multiplication)} \\
    &&&& | & ~ \divop &\quad\text{(Division)} \\
    &&&& | & ~ \modop &\quad\text{(Modulo)} \\
    &&&& | & ~ \expop &\quad\text{(Exponential)} \\
    &&&& | & ~ \implop &\quad\text{(Implication)} \\
    &&&& | & ~ \equivop &\quad\text{(Equivalence)} \\
    &&&& | & ~ \andop &\quad\text{(And)} \\
    &&&& | & ~ \orop &\quad\text{(Or)} \\
    &&&& | & ~ \eqop &\quad\text{(Equal)} \\
    &&&& | & ~ \neop &\quad\text{(Not Equal)} \\
    &&&& | & ~ \ltop &\quad\text{(Less Than)} \\
    &&&& | & ~ \gtop &\quad\text{(Greater Than)} \\
    &&&& | & ~ \leop &\quad\text{(Less Than or Equal)} \\
    &&&& | & ~ \geop &\quad\text{(Greater Than or Equal)} \\
\end{array}
\end{align*}
\newpage
\begin{align*}
\begin{array}{lcccrlr}
%
% Path
  \text{Path}\quad& P &\ni& p &::=& ~ \idxp ~ e ~ | ~ \slicep ~ (e, e) ~ | ~ \dotp ~ s \\
%
% Iter
  \text{Iter}\quad& Iter &\ni& iter &::=& ~ \listiter ~ | ~ \listniter ~ e ~|~ \listidxiter ~ (s, e) \\
%
% Value
  \text{Value}\quad& V &\ni& v &::=& ~ \numv ~ n &\quad\text{(Number)} \\
    &&&& | & ~ \boolv ~ b &\quad\text{(Boolean)} \\
    &&&& | & ~ \fnamev ~ s &\quad\text{(Function Name)} \\
    &&&& | & ~ \listv ~ \seq v &\quad\text{(List)} \\
    &&&& | & ~ \strv ~ \seq{(s, v)} &\quad\text{(Record)} \\
    &&&& | & ~ \tupv ~ \seq v &\quad\text{(Tuple)} \\
    &&&& | & ~ \casev ~ (s, \seq v) &\quad\text{(Tagged Tuple)} \\
    &&&& | & ~ \trapv &\quad\text{(Trap)} \\
    &&&& | & ~ \storev &\quad\text{(Store)} \\
% Primitives
  \text{String}\quad& \mathbb S &\ni& s \\
  \text{Integer}\quad& \mathbb Z &\ni& n \\
  \text{Boolean}\quad& \mathbb B &\ni& b \\
\end{array}
\end{align*}

% state
An AL state consists of a sequence of algorithms, a Wasm state, and a
continuation.
% algorithm
An algorithm is either a relation or a function, corresponding to the relation
and the function definitions in the DSL.
% Wasm state
A Wasm state consists of a Wasm value stack, a Wasm instruction stack, and a
store.
% Wasm value & instruction
A Wasm value and instruction are expressed as AL values, particularly in the
form of $\casev$.
% store
A store is expressed as a sequence of field name and AL value pairs.
% continuation
A continuation is \mt{} for an empty continuation, \toplevelcall{} for
top-level function call, \call{} for a function call with let binding, \exe{}
for executing given Wasm instructions, \wasm{} for executing a Wasm instruction
in the instruction stack, \algo{} for interpreting AL instructions in an
algorithm, or \ret{} for returning from an AL function.
% env
An environment is a finite mapping from variable names to AL values.
% context
A context is an AL value, particularyly in the form of \casev{}.
% instruction
An instruction is \ifi{} for conditionally choosing a branch, \eitheri{} for
nondeterministically choosing a branch, \enteri{} for entering a block,
\pushctxi{} for pushing a context, \pushi{} for pushing a Wasm value,
\popctxi{} for popping a context, \popi{} for popping a Wasm value, \popni{}
for popping $n$ Wasm values, \popalli{} for popping all Wasm values within a
context, \leti{} for let binding, \trapi{} for trapping, \returnreli{} for
returning from an AL relation, \returnfunci{} for returning from an AL
function, \executei{} for executing given Wasm instructions, \calli{} for an AL
function call with let binding, or \replacei{} for replacing some mutable data
structure.
% expression
An expression is a variable, number, boolean, function name, list, record (or
struct), tuple, tagged tuple, unary or binary operation, data structure access
or update, list or record concatenation, membership check, element selection,
length retrieval, tag check, context retrieval, context kind check, iteration,
match relation, or type relation.
% value
A value is a number, boolean, function name, list, record, tuple, tagged tuple,
trap, or store.


% terminology
In this section, AL-specific terms (state, value, function, and instruction)
are abbreviated as simply state, value, function, and instruction, while Wasm
terms remain unchanged.




\newpage
%% Semantics
\section{Semantics of AL}
\label{semantics}

%% Continuation

\begin{gather*}
\boxed{\leadsto \subseteq \Sigma \times \Sigma} \\
%
% TopLevelCall
\newline \\
  \\
  \hline
  (\seq{a}, w, \toplevelcall ~ (s, \seq v)) \leadsto (\seq{a}, w, \createalgo(\seq a, s, \seq v, \mt)) \\
%
% Call
\newline \\
  \\
  \hline
  (\seq{a}, w, \call ~ (s_{var}, s_{name}, \seq v, k)) \leadsto
  (\seq{a}, w, \createalgo(\seq a, s_{name}, \seq v, \call ~ (s_{var}, s_{name}, \seq v, k))) \\
%
% Execute-empty
\newline \\
  \\
  \hline
  (\seq{a}, w, \exe ~ (\epsilon, k)) \leadsto (\seq{a}, w, k) \\
%
% Execute-instr
\newline \\
  \casev ~ (s, \seq v) = wi \\
  \hline
  (\seq{a}, w, \exe ~ (wi ~ \seq{wi}, k)) \leadsto
  (\seq{a}, w, \createalgo(\seq a, s, \seq v, \exe ~ (\seq{wi}, k))) \\
%
% Wasm-empty
\newline \\
  \\
  \hline
  (\seq{a}, w, \wasm ~ (\epsilon, k)) \leadsto (\seq{a}, w, k) \\
%
% Wasm-instr
\newline \\
  (w', \casev ~ (s, \seq v)) = \popwasminstr(w) \\
  \hline
  (\seq{a}, w, \wasm ~ (c^+, k))
  \leadsto
  (\seq{a}, w', \createalgo(\seq a, s, \seq v, \wasm ~ (\seq c, k))) \\
%
% Al-empty
\newline \\
  \\
  \hline
  (\seq{a}, w, \algo ~ (\epsilon, \mu, \seq c, k)) \leadsto (\seq{a}, w, k) \\
%
% Al-instr
\newline \\
  \seq{a}, w, \algo ~ (\seq{i_1}, \mu, \seq c, k) \vdash i_0 \Rightarrow (w', k') \\
  \hline
  (\seq{a}, w, \algo ~ (i_0 ~ \seq{i_1}, \mu, \seq c, k)) \leadsto (\seq{a}, w', k') \\
%
% Return
\newline \\
  \mu = \getenv(k) \qquad k' = \setenv(k, \mu[s_{var} \mapsto v]) \\
  \hline
  (\seq{a}, w, \ret ~ (v, \call ~ (s_{var}, s_{name}, \seq v, k)) \leadsto (\seq a, w, k') \\
\end{gather*}

The semantics of AL is defined by a state transition system.
% TopLevelCall
Given the algorithms $\seq a$ generated from the DSL and the Wasm state $w$, a
top-level function $s$ can be invoked with arguments $\seq v$ to perform
either module instantiation or function invocation:
$(\seq a, w, \toplevelcall ~ (s, \seq v))$.
\toplevelcall{} changes to \algo{} with the function name $s$, the arguments
$\seq v$, and \mt{}.
% Empty
\mt{} is an empty continuation, representing the end of the transition, so no
further transition occurs.
% Call
Similarly to \toplevelcall{}, \call{} changes to \algo{} with the function name
$s_{name}$ and the arguments $\seq v$, but with the current continuation nested
within it.
%After the call, a return value is bound by $s_{var}$ in the \ret{} transition.
% Execute
\exe{} executes the Wasm instruction sequence $\seq{wi}$ one by one, each
containing a relation name $s$ and arguments $\seq v$.
The continuation changes to \algo{} with the relation name, the arguments, and
the current continuation, with the Wasm instruction being popped.
% Wasm
In \wasm{} continuation, a Wasm instruction is popped from the Wasm
instruction stack and executed until the context sequence $\seq c$ is
exhausted.
This execution changes the continuation to \algo{}, with the relation name $s$,
the arguments $\seq v$, and the current continuation nested within it.
% Algo
In \algo{} continuation, the body instructions $\seq i$ are executed
sequentially, allowing transitions to \call{}, \exe{}, \wasm{}, \ret{}, or
\algo{}.
% Return
The execution of a function call concludes with a return instruction, changing
the continuation to \ret{} with a return value $v$.
If the function call originates from \call{}, the result is assigned to the
variable $s_{var}$.
If it originates from \toplevelcall{}, the inner continuation $k$ is always
\mt, so the entire execution completes with a return value $v$:
$
(\seq a, w, \toplevelcall ~ (s, \seq v))
\leadsto^*
(\seq a, w', \ret ~ (v, \mt))
$.




%% Instruction

\begin{gather*}
  \text{NOTE: the input continuation $k$ is always \algo} \\
  \boxed{\seq{a}, ~ w, ~ k \vdash i \Rightarrow w, ~ k} \\
%
% If-true
\newline \\
  w, \getenv(k) \vdash e \Rightarrow v \qquad
  \istrue(v) \\
  \hline
  \seq{a}, w, k \vdash \ifi ~ (e, \seq{i_1}, \seq{i_2}) \Rightarrow (w, \prependinstr(k, \seq{i_1})) \\
%
% If-false
\newline \\
  w, \getenv(k) \vdash e \Rightarrow v \qquad
  \neg \istrue(v) \\
  \hline
  \seq{a}, w, k \vdash \ifi ~ (e, \seq{i_1}, \seq{i_2}) \Rightarrow (w, \prependinstr(k, \seq{i_2})) \\
%
% Either-1
\newline \\
  \\
  \hline
  \seq{a}, w, k \vdash \eitheri ~ (\seq{i_1}, \seq{i_2}) \Rightarrow (w, \prependinstr(k, \seq{i_1})) \\
%
% Either-2
\newline \\
  \\
  \hline
  \seq{a}, w, k \vdash \eitheri ~ (\seq{i_1}, \seq{i_2}) \Rightarrow (w, \prependinstr(k, \seq{i_2})) \\
%
% Enter
\newline \\
  (\seq{wv}, \seq{wi}, sto) = w \qquad
  \mu = \getenv(k) \qquad
  \seq c = \getctx(k) \\
  w, \mu \vdash e_1 \Rightarrow v_1 \qquad
  w, \mu \vdash e_2 \Rightarrow \listv ~ \seq{v_2} \qquad
  \seq{wi_{end}} = [ ~ \getendinstr(c) ~ | ~ c \leftarrow v_1 ~ \seq c ~ ] \\
  \hline
  \seq{a}, w, k \vdash \enteri ~ (e_1, e_2)
  \Rightarrow
  ((v_1 ~ \seq{wv}, \seq{v_2} ~ \seq{wi_{end}}~ \seq{wi}, sto), \wasm ~ (v_1 ~ \seq c, \clearctx(k))) \\
%
% PushCtx
\newline \\
  w, \getenv(k) \vdash e \Rightarrow v \\
  \hline
  \seq{a}, w, k \vdash \pushctxi ~ e
  \Rightarrow
  (\push(w, v), \addctx(k, v)) \\
%
% Push
\newline \\
  w, \getenv(k) \vdash e \Rightarrow v \\
  \hline
  \seq{a}, w, k \vdash \pushi ~ e \Rightarrow (\push(w, v), k) \\
%
% PopCtx
\newline \\
  wv_{ctx} ~ \seq{wv'} = \exit_{WasmValue}(\seq{wv}) \qquad
  \mu = \assign(\getenv(k), e, wv_{ctx}) \\
  \seq{wi'} = \exit_{WasmInstr}(\seq{wi}) \qquad
  k_1 = \exit_{K}(k) \\
  \hline
  \seq{a}, (\seq{wv}, \seq{wi}, sto), k \vdash \popctxi ~ e
  \Rightarrow
  ((\seq{wv'}, \seq{wi'}, sto), \setenv(k_1, \mu)) \\
%
% Pop
\newline \\
  (w', wv) = \pop(w) \qquad
  \mu' = \assign(\getenv(k), e, wv) \\
  \hline
  \seq{a}, w, k \vdash \popi ~ e \Rightarrow (w', \setenv(k, \mu')) \\
%
% PopN
\newline \\
  \mu = \getenv(k) \qquad
  w, \mu \vdash e_2 \Rightarrow \numv ~ n \\
  (w', wv^n) = \popn(w, n) \qquad
  \mu' = \assign(\mu, e_1, \listv ~ wv^n) \\
  \hline
  \seq{a}, w, k \vdash \popni ~ (e_1, e_2) \Rightarrow (w', \setenv(k, \mu')) \\
%
% PopAll
\newline \\
  (\seq{wv_0}, \seq{wv_1}) = \splitctx(\seq{wv}) \qquad
  \mu' = \assign(\getenv(k), e, \listv ~ \seq{wv_0}) \\
  \hline
  \seq{a}, (\seq{wv}, \seq{wi}, sto), k \vdash \popalli ~ e
  \Rightarrow
  ((\seq{wv_1}, \seq{wi}, sto), \setenv(k, \mu')) \\
%
% Let
\newline \\
  \mu = \getenv(k) \qquad
  w, \mu \vdash e_2 \Rightarrow v \qquad
  \mu' = \assign(\mu, e_1, v) \\
  \hline
  \seq{a}, w, k \vdash \leti ~ (e_1, e_2)
  \Rightarrow
  (w, \setenv(k, \mu')) \\
%
% Trap
\newline \\
  \\
  \hline
  \seq{a}, w, k \vdash \trapi \Rightarrow (w, \ret ~ (\trapv, \mt)) \\
%
% Return-rule
\newline \\
  \\
  \hline
  \seq{a}, w, \algo~ (\seq{i}, \mu, \seq c, k) \vdash \returnreli \Rightarrow (w, k) \\
%
% Return-func
\newline \\
  w, \mu \vdash e \Rightarrow v \\
  \hline
  \seq{a}, w, \algo ~ (\seq{i}, \mu, \seq c, k) \vdash \returnfunci ~ e \Rightarrow (w, \ret ~ (v, k)) \\
%
% Execute
\newline \\
  w, \getenv(k) \vdash e \Rightarrow \listv ~ \seq v \\
  \hline
  \seq{a}, w, k \vdash \executei ~ e \Rightarrow (w, \exe ~ (\seq v, k)) \\
%
% Call-fname
\newline \\
  \mu = \getenv(k) \qquad
  \fnamev ~ s_2 = \mu(s_1) \qquad
  \seq{v} = [ ~ v ~ | ~ e \leftarrow \seq{e}; ~ w, \mu \vdash e \Rightarrow v ~ ] \\
  \hline
  \seq{a}, w, k \vdash \calli ~ (s_0, s_1, \seq e) \Rightarrow (w, \call ~ (s_0, s_2, \seq v, k)) \\
%
% Call-fname
\newline \\
  \mu = \getenv(k) \qquad
  s_1 \not\in \domain(\mu) \qquad
  \seq{v} = [ ~ v ~ | ~ e \leftarrow \seq{e}; ~ w, \mu \vdash e \Rightarrow v ~ ] \\
  \hline
  \seq{a}, w, k \vdash \calli ~ (s_0, s_1, \seq e) \Rightarrow (w, \call ~ (s_0, s_1, \seq v, k)) \\
%
% Replace-frame
\newline \\
\text{NOTE: $e_1$ is either store or current frame} \\
  \mu = \getenv(k) \qquad
  w, \mu \vdash e_1 \Rightarrow v_1 \qquad
  \isframe(v_1) \qquad
  w, \mu \vdash e_2 \Rightarrow v_2 \qquad
  v_3 = \update(w, \mu, v_1, \seq p, v_2) \\
  \hline
  \seq{a}, w, k \vdash \replacei ~ (e_1, \seq{p}, e_2) \Rightarrow (\setcurframe(w, v_3), k) \\
%
% Replace-store
\newline \\
  \mu = \getenv(k) \qquad
  w, \mu \vdash e_1 \Rightarrow \storev \qquad
  w, \mu \vdash e_2 \Rightarrow v_2 \\
  sto = \update(w, \mu, \strv (\getstore(w)), \seq p, v_2) \\
  \hline
  \seq{a}, w, k \vdash \replacei ~ (e_1, \seq{p}, e_2)
  \Rightarrow (\setstore(w, sto), k) \\
\end{gather*}

Given algorithms $\seq a$, a Wasm state $w$, and a continuation $k$, executing an
instruction produces an updated Wasm state and a new continuation.
The input continuation is always \algo{}, with the arguments encapsulated
within it rather than enumerated individually.
% If
\ifi{} instruction evaluates $e$ and branches to $\seq i_1$ if the result is
true, or $\seq i_2$ otherwise.
% Either
\eitheri{} instruction nondeterministically selects either $\seq i_1$ or $\seq
i_2$.
% Enter
\enteri{} instruction evaluates $e_1$, pushing the resulting context $v_1$ to
the Wasm value stack.
It evaluates $e_2$ to get the Wasm instruction sequence $\seq{v_2}$.
Then, end instructions of the context $v_1$ and the contexts $\seq c$ in the
continuation $k$ are appended to $\seq v_2$ and pushed to the Wasm instruction
stack.
Finally, it changes the continuation to \wasm{} with the contexts $c$ and
the continuation $k$ with its contexts cleared.
% PushCtx
\pushctxi{} instruction evaluates $e$, pushing the resulting context $v$ to the
Wasm value stack and the continuation $k$.
% Push
\pushi{} instruction evaluates $e$, pushing the resulting value to the Wasm
value stack.
% PopCtx
\popctxi{} instruction pops all Wasm value up to and including a context
$wv_{ctx}$ from the Wasm value stack and all Wasm instructions up to and
including an end instruction from the Wasm instruction stack.
It also pops the context in the continuation $k$.
Then, it assign the context $wv_{ctx}$ to $e$.
% Pop
\popi{} instruction pops a Wasm value from the Wasm value stack and assign the
value to $e$.
% PopN
\popni{} instruction evalutes $e_2$ to get $n$, pops $n$ Wasm values from the
Wasm value stack, and assign the values to $e_1$.
% PopAll
\popalli{} instruction pops all the Wasm values up to, but not including, a
context from the Wasm value stack and assign the values to $e$.
% Let
\leti{} instruction evaluates $e_2$ and assign the resulting value to $e_1$.
% Trap
\trapi{} instruction results in \ret{} with \trapv{} and \mt{} to terminates
the execution.
% ReturnRel
\returnreli{} instruction results in the inner continuation $k$, indicating a
return from a relation algorithm.
% ReturnFunc
\returnfunci{} instruction evaluates $e$, results in \ret{} with the resulting
value and the inner continuation $k$.
% Execute
\executei{} instruction evaluates $e$ and result in \exe{} with the resulting
values $\seq v$ and the current continuation $k$.
% Call
The \calli{} instruction checks the environment; if $s_1$ exists, it gets the
function name; otherwise, it uses $s_1$ as the function name.
It evaluates $\seq e$ and construct \call{} with binding variable $s_0$, the
function name, and the resulting values.
% Replace
\replacei{} evaluates $e_1$ and $e_2$ to $v_1$ and $v_2$.
If $v_1$ is frame, it accesses the frame with $p$ and update the value to
$v_2$; if $v_1$ is store, it accesses the store with $p$ and update the value
to $v_2$.




%% Expression

\begin{gather*}
  \boxed{w, \mu \vdash e \Rightarrow v} \\
%
% Var
\newline \\
  v = \mu(s) \\
  \hline
  w, \mu \vdash \vare ~ s \Rightarrow v \\
%
% Num
\newline \\
  \hline
  w, \mu \vdash \nume ~ n \Rightarrow \numv ~ n \\
%
% Bool
\newline \\
  \hline
  w, \mu \vdash \boole ~ b \Rightarrow \boolv ~ b \\
%
% Fname
\newline \\
  \hline
  w, \mu \vdash \fnamee ~ s \Rightarrow \fnamev ~ s \\
%
% List
\newline \\
  \seq{v} = [ ~ v ~ | ~ e \leftarrow \seq{e}; ~ w, \mu \vdash e \Rightarrow v ~ ] \\
  \hline
  w, \mu \vdash \liste ~ \seq{e} \Rightarrow \listv ~ \seq{v} \\
%
% Str
\newline \\
  \seq{(s, v)} =
    [ ~
      (s, v)
    ~ | ~
      (s, e) \leftarrow \seq{(s, e)}; ~ w, \mu \vdash e \Rightarrow v
    ~ ] \\
  \hline
  w, \mu \vdash \stre ~ \seq{(s, e)} \Rightarrow \strv ~ \seq{(s, v)} \\
%
% Tup
\newline \\
  \seq{v} = [ ~ v ~ | ~ e \leftarrow \seq{e}; ~ w, \mu \vdash e \Rightarrow v ~ ] \\
  \hline
  w, \mu \vdash \tupe ~ \seq{e} \Rightarrow \tupv ~ \seq{v} \\
%
% Case
\newline \\
  \seq{v} = [ ~ v ~ | ~ e \leftarrow \seq{e}; ~ w, \mu \vdash e \Rightarrow v ~ ] \\
  \hline
  w, \mu \vdash \casee ~ (s, \seq{e}) \Rightarrow \casev ~ (s, \seq{v}) \\
%
% Un
\newline \\
  w, \mu \vdash e \Rightarrow v \\
  \hline
  w, \mu \vdash \une ~ (unop, e) \Rightarrow \unop(unop, v) \\
%
% Bin
\newline \\
  w, \mu \vdash e_1 \Rightarrow v_1 \qquad w, \mu \vdash e_2 \Rightarrow v_2 \\
  \hline
  w, \mu \vdash \bine ~ (binop, e_1, e_2) \Rightarrow \binop(binop, v_1, v_2) \\
%
% Acc
\newline \\
  w, \mu \vdash e \Rightarrow v \\
  \hline
  w, \mu \vdash \acce ~ (e, p) \Rightarrow \access(w, \mu, v, p) \\
%
% Upd
\newline \\
  w, \mu \vdash e_1 \Rightarrow v_1 \qquad w, \mu \vdash e_2 \Rightarrow v_2 \\
  \hline
  w, \mu \vdash \upde ~ (e_1, \seq p, e_2) \Rightarrow \update(w, \mu, v_1, \seq p, v_2) \\
%
% Cat
\newline \\
   w, \mu \vdash e_1 \Rightarrow \listv ~ \seq{v_1} \qquad
   w, \mu \vdash e_2 \Rightarrow \listv ~ \seq{v_2} \\
  \hline
  w, \mu \vdash \cate ~ (e_1, e_2) \Rightarrow \listv ~ (\seq{v_1} ~ \seq{v_1}) \\
%
% Comp
\newline \\
   w, \mu \vdash e_1 \Rightarrow \strv ~ \seq{(s_1, v_1)} \qquad
   w, \mu \vdash e_2 \Rightarrow \strv ~ \seq{(s_2, v_2)} \\
  \hline
  w, \mu \vdash \compe ~ (e_1, e_2) \Rightarrow \strv ~ (\seq{(s_1, v_1)} ~ \seq{(s_2, v_2)}) \\
%
% Mem
\newline \\
  w, \mu \vdash e_1 \Rightarrow v_1 \qquad
  w, \mu \vdash e_2 \Rightarrow \listv ~ \seq{v_2} \\
  \hline
  w, \mu \vdash \meme ~ (e_1, e_2) \Rightarrow \boolv ~ (v_1 \in \seq{v_2}) \\
%
% Choose
\newline \\
  w, \mu \vdash e \Rightarrow \listv ~ \seq{v} \qquad
  v \in \seq{v} \\
  \hline
  w, \mu \vdash \choosee ~ e \Rightarrow v \\
%
% Len
\newline \\
  w, \mu \vdash e \Rightarrow \listv ~ \seq{v} \\
  \hline
  w, \mu \vdash \lene ~ e \Rightarrow \numv ~ |\seq v| \\
%
% IsCaseOf
\newline \\
  w, \mu \vdash e \Rightarrow \casev ~ (s', \seq{v}) \\
  \hline
  w, \mu \vdash \iscaseofe ~ (e, s) \Rightarrow \boolv ~ (s = s') \\
%
% GetCurCtx
\newline \\
  v_{ctx} = \getcurctx(w, s) \\
  \hline
  w, \mu \vdash \getcurctxe ~ s \Rightarrow v_{ctx} \\
%
% CtxKind
\newline \\
  \casev (s', \seq v) = \getcurctx(w, s) \\
  \hline
  w, \mu \vdash \ctxkinde ~ s \Rightarrow \boolv ~ (s = s') \\
%
% Iter-
\newline \\
  f =
    (\lambda ~ s ~ acc. ~
      \listv ~ \seq v = \mu(s); ~
      [ ~ \mu_1[s \mapsto v] ~ | ~ (v, \mu_1) \leftarrow \zip (\seq v, acc) ~ ]
    )
  \\
  \seq{\mu_2} = \fold(f, \seq{s}, \duplicateenv(\mu, \seq{s})) \qquad
  \seq{v'} =
    [ ~
      v'
    ~ | ~
      \mu_3 \leftarrow \seq {\mu_2}; ~ w, \mu_3 \vdash e \Rightarrow v'
    ~ ] \\
  \hline
  w, \mu \vdash \itere ~ (e, \listiter, \seq{s}) \Rightarrow \listv ~ \seq{v'} \\
%
% Iter-n
\newline \\
  w, \mu \vdash e_2 \Rightarrow \numv ~ n \\
  f_i =
    (\lambda ~ x ~ acc. ~
      \listv ~ \seq v = \mu(x); ~
      [ ~ \mu_1[s \mapsto v] ~ | ~ (v, \mu_1) \leftarrow \zip (\seq v, acc) ~ ]
    )
  \\
  \seq{\mu_2} = \fold(f, \seq{s}, \mu^n) \qquad
  \seq{v'} =
    [ ~
      v'
    ~ | ~
      \mu_3 \leftarrow \seq {\mu_2}; ~ w, \mu_3 \vdash e_1 \Rightarrow v'
    ~ ] \\
  \hline
  w, \mu \vdash \itere ~ (e_1, \listniter ~ e_2, \seq{s}) \Rightarrow \listv ~ \seq{v'} \\
%
% Iter-idx
\newline \\
  w, \mu \vdash e_2 \Rightarrow \listv ~ n \qquad
  \seq{\mu_1} =
    [ ~
      \fold(
        (\lambda ~ x ~ acc. ~ acc[x \mapsto \numv ~ i]),
        \seq{s},
        \mu
      )
    ~ | ~
      i \leftarrow [0 ~ .. ~ n-1]
    ~ ]
  \\
  \seq{v'} =
    [ ~
      v'
    ~ | ~
      \mu_2 \leftarrow \seq {\mu_1}; ~ w, \mu_2 \vdash e_1 \Rightarrow v'
    ~ ] \\
  \hline
  w, \mu \vdash \itere ~ (e_1, \listidxiter ~ e_2, \seq{s}) \Rightarrow \listv ~ \seq{v'} \\
%
% Match
\newline \\
  w, \mu \vdash e_1 \Rightarrow v_1 \qquad
  w, \mu \vdash e_2 \Rightarrow v_2 \\
  \hline
  w, \mu \vdash \matche ~ (e_1, e_2) \Rightarrow \boolv ~ (\match(v_1, v_2)) \\
%
% HasType
\newline \\
  w, \mu \vdash e \Rightarrow v \\
  \hline
  w, \mu \vdash \hastypee ~ (e, s) \Rightarrow \boolv ~ (\hastype(v, s)) \\
\end{gather*}

Given Wasm state $w$, and an environment $\mu$, an expression evaluates to a
value.
% Var
\vare{} expression looks up the environment and gets the value.
% Num Bool Fname
\nume{}, \boole{}, and \fnamee{} expressions directly reduce to \numv{},
\boolv{}, and \fnamev{}, respectively.
% Tuple Case List Struct
\liste{}, \stre{}, \tupe{}, and \casee{} expressions evaluate their
sub-expressions and reduce to \listv{}, \strv{}, \tupv{}, and \casev{},
respectively.
% Un Bin
\une{} and \bine{} expression evaluate their sub-expressions and perform
\unop{} and \binop{}, respectively.
% Acc
\acce{} expression evaluates $e$ and accesses the resulting value using a path
$p$.
% Upd
\upde{} expression evaluates $e_1$ and $e_2$ to $v_1$ and $v_2$, and updates
the value at location $v_1$ along paths $\seq p$ to the value $v_2$.
% Cat Comp
\cate{} and \compe{} expressions each evaluate two sub-expressions,
concatenating the resulting lists and resulting records, respectively.
% Mem
\meme{} expression evaluates $e_1$ and $e_2$ to the value $v_1$ and $\listv ~
\seq v_2$, and checks whether $v_1$ is a member of $\seq v_2$.
% Choose Len
\choosee{} and \lene{} expression each evaluate $e$ to the list value $\listv
~ \seq v$, choosing an element from the list and calculating the length of the
list, respectively.
% IsCaseOf
\iscaseofe{} expression evaluates $e$ to the tagged tuple $\casev ~ (s', \seq
v)$ and checks whether the tag $s'$ matches $s$.
% GetCurCtx
\getcurctxe{} expression retrieves the current context from the Wasm value
stack.
% CtxKind
\ctxkinde{} expression retrieves the current context $\casev ~ (s', \seq v)$
and checks whether the kind $s'$ matches $s$.
% Iter
\itere{} expression behaves differently depending on $iter$.
\listiter{} and \listniter{} are essentially equivalent.
They assume that the all the values mapped to the variables $\seq s$ by the
envinronment $\mu${} are list values with same length.
\listniter{} restricts the length to $n$, which is obtained from evaluating
$e_2$, while \listiter{} allows any length.
It generates a sequence of environments $\mu_2$ where the k-th environment maps
each variables to the k-th elements of the list.
Each environment $\mu_3$ is respectively used to evaluate $e_1$, resulting in
the list value $\listv ~ \seq{v'}$.
\listidxiter{} behaves similarly except that the variables $\seq s$ are
unbound.
It evaluates $e_2$ to obtain $n$ and then behaves like \listniter{} as if
the list values [0 ... n-1] was bound to the variables $s$.
% IsValid Match HasType
\matche{} and \hastypee{} expressions evaluate their sub-exressions and perform
\match{} and \hastype{}, respectively.
Note that \match{} and \hastype{} represent the match relation and the type
relation in the Wasm static semantics.
Their formalization is omitted as they are highly specific to Wasm and
excessively verbose.




%% Helper functions

\begin{align*}
%
% domain
\newline \\
  &\domain(\mu) = \{ ~ x ~ | ~ \mu(x) = v ~ \} \\
%
% get_algo_name
\newline \\
  &\getalgoname(\rel ~ (s, \seq e, \seq i)) = s \\
  &\getalgoname(\fun ~ (s, \seq e, \seq i)) = s \\
%
% lookup
\newline \\
  &\lookup(a ~ \seq{a}, s) =
    \begin{cases}
      a &\quad\quad \premise{if}\quad ~ s = \getalgoname(a) \\
      \lookup(\seq{a}, s) &\quad\quad \premise{otherwise}
    \end{cases}
  \\
%
% create_algo
\newline \\
  &\createalgo(\seq a, s, \seq v, k) =
  \algo(\seq i, \mu, \epsilon, k) \\
  &\qquad\qquad\qquad\qquad\premise{if}\quad
  (\rel ~ (s, \seq e, \seq i) \quad\lor\quad \fun ~ (s, \seq e, \seq i)) = \lookup(\seq a, s)
  \quad\land\quad
  \mu = \assign([\text{"s"} \mapsto \storev], \seq e, \seq v) \\
%
% split_ctx
\newline \\
  &\text{NOTE: an input of \splitctx{} (WasmValue sequence) contains at least one context} \\
  &\splitctx(wv ~ \seq{wv}) =
    \begin{cases}
      (\epsilon, wv ~ \seq{wv}) &\quad\quad \premise{if}\quad ~ \isctx(wv) \\
      (wv ~ \seq{wv_0}, \seq{wv_1}) &\quad\quad \premise{if}\quad ~ \neg\isctx(wv) \quad\land\quad (\seq{wv_0}, \seq{wv_1}) = \splitctx(\seq{wv})
    \end{cases}
  \\
%
% execute
\newline \\
  &\execute(v, k) = \exe(v, k) \\
%
% exit_cont-wasm_0
\newline \\
&\text{NOTE: input cont of exit is either \algo ~ or \wasm} \\
  &\exit_{K}(\wasm ~ (\epsilon, k)) = \wasm ~ (\epsilon, \exit_{K}(k)) \\
% exit_cont-wasm_nonzero
  &\exit_{K}(\wasm ~ (c ~ \seq{c}, k)) = \wasm ~ (\seq{c}, k) \\
% exit_cont-al
  &\exit_{K}(\algo ~ (\seq i, \mu, \seq c, k)) = \algo ~ (\seq i, \mu, \seq c, \exit(k)) \\
%
% exit_wasm_instr-end
\newline \\
  &\exit_{WasmInstr}(wi ~ \seq{wi}) =
    \begin{cases}
      \seq{wi} &\quad\quad \premise{if}\quad ~ \isendinstr(wi) \\
      \exit_{WasmInstr}(\seq{wi}) &\quad\quad \premise{otherwise} \\
    \end{cases}
  \\
%
% exit_wasm_value
\newline \\
  &\exit_{WasmValue}(\seq{wv}) = \seq{wv_1}
  \quad\quad\premise{if}\quad ~ (\seq{wv_0}, \seq{wv_1}) = \splitctx(\seq{wv}) \\
%
% duplicate_env-empty
\newline \\
  &\duplicateenv(\mu, epsilon) = \mu \\
% duplicate_env-nonempty
  &\duplicateenv(\mu, s ~ \seq{s}) = \mu^{|\seq v|}
  \quad\quad\premise{if}\quad ~ \listv ~ \seq v = \mu(s) \\
%
% get_env
\newline \\
  &\getenv(\algo ~ (\seq i, \mu, \seq c, k)) = \mu \\
%
% set_env
\newline \\
  &\setenv(\algo ~ (\seq i, \mu, \seq c, k), \mu') = \algo ~ (\seq i, \mu', \seq c, k) \\
%
% prepend_instr
\newline \\
  &\prependinstr(\algo ~ (\seq i, \mu, \seq c, k), \seq{i'}) = \algo ~ (\seq{i'} ~ \seq i, \mu, \seq c, k) \\
%
% get_ctx
\newline \\
  &\getctx(\algo ~ (\seq i, \mu, \seq c, k)) = \seq c \\
%
% add_ctx
\newline \\
  &\addctx(\algo ~ (\seq i, \mu, \seq c, k), c') = \algo ~ (\seq i, \mu, c' ~ \seq c, k) \\
%
% clear_ctx
\newline \\
  &\clearctx(\algo ~ (\seq i, \mu, \seq c, k)) = \algo ~ (\seq i, \mu, \epsilon, k) \\
%
% get_store
\newline \\
  &\getstore((\seq{wv}, \seq{wi}, sto)) = sto \\
%
% set_store
\newline \\
  &\setstore((\seq{wv}, \seq{wi}, sto), sto') = (\seq{wv}, \seq{wi}, sto') \\
%
% pop_wasm_instr
\newline \\
  &\popwasminstr((\seq{wv}, wi ~ \seq{wi}, sto)) = ((\seq{wv}, \seq{wi}, sto), wi) \\
%
% push
\newline \\
  &\push((\seq{wv}, \seq{wi}, sto), wv) = (wv ~ \seq{wv}, \seq{wi}, sto) \\
%
% pop
\newline \\
  &\pop((wv ~ \seq{wv}, \seq{wi}, sto)) = ((\seq{wv}, \seq{wi}, sto), wv) \\
%
% popn
\newline \\
  &\popn((wv^n ~ \seq{wv}, \seq{wi}, sto), n) = ((\seq{wv}, \seq{wi}, sto), wv^n) \\
%
% unop
\newline \\
  &\unop(\notop, v) =
    \begin{cases}
      \boolv ~ true &\quad\quad\premise{if}\quad ~ \istrue(v) \\
      \boolv ~ false &\quad\quad\premise{otherwise} \\
    \end{cases}
  \\
  &\unop(\minusop, \numv ~ n) = \numv ~ (-n) \\
%
% get_cur_ctx
\newline \\
  &\getcurctx((\seq{wv}, \seq{wi}, sto)) =
  wv_{ctx} \quad\quad\premise{if}\quad ~ (\seq{wv_0}, wv_{ctx} ~ \seq{wv_1}) = \splitctx(\seq{wv}) \\
%
% get_cur_frame-frame
\newline \\
  &\getcurframe((\seq{wv}, \seq{wi}, sto)) =
    \begin{cases}
      wv_{ctx} &\quad\quad\premise{if}\quad
      (\seq{wv_0}, wv_{ctx} ~ \seq{wv_1}) = \splitctx(\seq{wv}) \quad\land\quad \isframe(wv_{ctx})\\
      \getcurframe(\seq{wv_1}) &\quad\quad\premise{if}\quad
      (\seq{wv_0}, wv_{ctx} ~ \seq{wv_1}) = \splitctx(\seq{wv}) \quad\land\quad \neg \isframe(wv_{ctx}) \\
    \end{cases}
  \\
%
% set_cur_frame-frame
\newline \\
  &\setcurframe((\seq{wv}, \seq{wi}, sto), v_{frame})
  =
  \begin{cases}
    (\seq{wv_0} ~ v_{frame} ~ \seq{wv_1}, \seq{wi}, sto) \\
    \qquad\qquad\qquad\quad\premise{if}\quad
    (\seq{wv_0}, wv_{ctx} ~ \seq{wv_1}) = \splitctx(\seq{wv}) \quad \isframe(wv_{ctx}) \\
    (\seq{wv_0} ~ wv_{ctx} ~ \setcurframe(\seq{wv_1}, v_{frame}), \seq{wi}, sto) \\
    \qquad\qquad\qquad\quad\premise{if}\quad
    (\seq{wv_0}, wv_{ctx} ~ \seq{wv_1}) = \splitctx(\seq{wv}) \quad \neg \isframe(wv_{ctx}) \\
  \end{cases}
  \\
%
% access-idx
\newline \\
  &\access(w, \mu, \listv ~ \seq{v}, \idxp ~ e) = \seq{v}[n]
  \qquad\qquad\qquad~\,\quad\premise{if}\quad w, \mu \vdash e \Rightarrow \numv ~ n \\
% access-slice
  &\access(w, \mu, \listv ~ \seq{v}, \slicep ~ (e_1, e_2)) = \seq{v}[n_1: n_2]
  \quad\quad\premise{if}\quad
  w, \mu \vdash e_1 \Rightarrow \numv ~ n_1 \quad\land\quad
  w, \mu \vdash e_2 \Rightarrow \numv ~ n_2 \\
% access-dot
  &\access(w, \mu, \strv ~ ((s_0, v_0) ~ \seq{(s_1, v_1)}), \dotp ~ s) =
  \begin{cases}
    v_0 &\quad\quad\premise{if}\quad s_0 = s \\
    \access(w, \mu, \strv ~ \seq{(s_1, v_1)}, \dotp ~ s) &\quad\quad\premise{otherwise} \\
  \end{cases}
  \\
% access-store
  &\access(w, \mu, \storev, p) =\access(w, \mu, \strv ~ sto, p)
  \quad\quad\premise{if}\quad w = (\seq{wv}, \seq{wi}, sto)\\
%
% update
\newline \\
  % idx
  &\update(w, \mu, \listv ~ \seq{v}, (\idxp ~ e) ~ \seq{p}, v) =
  \listv ~ (\updateidx(\seq v, n, \update(w, \mu, \seq{v}[n], \seq{p}, v))) \\
  &\qquad\qquad\qquad\qquad\qquad\qquad\qquad\qquad\qquad\qquad\qquad\qquad\qquad\qquad\qquad\qquad\qquad\qquad\qquad\qquad\premise{if}\quad
  w, \mu \vdash e \Rightarrow \numv ~ n \\
  % slice
  &\update(w, \mu, \listv ~ \seq{v}, (\slicep ~ e) ~ \seq{p}, v) =
  \listv ~ (\updateslice(\seq v, n_1, n_2, \update(w, \mu, \seq{v_i}[n_1: n_2], \seq{p}, v))) \\
  &\qquad\qquad\qquad\qquad\qquad\qquad\qquad\qquad\qquad\qquad\qquad\qquad\qquad\premise{if}\quad
  w, \mu \vdash e_1 \Rightarrow \numv ~ n_1 \quad\land\quad
  w, \mu \vdash e_2 \Rightarrow \numv ~ n_2 \\
  % dot
  &\update(w, \mu, \strv ~ \seq{(s, v)}, (\dotp ~ s') ~ \seq{p}, v)
  =
  \strv ~ (\updatedot(\seq{(s, v)}, s', \update(w, \mu, \access(v, \dotp ~ s'), \seq{p}, v))) \\
%
% update_idx
\newline \\
  &\updateidx(v_0 ~ \seq{v_1}, n, v_2) =
  \begin{cases}
    v_2 ~ \seq{v_1}
    &\quad\quad\premise{if}\quad n = 0 \\
    v_0 ~ \updateidx(\seq{v_1}, n-1, v_2)
    &\quad\quad\premise{if}\quad n > 0 \\
  \end{cases} \\
%
% update_slice
\newline \\
  &\updateslice(v_0^n ~ \seq{v_1}, 0, n, v_2^n) = v_2^n ~ \seq{v_1} \\
  &\updateslice(v_0 ~ \seq{v_1}, m, n, v_2^n) =  v_0 ~ \updateslice(\seq{v_1}, m-1, n, v_2)
  \quad\quad\premise{if}\quad m > 0 \\
%
% update_dot
\newline \\
  &\updatedot((s_0, v_0) ~ \seq{(s_1, v_1)}, s_2, v_2) =
  \begin{cases}
    (s_2, v_2) ~ \seq{(s_1, v_1)}
    &\quad\quad\premise{if}\quad s_0 = s_2 \\
    (s_0, v_0) ~ \updatedot(\seq{(s_1, v_1)}, s_2, v_2)
    &\quad\quad\premise{otherwise}
  \end{cases}
  \\
%
% get_end_algo
\newline \\
  &\getendinstr(v) =
  \begin{cases}
    \casev ~ (\text{"EXITING"}, \epsilon) &\quad\quad\premise{if}\quad v = \casev ~ (\text{"LABEL"}, \seq v)\\
    \casev ~ (\text{"RETURNING"}, \epsilon) &\quad\quad\premise{if}\quad v = \casev ~ (\text{"FRAME"}, \seq v)
  \end{cases} \\
%
% is_end_instr
\newline \\
  &\isendinstr(wi) =
  \begin{cases}
    true &\quad\quad\premise{if}\quad wi = \casev ~ (\text{"EXITING"}, \epsilon) \quad\lor\quad wi = \casev ~ (\text{"RETURNING"}, \epsilon)\\
    false &\quad\quad\premise{otherwise}
  \end{cases} \\
%
% is_true
\newline \\
  &\istrue(v) =
  \begin{cases}
  b &\quad\quad\premise{if}\quad v = \boolv ~ b \\
  \forall v \in \seq v. \istrue (v) &\quad\quad\premise{if}\quad v = \listv ~ \seq v \\
  \end{cases} \\
%
% is_frame
\newline \\
  &\isframe(v) =
  \begin{cases}
    true
    &\quad\quad\premise{if}\quad v = \casev ~ (\text{"FRAME"}, \seq v) \\
    false &\quad\quad\premise{otherwise}\\
  \end{cases}
  \\
%
% is_ctx
\newline \\
  &\isctx(v) =
  \begin{cases}
    true
    &\quad\quad\premise{if}\quad v = \casev ~ (\text{"FRAME"}, \seq v) \quad\lor\quad v = \casev ~ (\text{"LABEL"}, \seq v) \\
    false &\quad\quad\premise{otherwise}\\
  \end{cases}
%
% zip
\newline \\
  &\zip(eps, eps) = eps \\
  &\zip(x ~ \seq{x}, y ~ \seq{y}) = (x, y) ~ \zip (\seq{x}, \seq{y}) \\
%
% fold
\newline \\
  &\fold(f, x^n, acc) = f(x_0, f(x_1, ~ ... ~ f(x_{n-2}, f(x_{n-1}, acc)) ~ ... ~ ))
\end{align*}





