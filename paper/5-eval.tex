% !TEX root = main.tex

\chapter{Evaluation}
\label{ch:eval}
\noindent

We evaluate our approach with the following three research questions.
\begin{itemize}
  \item RQ1. (Correctness) Does the formalization correctly capture the Wasm
    control flow? (\S 5.1)
  \item RQ2. (Generality) Does the formalization capture the fundamental
    control flow semantics rather than being tailored to the current version of
    the specification? (\S 5.2)
  \item RQ3. \red{(Efficiency) Does our approach efficient? (or scalable?) (\S
    5.3)}
\end{itemize}

\section{RQ1: Correctness}
To demonstrate that our formalization correctly solves the problem, we revisit
\cref{ex:bug}.
The code below is a simplified version of Wasm interpreter program written in AL.
\begin{verbatim}
(* Top-level function: invoke *)
invoke funidx:
  PushCtxI Frame
  EnterI (Label(\epsilon), get_function_body(funidx))
  ReturnFunI get_return_values(funidx)

(* Wasm instructions *)
Br 0:
  PopCtxI Label(instr*)
  ExecuteI instr*

Unreachable:
  TrapI

(* Admin instructions *)
Exiting:
  PopCtxI Label(instr*)

Returning:
  PopCtxI Frame

\end{verbatim}
Using the semantics defined in \cref{ch:formal} together with the AL program,
\cref{ex:bug} produces the same as the \officialp{}; invoking the function has
no effect.
We adopt the simplified notation $wc \vdash wi^*$ for a Wasm state, $L$ for a
label and $F$ for a frame, as in the notation from \cref{sec:spectecp}.
Additionally, we omit the environment argument in \algo{} and the arguments of AL
instructions.

\[
\begin{array}{l}
  \epsilon \vdash \epsilon; ~ \toplevelcall(\text{``invoke"}, \numv ~ (\helper{funidx\_of}(\text{``\$brAndReturning"})))
  \\
  \leadsto \epsilon \vdash \epsilon; ~ \algo ~ (\pushctxi ~ \enteri ~ \returnfuni, ~ \epsilon, ~ \mt)
  \\
  \leadsto F \vdash \epsilon; ~ \algo ~ (\enteri ~ \returnfuni, ~ F, ~ \mt)
  \\
  \leadsto L ~ F \vdash (br ~ \epsilon) ~ (unreachable) ~ (exiting) ~ (returning); \wasm ~ (L ~ F, ~ \algo ~ (\returnfuni, ~ \epsilon, ~ \mt))
  \\
  \leadsto L ~ F \vdash (unreachable) ~ (exiting) ~ (returning); \\
  \quad\qquad\qquad\qquad\qquad \algo ~ (\popctxi ~ \executei, ~ \epsilon, ~ \wasm ~ (L ~ F, ~ \algo ~ (\returnfuni, ~ \epsilon, ~ \mt)))
  \\
  \leadsto F \vdash (returning); ~
    \algo ~ (\executei, ~ \epsilon, ~ \wasm ~ (F, ~ \algo ~ (\returnfuni, ~ \epsilon, ~ \mt)))
  \\
  \leadsto F \vdash (returning); ~
    \algo ~ (\epsilon, ~ \epsilon, ~ \wasm ~ (F, ~ \algo ~ (\returnfuni, ~ \epsilon, ~ \mt)))
  \\
  \leadsto F \vdash (returning); ~ \wasm ~ (F, ~ \algo ~ (\returnfuni, ~ \epsilon, ~ \mt))
  \\
  \leadsto F \vdash \epsilon; ~
    \algo ~ (\popctxi, ~ \epsilon, ~ \wasm ~ (\epsilon, ~ \algo ~ (\returnfuni, ~ \epsilon, ~ \mt)))
  \\
  \leadsto \epsilon \vdash \epsilon; ~
    \algo ~ (\epsilon, ~ \epsilon, ~ \wasm ~ (\epsilon, ~ \algo ~ (\returnfuni, ~ \epsilon, ~ \mt)))
  \\
  \leadsto \epsilon \vdash \epsilon; ~ \wasm ~ (\epsilon, ~ \algo ~ (\returnfuni, ~ \epsilon, ~ \mt))
  \\
  \leadsto \epsilon \vdash \epsilon; ~ \algo ~ (\returnfuni, ~ \epsilon, ~ \mt)
  \\
  \leadsto \epsilon \vdash \epsilon; ~ \ret ~ (\epsilon, ~ \mt)
  \\
\end{array}
\]

Furthermore, we implement AL interpreter according to the formalization to
evaluate its correctness.
We write the Wasm specification using the DSL and generat a Wasm interpreter
program written in AL.
By executing the program with the AL interpreter against the official Wasm test
suite, we indirectly test the AL interpreter.
We test it using execution tests from official Wasm test suite.
Tests related to inifinite loop detection are excluded, as the specification
does not define this behavior.
The result shows that the interpreter successfully pass all 51,549 tests in 70
seconds on \red{(lambda 9)}.


\section{RQ2: Generality}
The Wasm exception handler proposal introduces a new control structure,
\textit{handler}, along with new control instructions.
In our formalization, we initially assumed that a context is either a label or
a frame.
By simply extending the formalization to include a handler case, as shown
below, AL can effectively describe the exception handler semantics:

\begin{align*}
%
% Wasm Context
  \text{Wasm Context}\quad ~ WC ~\ni~ wc ~::=~
    ~ \casev ~ (\text{``Label"}, \seq v)
    ~ | ~ \casev ~ (\text{``Frame"}, \seq v)
    ~ | ~ \casev ~ (\text{``Handler"}, \seq v) \\
\end{align*}
\begin{align*}
%
% get_end_algo
\newline \\
  &\getendinstr(wc) =
  \begin{cases}
    \casev ~ (\text{``Exiting"}, \epsilon) &\quad\quad\premise{if}\quad wc = \casev ~ (\text{``Label"}, \seq v) \\
    \casev ~ (\text{``Returning"}, \epsilon) &\quad\quad\premise{if}\quad wc = \casev ~ (\text{``Frame"}, \seq v) \\
    \casev ~ (\text{``Exiting-handler"}, \epsilon) &\quad\quad\premise{if}\quad wc = \casev ~ (\text{``Handler"}, \seq v) \\
  \end{cases} \\
%
% is_end_instr
\newline \\
  &\isendinstr(wi) =
  \begin{cases}
    true &\quad\quad\premise{if}\quad wi = \casev ~ (\text{``Exiting"}, \epsilon) \\
    &\quad\quad\lor\quad
    wi = \casev ~ (\text{``Returning"}, \epsilon) \\
    &\quad\quad\lor\quad
    wi = \casev ~ (\text{``Exiting-handler"}, \epsilon) \\
    false &\quad\quad\premise{otherwise}
  \end{cases} \\
\end{align*}

To evaluate the extended formalization, we updated the AL interpreter
accordingly.
We incorporated the exception handler feature into the specification and
generated a Wasm interpreter.
The interpreter successfully passed all 51,614 execution tests in the exception handlers
proposal, excluding infinite loop tests, in 116 seconds.

\section{\red{RQ3: Efficiency}}
The Wasm tail call proposal introduces tail call optimization so that it
contains tests with heavy recursive calls.
As a result, running the test with previous AL interpreter results in a call
stack overflow.
To evaluate the \red{efficiency} of our approach, we similarly generate a Wasm
interpreter including tail call features.
It successfully passes all 51,639 execution tests in the tail call proposal,
excluding infinite loop tests, in 1,284 seconds.
