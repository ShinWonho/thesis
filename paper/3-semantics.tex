% !TEX root = main.tex

\chapter{Semantics}
\label{ch:semantics}
\noindent


% AL semantics formalize
%
%
% Prose notation:
% - Implicit when to exit:
%   • Returning from a function
%     When the end of a function is reached without a jump (i.e., return) or trap aborting it, then the following steps are performed.
%
%   • Exiting with label
%     When the end of a block is reached without a jump or trap aborting it, then the following steps are performed.
%
%   ! end of function/label is defined verbally, not in algorithmic notation
%
% - Separate context and instructions:
%   • Push ctx \& Jump instrs
%
%
% abstract machine of prose notation: vm
% 1. state: stack, store
% 2. cosumes sequence of WebAssembly instruction
% 3. mutate state
%
% - store
%   - instantiate -> alloc
%   - invoke -> use
% - stack contains value \& context
%   - value
%     ex: binop
%   - context
%     1) label: allow structured control flow (continuation)
%     2) frame: store call information (local variable, module instance)
%     - enter \& exit
%     - ex: loop, invoke
%     - ex: br, implicit exit
%
%
% - wrong interpretation of "end" in AL interpreter \red{(How long does it occur?)}
% - example


\newcommand{\cat}{\texttt{++}}



%% Symbol definition

\renewcommand{\symbol}[1]{\textbf{#1}}

% Continuation
\newcommand{\mt}{\symbol{Empty}}
\newcommand{\call}{\symbol{Call}}
\newcommand{\al}{\symbol{Al}}
\newcommand{\enter}{\symbol{Enter}}
\newcommand{\wasm}{\symbol{Wasm}}
\newcommand{\exe}{\symbol{Execute}}
\newcommand{\ret}{\symbol{Return}}

% Instruction
\newcommand{\ifi}{\symbol{IfI}}
\newcommand{\eitheri}{\symbol{EitherI}}
\newcommand{\enteri}{\symbol{EnterI}}
\newcommand{\pushctxi}{\symbol{PushCtxI}}
\newcommand{\pushi}{\symbol{PushI}}
\newcommand{\popi}{\symbol{PopI}}
\newcommand{\popni}{\symbol{PopNI}}
\newcommand{\popalli}{\symbol{PopAllI}}
\newcommand{\leti}{\symbol{LetI}}
\newcommand{\trapi}{\symbol{TrapI}}
\newcommand{\nopi}{\symbol{NopI}}
\newcommand{\returni}{\symbol{ReturnI}}
\newcommand{\executei}{\symbol{ExecuteI}}
\newcommand{\executeseqi}{\symbol{ExecuteSeqI}}
\newcommand{\performi}{\symbol{PerformI}}
\newcommand{\exiti}{\symbol{ExitI}}
\newcommand{\replacei}{\symbol{ReplaceI}}

% Expression
\newcommand{\vare}{\symbol{VarE}}
\newcommand{\nume}{\symbol{NumE}}
\newcommand{\boole}{\symbol{BoolE}}
\newcommand{\fnamee}{\symbol{FnameE}}
\newcommand{\une}{\symbol{UnE}}
\newcommand{\bine}{\symbol{BinE}}
\newcommand{\acce}{\symbol{AccE}}
\newcommand{\upde}{\symbol{UpdE}}
\newcommand{\stre}{\symbol{StrE}}
\newcommand{\compe}{\symbol{CompE}}
\newcommand{\cate}{\symbol{CatE}}
\newcommand{\meme}{\symbol{MemE}}
\newcommand{\lene}{\symbol{LenE}}
\newcommand{\tupe}{\symbol{TupE}}
\newcommand{\casee}{\symbol{CaseE}}
\newcommand{\calle}{\symbol{CallE}}
\newcommand{\itere}{\symbol{IterE}}
\newcommand{\liste}{\symbol{ListE}}
\newcommand{\getcurctxe}{\symbol{GetCurContextE}}
\newcommand{\choosee}{\symbol{ChooseE}}
\newcommand{\iscaseofe}{\symbol{IsCaseOfE}}
\newcommand{\isvalide}{\symbol{IsValidE}}
\newcommand{\ctxkinde}{\symbol{CtxKindE}}
\newcommand{\matche}{\symbol{MatchE}}
\newcommand{\hastypee}{\symbol{HasTypeE}}

% Unary operator
\newcommand{\notop}{\symbol{NotOp}}
\newcommand{\minusop}{\symbol{MinusOp}}

% Binary operator
\newcommand{\addop}{\symbol{AddOp}}
\newcommand{\subop}{\symbol{SubOp}}
\newcommand{\mulop}{\symbol{MulOp}}
\newcommand{\divop}{\symbol{DivOp}}
\newcommand{\modop}{\symbol{ModOp}}
\newcommand{\expop}{\symbol{ExpOp}}
\newcommand{\implop}{\symbol{ImplOp}}
\newcommand{\equivop}{\symbol{EquivOp}}
\newcommand{\andop}{\symbol{AndOp}}
\newcommand{\orop}{\symbol{OrOp}}
\newcommand{\eqop}{\symbol{EqOp}}
\newcommand{\neop}{\symbol{NeOp}}
\newcommand{\ltop}{\symbol{LtOp}}
\newcommand{\gtop}{\symbol{GtOp}}
\newcommand{\leop}{\symbol{LeOp}}
\newcommand{\geop}{\symbol{GeOp}}

% Path
\newcommand{\idxp}{\symbol{Idx}}
\newcommand{\slicep}{\symbol{Slice}}
\newcommand{\dotp}{\symbol{Dot}}

% Iter
\newcommand{\listiter}{\symbol{List}}
\newcommand{\listniter}{\symbol{ListN}}
\newcommand{\listidxiter}{\symbol{Index}}

% Value
\newcommand{\numv}{\symbol{NumV}}
\newcommand{\boolv}{\symbol{BoolV}}
\newcommand{\listv}{\symbol{ListV}}
\newcommand{\strv}{\symbol{StrV}}
\newcommand{\casev}{\symbol{CaseV}}
\newcommand{\tupv}{\symbol{TupV}}
\newcommand{\fnamev}{\symbol{FnameV}}
\newcommand{\trapv}{\symbol{TrapV}}




%% Helper function

\newcommand{\helper}[1]{\text{#1}}

% Helper function
\newcommand{\execute}{\helper{execute}}
\newcommand{\getenv}{\helper{get\_env}}
\newcommand{\setenv}{\helper{set\_env}}
\newcommand{\prependinstr}{\helper{prepend\_instr}}
\newcommand{\increasectxcnt}{\helper{increase\_ctx\_cnt}}
\newcommand{\push}{\helper{push}}
\newcommand{\pop}{\helper{pop}}
\newcommand{\popn}{\helper{popn}}
\newcommand{\exit}{\helper{exit}}
\newcommand{\splitctx}{\helper{split\_ctx}}
\newcommand{\getcurctx}{\helper{get\_cur\_ctx}}
\newcommand{\getcurframe}{\helper{get\_cur\_frame}}
\newcommand{\setcurframe}{\helper{set\_cur\_frame}}
\newcommand{\istrue}{\helper{is\_true}}
\newcommand{\duplicateenv}{\helper{duplicate\_env}}
% TODO
\newcommand{\lookup}{\helper{lookup}}
\newcommand{\assign}{\helper{assign}}
\newcommand{\iswasmvalueinstr}{\helper{is\_wasm\_value\_instr}}
\newcommand{\foldright}{\helper{fold\_right}}
\newcommand{\fold}{\helper{fold}}
\newcommand{\isframe}{\helper{is\_frame}}
\newcommand{\isstore}{\helper{is\_store}}
\newcommand{\update}{\helper{update}}
\newcommand{\unop}{\helper{unop}}
\newcommand{\binop}{\helper{binop}}
\newcommand{\access}{\helper{access}}
\newcommand{\isvalid}{\helper{is\_valid}}
\newcommand{\match}{\helper{match}}
\newcommand{\hastype}{\helper{has\_type}}
\newcommand{\zip}{\helper{zip}}




%% Syntax

\begin{align*}
%
% State
  s \in S =& W \times Cont \\
%
% Wasm state
  w \in W =& \seq{WasmValue} \times \seq{WasmInstr} \times Store \\
  wv \in WasmValue =& V \\
  wi \in WasmInstr =& V \\
%
% Store
  sto \in Store =& String \mapsto \seq V \\
%
% Continuation
  c \in Cont =& ~ \mt \\
    | & ~ \call ~ String \times \seq V \times Cont \\
    | & ~ \al ~ \seq I \times Env \times \mathbb N \times Cont \\
    | & ~ \wasm ~ \mathbb N \times Cont \\
    | & ~ \exe ~ wi \times Cont \\
    | & ~ \ret ~ V \\
%
% Environment
  \mu \in Env =& String \mapsto V \\
%
% Instruction
  i \in I =& ~ \ifi ~ E \times \seq I \times \seq I \\
    | & ~ \eitheri ~ \seq I \times \seq I \\
    | & ~ \enteri ~ E \times E \\
    | & ~ \pushctxi ~ E \\
    | & ~ \pushi ~ E \\
    | & ~ \popi ~ E \\
    | & ~ \popni ~ E \times E \\
    | & ~ \popalli ~ E \\
    | & ~ \leti ~ E \times E \\
    | & ~ \trapi \\
    | & ~ \returni ~ E_{\bot} \\
    | & ~ \executei ~ E \\
    | & ~ \executeseqi ~ E \\
    | & ~ \performi ~ String \times \seq{E} \\
    | & ~ \exiti ~ String \\
    | & ~ \replacei ~ E \times \seq P \times E \\
%
% Expression
  e \in E =& ~ \vare ~ String \\
    | & ~ \nume ~ \mathbb N \\
    | & ~ \boole ~ \mathbb B \\
    | & ~ \fnamee ~ String \\
    | & ~ \une ~ Unop \times E \\
    | & ~ \bine ~ Binop \times E \times E \\
    | & ~ \acce ~ E \times P \\
    | & ~ \upde ~ E \times \seq P \times E \\
    | & ~ \stre ~ \seq{(String \times E)} \\
    | & ~ \compe ~ E \times E \\
    | & ~ \cate ~ E \times E \\
    | & ~ \meme ~ E \times E \\
    | & ~ \lene ~ E \\
    | & ~ \tupe ~ \seq E \\
    | & ~ \casee ~ String \times \seq E \\
    | & ~ \calle ~ String \times \seq E \\
    | & ~ \itere ~ E \times Iter \times \seq{String} \\
    | & ~ \liste ~ \seq E \\
    | & ~ \getcurctxe ~ String \\
    | & ~ \choosee ~ E \\
    | & ~ \iscaseofe ~ E \times String \\
    | & ~ \ctxkinde ~ String \\
    | & ~ \matche ~ E \times E \\
    | & ~ \hastypee ~ E \times String \\
%
% Operator
  unop \in Unop =& ~ \notop ~ | ~ \minusop \\
  binop \in Binop =& ~ \addop \\
    | & ~ \subop \\
    | & ~ \mulop \\
    | & ~ \divop \\
    | & ~ \modop \\
    | & ~ \expop \\
    | & ~ \implop \\
    | & ~ \equivop \\
    | & ~ \andop \\
    | & ~ \orop \\
    | & ~ \eqop \\
    | & ~ \neop \\
    | & ~ \ltop \\
    | & ~ \gtop \\
    | & ~ \leop \\
    | & ~ \geop \\
%
% Path
  p \in P =& ~ \idxp ~ E ~ | ~ \slicep ~ E \times E ~ | ~ \dotp ~ String \\
%
% Iter
  iter \in Iter =& ~ \listiter ~ | ~ \listniter ~ E ~|~ \listidxiter ~ String \times E \\
%
% Value
  v \in V =& ~ \numv ~ \mathbb N \\
    | & ~ \boolv ~ \mathbb B \\
    | & ~ \listv ~ \seq V \\
    | & ~ \strv ~ \seq{(String \times V)} \\
    | & ~ \casev ~ String \times \seq V \\
    | & ~ \tupv ~ \seq V \\
    | & ~ \fnamev String \\
\end{align*}




%% Continuation

\newpage

\begin{gather*}
\newline \\
  S \leadsto S
%
% Call
\newline \\
  \seq e, \seq i = \lookup(str) \quad \mu = \assign(\seq e, \seq v) \\
  \hline
  (w, \call ~ (str, \seq v, c)) \leadsto (w, \al ~ (\seq i, \mu, c)) \\
%
% Al-empty_zero
\newline \\
  \hline
  (w, \al ~ (\epsilon, \mu, 0, c)) \leadsto (w, c) \\
%
% Al-empty_nonzero
\newline \\
  n > 0 \\
  \hline
  (w, \al ~ (\epsilon, \mu, n, c)) \leadsto (w, \wasm ~ (n, c)) \\
%
% Al-instr
\newline \\
  \hline
  (w, \al ~ (i_0 ~ \seq{i_1}, \mu, n, c)) \leadsto \sem{i_0}(w, \al ~ (\seq{i_1}, \mu, n, c)) \\
%
% Wasm-zero
\newline \\
  \hline
  (w, \wasm ~ (0, c)) \leadsto (w, c) \\
%
% Wasm-value
\newline \\
  n > 0 \quad \iswasmvalueinstr(wi) \\
  \hline
  ((wi ~ \seq{wi'}, \seq{wv}, sto), \wasm ~ (n, c))
  \leadsto
  ((\seq{wi'}, wi ~ \seq{wv}, sto), \wasm ~ (n, c)) \\
%
% Wasm-instr
\newline \\
  n > 0 \quad \neg \iswasmvalueinstr(wi) \\
  \casev ~ (str, \seq v) = wi \quad
  \seq e, \seq i = \lookup(str) \quad
  \mu = \assign(\epsilon, \seq e, \seq v) \\
  \hline
  ((wi ~ \seq{wi'}, \seq{wv}, sto), \wasm ~ (n, c))
  \leadsto
  ((\seq{wi'}, \seq{wv}, sto), \al ~ (\seq i, \seq v, \wasm ~ (n, c))) \\
%
% Execute-value
\newline \\
  \iswasmvalueinstr(wi') \\
  \hline
  ((\seq{wi}, \seq{wv}, sto), \exe ~ (wi', c))
  \leadsto
  ((\seq{wi}, wi' ~ \seq{wv}, sto), c) \\
%
% Execute-instr
\newline \\
  \neg \iswasmvalueinstr(wi') \\
  \casev ~ (str, \seq v) = wi \quad
  \seq e, \seq i = \lookup(str) \quad
  \mu = \assign(\epsilon, \seq e, \seq v) \\
  \hline
  (w, \exe ~ (wi', c)) \leadsto (w, \al ~ (\seq i, \seq v, c)) \\
\end{gather*}




%% Instruction

\newpage

\begin{gather*}
  \text{NOTE: input cont is always \al} \\
  \sem{i} : State \rightarrow State \\
%
% If-true
\newline \\
  \mu = \getenv(c) \quad v = \sem{e}(\mu, w) \quad
  \istrue(v) \quad c' = \prependinstr(c, \seq{i_1}) \\
  \hline
  \sem{\ifi ~ (e, \seq{i_1}, \seq{i_2})}(w, c) = (w, c') \\
%
% If-false
\newline \\
  \mu = \getenv(c) \quad v = \sem{e}(\mu, w) \quad
  \neg \istrue(v) \quad c' = \prependinstr(c, \seq{i_2}) \\
  \hline
  \sem{\ifi ~ (e, \seq{i_1}, \seq{i_2})}(w, c) = (w, c') \\
%
% Either-1
\newline \\
  \hline
  \sem{\eitheri ~ (\seq{i_1}, \seq{i_2})}(w, c) = (w, \prependinstr(c, \seq{i_1})) \\
%
% Either-2
\newline \\
  \hline
  \sem{\eitheri ~ (\seq{i_1}, \seq{i_2})}(w, c) = (w, \prependinstr(c, \seq{i_2})) \\
%
% Enter
\newline \\
  \mu = \getenv(c) \quad v_1 = \sem{e_1}(\mu, w) \quad \listv ~ \seq{v_2} = \sem{e_2}(\mu, w) \\
  \hline
  \sem{\enteri ~ (e_1, e_2)}((\seq{wv}, \seq{wi}, sto), c)
  =
  ((v_1 ~ \seq{wv}, \seq{v_2} ~ \seq{wi}, sto), \wasm ~ (n+1, c)) \\
%
% PushCtx
\newline \\
  \mu = \getenv(c) \quad v = \sem{e}(\mu, w) \\
  \hline
  \sem{\pushctxi ~ e}(w, c)
  =
  (\push_{WasmValue}(w, v), \increasectxcnt(c)) \\
%
% Push
\newline \\
  v = \sem{e}(\mu, w) \\
  \hline
  \sem{\pushi ~ e}(w, c) = (\push_{WasmValue}(w, v), c) \\
%
% Pop
\newline \\
  \mu = \getenv(c) \quad
  wv, w' = \pop_{WasmValue}(w) \quad
  \mu' = \assign(\mu, e, wv) \\
  \hline
  \sem{\popi ~ e}(w, c) = (w', \setenv(c, \mu')) \\
%
% PopN
\newline \\
  \mu = \getenv(c) \quad
  \sem{e_2}(\mu, w) = \numv ~ n \quad
  wv^n, w' = \popn_{WasmValue}(w, n) \quad
  \mu' = \assign(\mu, e_1, \listv ~ wv^n) \\
  \hline
  \sem{\popni ~ (e_1, e_2)}(w, c) = (w', \setenv(c, \mu')) \\
%
% PopAll
\newline \\
  \mu = \getenv(c) \quad
  \seq{wv_0}, \seq{wv_1} = \splitctx(wv) \quad
  \mu' = \assign(\mu, e, \listv ~ \seq{wv_0}) \\
  \hline
  \sem{\popalli ~ e}(\seq{wv}, \seq{wi}, sto), c)
  =
  ((\seq{wv_1}, \seq{wi}, sto), \setenv(c, \mu')) \\
%
% Let
\newline \\
  \mu = \getenv(c) \quad
  v = \sem{e_2}(\mu, w) \quad
  \mu' = \assign(\mu, e_1, v) \\
  \hline
  \sem{\leti ~ (e_1, e_2)}(w, c)
  =
  (w, \setenv(c, \mu')) \\
%
% Trap
\newline \\
  \hline
  \sem{\trapi}(w, c) = (w, \ret ~ \trapv) \\
%
% Return-none
\newline \\
  \hline
  \sem{\returni}(w, \al~ (\seq{i}, \mu, n, c)) = (w, c) \\
%
% Return-some
\newline \\
  \sem{e}(\mu, w) = v \\
  \hline
  \sem{\returni ~ e}(w, \al ~ (\seq{i}, \mu, n, c)) = (w, \ret ~ v) \\
%
% Execute
\newline \\
  \mu = \getenv(c) \quad
  \sem{e}(\mu, w) = v \\
  \hline
  \sem{\executei ~ e}(w, c) = (w, \exe ~ (v, c)) \\
%
% ExecuteSeq
\newline \\
  \mu = \getenv(c) \quad
  \sem{e}(\mu, w) = \listv ~ \seq v \quad
  c' = \foldright(\execute, \seq v, c) \\
  \hline
  \sem{\executeseqi ~ e}(w, c) = (w, c') \\
%
% Perform
\newline \\
  \mu = \getenv(c) \quad
  \seq v = \seq{\sem{e}}(\mu, w) \\
  \hline
  \sem{\performi ~ (str, \seq e)}(w, c) = (w, \call ~ (str, \seq v, c)) \\
%
% Exit
\newline \\
  \seq{wv'} = \exit_{WasmValue}(\seq{wv}) \quad
  c' = \exit_{Cont}(c) \quad
  \seq{wi'} = \exit_{WasmInstr}(\seq{wi}) \\
  \hline
  \sem{\exiti}((\seq{wv}, \seq{wi}, sto), c)
  =
  ((\seq{wv'}, \seq{wi'}, sto), c') \\
%
% Replace-frame
\newline \\
\text{NOTE: $e_1$ is either store or current frame} \\
  \isframe(e_1) \quad
  \mu = \getenv(c) \quad
  v_2 = \sem{e_2}(\mu, w) \quad
  v_{frame} = \getcurframe(w) \quad
  v_{frame}' = \update(v_{frame}, \seq p, v_2) \\
  \hline
  \sem{\replacei ~ (e_1, \seq p, e_2)}(w, c)
  = (\setcurframe(w, v_{frame}', c) \\
%
% Replace-store
\newline \\
  \isstore(e_1) \quad
  \mu = \getenv(c) \quad
  v_2 = \sem{e_2}(\mu, w) \quad
  sto' = \update_{Sto}(sto, \seq p, v_2) \\
  \hline
  \sem{\replacei ~ (e_1, \seq p, e_2)}((\seq{wv}, \seq{wi}, sto), c)
  = ((\seq{wv}, \seq{wi}, sto'), c) \\
\newline \\
\end{gather*}





%% Expression

\begin{gather*}
  \sem{e} : Env \times W \rightarrow V \\
%
% Var
\newline \\
  v = \mu(str) \\
  \hline
  \sem{\vare ~ str}(\mu, w) = v \\
%
% Num
\newline \\
  \hline
  \sem{\nume ~ n}(\mu, w) = \numv ~ n \\
%
% Bool
\newline \\
  \hline
  \sem{\boole ~ b}(\mu, w) = \boolv ~ b \\
%
% Fname
\newline \\
  \hline
  \sem{\fnamee ~ str}(\mu, w) = \fnamev ~ str \\
%
% Un
\newline \\
  v = \sem{e}(\mu, w) \\
  \hline
  \sem{\une ~ (unop, e)}(\mu, w) = \unop(unop, v) \\
%
% Bin
\newline \\
  v_1 = \sem{e_1}(\mu, w) \quad v_2 = \sem{e_2}(\mu, w) \\
  \hline
  \sem{\bine ~ (binop, e_1, e_2)}(\mu, w) = \binop(binop, v_1, v_2) \\
%
% Acc
\newline \\
  v = \sem{e}(\mu, w) \\
  \hline
  \sem{\acce ~ (e, p)}(\mu, w) = \access(v, p) \\
%
% Upd
\newline \\
  v_1 = \sem{e_1}(\mu, w) \quad v_2 = \sem{e_2}(\mu, w) \\
  \hline
  \sem{\upde ~ (e_1, \seq p, e_2)}(\mu, w) = \update(v_1, \seq p, v_2) \\
%
% Str
\newline \\
  \seq{(str, v)} = [ ~ str, \sem{e}(\mu, w) ~ | ~  (str, e) \leftarrow \seq{(str, e)} ~ ] \\
  \hline
  \sem{\stre ~ \seq{(str, e)}}(\mu, w) = \strv ~ \seq{(str, v)} \\
%
% Comp
\newline \\
  \strv ~ \seq{(str_1, v_1)} = \sem{e_1}(\mu, w) \quad
  \strv ~ \seq{(str_2, v_2)} = \sem{e_2}(\mu, w) \\
  \hline
  \sem{\compe ~ (e_1, e_2)}(\mu, w) = \strv ~ (\seq{(str_1, v_1)} ~ \seq{(str_2, v_2)}) \\
%
% Cat
\newline \\
  \listv ~ \seq{v_1} = \sem{e_1}(\mu, w) \quad
  \listv ~ \seq{v_2} = \sem{e_2}(\mu, w) \\
  \hline
  \sem{\cate ~ (e_1, e_2)}(\mu, w) = \listv ~ (\seq{v_1} ~ \seq{v_1}) \\
%
% Mem
\newline \\
  v_1 = \sem{e_1}(\mu, w) \quad
  \listv ~ \seq{v_2} = \sem{e_2}(\mu, w) \\
  \hline
  \sem{\meme ~ (e_1, e_2)}(\mu, w) = \boolv ~ (v_1 \in \seq{v_2}) \\
%
% Len
\newline \\
  \listv ~ \seq{v} = \sem{e}(\mu, w) \\
  \hline
  \sem{\lene ~ e}(\mu, w) = \numv ~ |\seq v| \\
%
% Tup
\newline \\
  \seq{v} = [ ~ \sem{e}(\mu, w) ~ | ~ e \leftarrow \seq e ~ ] \\
  \hline
  \sem{\tupe ~ \seq{e}}(\mu, w) = \tupv ~ \seq{v} \\
%
% Case
\newline \\
  \seq{v} = [ ~ \sem{e}(\mu, w) ~ | ~ e \leftarrow \seq e ~ ] \\
  \hline
  \sem{\casee ~ (str, \seq{e})}(\mu, w) = \casev ~ (str, \seq{v}) \\
%
% Call
\newline \\
  \seq{v} = [ ~ \sem{e}(\mu, w) ~ | ~ e \leftarrow \seq e ~ ] \quad
  (w, \call ~ (str, \seq v, \mt)) \seq{\leadsto} (w, \ret ~ v') \\
  \hline
  \sem{\calle ~ (str, \seq{e})}(\mu, w) = v' \\
%
% Iter-
\newline \\
  f =
    (\lambda ~ str ~ acc. ~
      \listv ~ \seq v = \mu(str); ~
      [ ~ \mu_1[str \mapsto v] ~ | ~ (v, \mu_1) \leftarrow \zip (\seq v, acc) ~ ]
    )
  \\
  \seq{\mu_2} = \fold(f, \seq{str}, \duplicateenv(\mu, \seq{str})) \quad
  \seq{v'} = [ ~ \sem{e}(\mu_3, w) ~ | ~ \mu_3 \leftarrow \seq {\mu_2} ~ ] \\
  \hline
  \sem{\itere ~ (e, \listiter, \seq{str})}(\mu, w) = \listv ~ \seq{v'} \\
%
% Iter-n
\newline \\
  \numv ~ n = \sem{e_2}(\mu, w) \\
  f_i =
    (\lambda ~ x ~ acc. ~
      \listv ~ \seq v = \mu(x); ~
      [ ~ \mu_1[str \mapsto v] ~ | ~ (v, \mu_1) \leftarrow \zip (\seq v, acc) ~ ]
    )
  \\
  \seq{\mu_2} = \fold(f, \seq{str}, \mu^n) \quad
  \seq{v'} = [ ~ \sem{e_1}(\mu_3, w) ~ | ~ \mu_3 \leftarrow \seq {\mu_2} ~ ] \\
  \hline
  \sem{\itere ~ (e_1, \listniter ~ e_2, \seq{str})}(\mu, w) = \listv ~ \seq{v'} \\
%
% Iter-idx
\newline \\
  \seq{\mu_1} =
    [ ~
      \fold(
        (\lambda ~ x ~ acc. ~ acc[x \mapsto \numv ~ i]),
        \seq{str},
        \mu
      )
    ~ | ~
      i \leftarrow [0 ~ .. ~ n-1]
    ~ ]
  \quad
  \seq{v'} = [ ~ \sem{e_1}(\mu_2, w) ~ | ~ \mu_2 \leftarrow \seq {\mu_1} ~ ] \\
  \hline
  \sem{\itere ~ (e_1, \listidxiter ~ e_2, \seq{str})}(\mu, w) = \listv ~ \seq{v'} \\
%
% List
\newline \\
  \seq{v} = [ ~ \sem{e}(\mu, w) ~ | ~ e \leftarrow \seq e ~ ] \\
  \hline
  \sem{\liste ~ \seq{e}}(\mu, w) = \listv ~ \seq{v} \\
%
% GetCurCtx
\newline \\
  v_{ctx} = \getcurctx(w, str) \\
  \hline
  \sem{\getcurctxe ~ str}(\mu, w) = v_{ctx} \\
%
% Choose
\newline \\
  \listv ~ \seq{v} = \sem{e}(\mu, w) \quad
  v' \in \seq{v} \\
  \hline
  \sem{\choosee ~ e}(\mu, w) = v' \\
%
% IsCaseOf
\newline \\
  \casev ~ (str', \seq{v}) = \sem{e}(\mu, w) \\
  \hline
  \sem{\iscaseofe ~ (e, str)}(\mu, w) = \boolv (str = str') \\
%
% IsValid
\newline \\
  v = \sem{e}(\mu, w) \\
  \hline
  \sem{\isvalide ~ e}(\mu, w) = \boolv (\isvalid(v)) \\
%
% CtxKind
\newline \\
  \casev (str', \seq v) = \getcurctx(w, str) \\
  \hline
  \sem{\ctxkinde ~ str}(\mu, w) = \boolv (str = str') \\
%
% Match
\newline \\
  v_1 = \sem{e_1}(\mu, w) \quad
  v_2 = \sem{e_2}(\mu, w) \\
  \hline
  \sem{\matche ~ (e_1, e_2)}(\mu, w) = \boolv (\match(v_1, v_2)) \\
%
% HasType
\newline \\
  v_= \sem{e}(\mu, w) \\
  \hline
  \sem{\hastypee ~ (e, str)}(\mu, w) = \boolv (\hastype(v, str)) \\
\end{gather*}




%% Helper functions

\begin{align*}
&\text{NOTE: an input of is\_ctx (WasmValue sequence) contains at least one context} \\
%
% pop_all-ctx
\newline \\
  &\text{is\_ctx}(\text{hd}(\seq{wv})) \\
  \hline
  &\splitctx(\seq{wv}) = \epsilon, \seq{wv} \\
%
% pop_all-val
  &\text{is\_not\_ctx}(wv_0) \quad \seq{wv_2}, \seq{wv_3} = \splitctx(\seq{wv_1}) \\
  \hline
  &\splitctx(wv_0 ~ \seq{wv_1}) = wv_0 ~ \seq{wv_2}, \seq{wv_3} \\
%
% execute
\newline \\
  &\execute(v, c) = \exe(v, c) \\
%
% exit_cont-wasm_0
\newline \\
&\text{NOTE: input cont of exit is either \al ~ or \wasm} \\
  &\exit_{Cont}(\wasm (0, c)) = \wasm(0, \exit_{Cont}(c)) \\
%
% exit_cont-wasm_nonzero
  &n > 0 \\
  \hline
  &\exit_{Cont}(\wasm (n, c)) = \wasm(n-1, c) \\
%
% exit_cont-al
  &\exit_{Cont}(\al (\seq i, \mu, n, c)) = \al (\seq i, \mu, n, \exit(c)) \\
%
% exit_wasm_instr-end
\newline \\
  &\text{is\_end}(wi) \\
  \hline
  &\exit_{WasmInstr}(wi ~ \seq{wi'}) = \seq{wi'} \\
%
% exit_wasm_instr-normal_instr
  &\text{is\_not\_end}(wi) \\
  \hline
  &\exit_{WasmInstr}(wi ~ \seq{wi'}) = \exit_{WasmInstr}(\seq{wi'}) \\
%
% exit_wasm_value
  &\seq{wv_1}, \seq{wv_2} = \splitctx(\seq{wv_0}) \\
  \hline
  &\exit_{WasmValue}(\seq{wv_0}) = \text{tail}(\seq{wv_2}) \\
%
% duplicate_env-empty
\newline \\
  &\duplicateenv(\mu, epsilon) = \mu \\
%
% duplicate_env-nonempty
  &\listv ~ \seq v = \mu(str) \\
  \hline
  &\duplicateenv(\mu, str ~ \seq{str'}) = \mu^{|\seq v|} \\
%
% get_env
\newline \\
  &\getenv(\al ~ (\seq i, \mu, n, c)) = \mu \\
%
% set_env
\newline \\
  &\setenv(\al ~ (\seq i, \mu, n, c), \mu') = \al ~ (\seq i, \mu', n, c) \\
%
% prepend_instr
\newline \\
  &\prependinstr(\al ~ (\seq i, \mu, n, c), \seq{i'}) = \al ~ (\seq{i'} ~ \seq i, \mu, n, c) \\
%
% increase_ctx_cnt
\newline \\
  &\increasectxcnt(\al ~ (\seq i, \mu, n, c)) = \al ~ (\seq i, \mu, n + 1, c) \\
%
% push
\newline \\
  &\push((\seq{wv}, \seq{wi}, sto), wv') = wv' ~ \seq{wv}, \seq{wi}, sto \\
%
% pop
\newline \\
  &\pop((wv ~ \seq{wv'}, \seq{wi}, sto)) = \seq{wv'}, \seq{wi}, sto \\
%
% popn
\newline \\
  &\popn((wv^n ~ \seq{wv'}, \seq{wi}, sto), n) = \seq{wv'}, \seq{wi}, sto \\
%
% get_cur_ctx
\newline \\
  &\seq{wv_1}, wv_{ctx} ~ \seq{wv_2} = \splitctx(\seq{wv}) \\
  \hline
  &\getcurctx((\seq{wv}, \seq{wi}, sto)) = wv_{ctx} \\
%
% get_cur_frame-frame
\newline \\
  &\seq{wv_1}, wv_{ctx} ~ \seq{wv_2} = \splitctx(\seq{wv}) \quad \isframe(wv_{ctx}) \\
  \hline
  &\getcurframe((\seq{wv}, \seq{wi}, sto)) = wv_{ctx} \\
% get_cur_frame-other
  &\seq{wv_1}, wv_{ctx} ~ \seq{wv_2} = \splitctx(\seq{wv}) \quad \neg \isframe(wv_{ctx}) \\
  \hline
  &\getcurframe((\seq{wv}, \seq{wi}, sto)) = \getcurframe(\seq{wv_2}) \\
%
% set_cur_frame-frame
\newline \\
  &\seq{wv_1}, wv_{ctx} ~ \seq{wv_2} = \splitctx(\seq{wv}) \quad \isframe(wv_{ctx}) \\
  \hline
  &\setcurframe((\seq{wv}, \seq{wi}, sto), v_{frame})
  =
  \seq{wv_1} ~ v_{frame} ~ \seq{wv_2}, \seq{wi}, sto \\
%
% set_cur_frame-other
\newline \\
  &\seq{wv_1}, wv_{ctx} ~ \seq{wv_2} = \splitctx(\seq{wv}) \quad \neg \isframe(wv_{ctx}) \\
  \hline
  &\setcurframe((\seq{wv}, \seq{wi}, sto), v_{frame})
  =
  \seq{wv_1} ~ wv_{ctx} ~ \setcurframe(\seq{wv_2}, v_{frame}), \seq{wi}, sto \\
%
% is_true
\newline \\
  &\istrue(\boolv ~ b) = b \\
  &\istrue(\listv ~ \seq v) = \forall v' \in \seq v. \istrue (v') \\
\end{align*}





