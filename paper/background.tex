% !TEX root = main.tex

\chapter{Background} \label{ch:background} \noindent

\section{WebAssembly} \label{sec:webassembly}

WebAssembly (Wasm) is a portable, binary instruction format designed for safe
and efficient execution across diverse environments.
Initially developed to enhance performance in web browsers, diverse ecosystems
leverages the advantages of Wasm, including cloud and edge
computing~\cite{lucet, cloudflare}, the Internet of Things (IoT)\cite{wasm-iot}, and
blockchain technologies\cite{wasm-blockchain}.
Wasm programs are compiled from high-level languages (e.g., C, C++, Rust) into
a compact, platform-independent binary format, enabling consistent execution
across different hardware and operating systems.


% Structure
A Wasm program is comprised of \textit{modules}.
A module is the unit of deployment, loading, and compilation, which contains
definitions such as functions, tables, memories, and globals.
Functions represent units of executable code.
Each function has a type specifying its parameter and result types, along with
a body composed of Wasm instructions.
Tables represent indexed collections of references, such as function
references.
Tables are commonly used for indirect function calls, enabling features like
dynamic dispatch.
Memories represent linear memory, a contiguous, resizable array of bytes that
serves as the module’s heap.
Memory is used for storing and accessing data such as arrays, strings, or other
data structures.
Globals represent mutable or immutable variables that hold single values.
Globals can store values such as integers or floating-point numbers and are
accessible throughout the module.


% Instantiation & store
Before executing the code in a module, the module must be instantiated to
validate its content and allocate its definitions into a runtime data structure
called the \textit{store}.
The store acts as the memory space where executable code for functions are
stored, where function references are placed in tables, where the linear memory
is allocated for data, and where global variables are maintained.
Once instantiated, functions in the module can be invoked.
Invoking a function triggers the execution of its body, a sequence of
Wasm instructions.
These instructions operate on the module’s internal state, including the values
stored in the store.


% Execution
Wasm execution is based on a stack machine.
An instruction consumes values from the stack as operands, performs an
operation, and pushes the resulting values onto the stack.
For example, the \texttt{testop} instruction exemplifies how this execution
model operates in practice.
\cref{fig:testop} illustrates the Wasm specification document for the
\texttt{testop} instruction.
It describes its semantics using two complementary forms: prose notation and
formal notation.
The prose notation is written in the upper part, and the formal notation is
written in the lower grey box.
In the prose notation, the instruction pops a value from the stack, performs a
test operation, and pushes the result back onto the stack.
In the formal notation, this is compactly expressed as a reduction rule
consisting of the left-hand side (LHS), right-hand side (RHS), and premises:
$(t.const ~ c_1) ~ t.testop$, $(i32.const ~ c)$, and $c = testop_t(c_1)$,
respectively.
The LHS indicates that the value at the top of the stack is $(t.const ~ c_1)$,
and the instruction to execute is $t.testop$.
The RHS shows that the value at the top of the stack is updated to $(i32.const
~ c)$, which occurs if the premise $c = testop_t(c_1)$ holds true.
% Additionally, the stack also stores structures related to the control flow
% along with the input and output values of the instructions: A \textit{frame}
% for function calls and a \textit{label} for branching.


% Structured Control Flow
\red{Wasm supports control instructions for structured control flow, which
eliminates the need for arbitrary jumps, ensuring structured and predictable
execution.
Control instructions include \texttt{blocks}, \texttt{call}, \texttt{br}, and
\texttt{return}, which are explicitly delimited in the code.
Each control instruction is associated with \textit{contexts} that facilitate
structured execution and support handling nested or complex control flows.
For instance, the \texttt{block} instruction groups a sequence of instructions
and provides a scoped execution context with a \textit{label}, while the
\texttt{call} instruction executes body of a function with a \textit{frame}.
The \texttt{br} and \texttt{return} instructions jump to the beginning or
the end of the block and the function, specified within the contexts.
}
