% !TEX root = main.tex

\chapter{Introduction}
\label{ch:intro}
\noindent

\begin{itemize}
\item Wasm
  \begin{itemize}
  \item is low-level bytecode language, platform-indep, ...
  \item is compile target for Web applications
  \item is used in wide area such as IoT, ...
  \end{itemize}
\item Wasm Spec
  \begin{itemize}
  \item descibes syntax and semantics of Wasm
  \item should be rigorous because ... (brief explanation for two format)
  \item has two format to describe semantics (formal\&prose notation)
  \item formal notation is written in math expression: concise, clear, amenable to proofs, ...
  \item prose notation is step by step instruction written English: vague (in some part), amenable to developing engines
  \end{itemize}
\item SpecTec
  \begin{itemize}
  \item is proposed to alleviate spec writing process
  \item is accepted by Wasm committee to be an official Wasm specification tool
  \item define AL to describe Wasm semantics in prose notation
  \item AL is designed to be similar to prose notation (describe with fig)
  \item implement AL interpreter to make the spec executable
  \item executable spec passes official wasm test $\leftarrow$ correctness
  \end{itemize}
\item Problem
  \begin{itemize}
  \item no semantics in AL
  \item prose notation is still vague unless read AL interpreter code
  \item ???
  \end{itemize}
\item Contribution
  \begin{itemize}
  \item formalize AL syntax$\&$semantics
  \item make prose notation clear to understand
  \item ???
  \end{itemize}
\end{itemize}


% breif overview
WebAssembly (Wasm)~\cite{wasm} is a low-level bytecode language that is safe, fast,
portable, and compact.
It is widely used as a compilation target for web applications.
Beyond web development, Wasm's advantages are also deployed in areas such as
edge computing~\cite{wasm-edge}, IoT~\cite{wasm-iot}, and
blockchain~\cite{wasm-block}.


% risk of implementation divergence
There are numerous Wasm engines, with all major browsers implementing their own
multi-tiered versions~\cite{v8, spidermonkey, webkit}.
Additionally, specialized engines target specific domains, such as embedded
systems and edge computing.
However, ensuring portability across these diverse implementations introduces
the risk of divergence among them.


% rigorous standardization -> spec is rigorous: formal notation & prose notation
To address this challenge, Wasm has been rigorously standardized by the
W3C~\cite{wasm-w3c}.
The Wasm specification is particularly rigorous, describing its semantics in
two complementary forms: formal notation and prose notation.
The formal notation employs mathematical rules to succinctly define the
semantics, supporting proofs such as type soundness.
Conversely, the prose notation uses pseudocode-like algorithms to explain the
semantics through step-by-step instructions.
Since most Wasm users and engine developers are less familiar with mathematical
formalism, they primarily rely on the prose notation.


% challenging specification process
The demanding standardization process places a significant burden on
specification authors.
Crafting this specification document is labor-intensive, and as Wasm evolves,
the manual effort required becomes increasingly challenging to scale.
Moreover, the dual requirement of maintaining both the formal notation and
prose notation exacerbates these difficulties.
The formal notation is written in LaTeX, and the prose notation is authored
in reStructuredText; neither of which is particularly user-friendly for
collaborative review.
This lack of accessibility in the specification's tooling increases the
likelihood of inconsistensies and errors, further complicating the
standardization effort.


% SpecTec
To mitigate this problem, we \red{had been} developing SpecTec, a framework for
mechanizing WebAssembly specification.
It provides domain specific language (DSL) that enables the declarative
definition of Wasm syntax and semantics, akin to the formal notation.
Additionally, to capture the pseudocode-like structure of the prose notation,
SpecTec incorporates an imperative language, \textit{AL}.
SpecTec performs type-checking on the DSL to prevent meta-level specification
errors, translates the DSL into AL, and generates various artifacts, including
the specification document, a Wasm interpreter, and mechanized definitions for
theorem provers, along with the DSL and AL.
Notably, AL plays a central role in the generation of the Wasm interpreter,
bridging the gap between formal specification and executable code.


% problem
However, at that time, AL was unable to capture the Wasm control flow
correctly.
The Wasm interpreter generated using AL produced wrong results when running
Wasm official test suites.
Additionally, the lack of formal semantics or a formalized definition for AL
made it difficult to identify and address the fundamental issues in its design.
Since running tests is the primary method for assessing the correctness of the
specification, this limitation hindered the development of both the generated
interpreter and the specification itself.
This underscored the need for a more robust approach to formally define AL and
ensure its alignment with Wasm's complex semantics.





% contribution
build AL's computation model for Wasm, formalize AL semantics


