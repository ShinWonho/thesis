% !TEX root = main.tex

\chapter{Background}
\label{ch:background}
\noindent


\section{WebAssembly}

% breif overview
WebAssembly is a low-level bytecode language that is safe, fast, portable, and
compact~\cite{wasm}.
It is widely used as a compile target for web applications.
Not only that, other areas such as edge computing~\cite{wasm-edge},
IoT~\cite{wasm-iot}, and block chains~\cite{wasm-block} deploy the advantages
of WebAssembly.

% risk of implementation divergence
There are dozens of WebAssembly engines; all the browsers have there own
implementations of WebAssembly with multiple tiers, and there are many engines
that target for specific areas including embedded systems and edge computing.
However, WebAssembly should be portable across these implementations, it risks
the implementations divergence.

% rigorous standardization -> formal notation & prose notation
To mitigate this problem, WebAssembly has been standardized very rigourously by
the W3C~\cite{wasm-w3c}.
One of the requirements to standardize a feature is to describe the semantics
of WebAssembly in two forms: formal notation and prose notation.
Formal notation uses mathematical rules to compactly describe the semantics,
and it is used for proofs such as type soundness.
On the other hand, prose notation uses psudocodes to explain the semantics
step by step.
People such as WebAssembly engine developers who are not familiar with
mathematical rules utilize it.

% structure: module, function, ...

% execution: instantiation, invocation, ...

% control structure: label frame block ...


\section{SpecTec}
