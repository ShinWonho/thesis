% -*- TeX:UTF-8 -*-
%%
%% KAIST 학위논문양식 LaTeX용 (ver 0.5) 예시
%%
%% @version 0.4
%% @author  채승병 Chae,Seungbyung (mailto:chess@kaist.ac.kr)
%% @date    2004. 11. 12.
%%
%% @requirement
%% teTeX, fpTeX, teTeX 등의 LaTeX2e 배포판
%% + 은광희 님의 HLaTeX 0.991 이상 버젼 또는 홍석호 님의 HPACK 1.0
%% : 설치에 대한 자세한 정보는 http://www.ktug.or.kr을 참조바랍니다.
%%
%% @note
%% 기존에 널리 쓰여오던 차재춘 님의 학위논문양식 클래스 파일의 형식을
%% 따르지 않고 전면적으로 다시 작성하였습니다. 논문 정보 입력부분에서
%% 과거 양식과 다른 부분이 많으니 아래 예시에 맞춰 바꿔주십시오.
%%
%%
%% @acknowledgement
%% 본 예시 논문은 물리학과 박사과정 김용현 님의 호의로 제공되었습니다.
%%
%% -------------------------------------------------------------------
%% @information
%% 이 예제 파일은 hangul-ucs를 사용합니다. UTF-8 입력 인코딩으로
%% 작성되었습니다. hlatex의 hfont는 이용하지 않습니다. --2006/02/11
%% 본 템플릿은 전산학부 김민혁 교수에의해서 버그 수정되었습니다. -- 2016/11/25
%% 본 템플릿은 전산학부 김민혁 교수에의해서 추가 버그 수정되었습니다. -- 2023/06/15

% @class kaist.cls
% @options [default: doctor, korean, final]
% - doctor: 박사과정 | master : 석사과정
% - korean: 한글논문 | english: 영문논문
% - final : 최종판   | draft  : 시험판
% - pdfdoc : 선택하지 않으면 북마크와 colorlink를 만들지 않습니다.

% TODO: This is a draft
\documentclass[master,english,draft]{kaist-ucs} % 석사과정
%\documentclass[doctor,english,final]{kaist-ucs} % 박사과정
% XXX: Show figure in draft mode
\setkeys{Gin}{draft=false}


% If you want make pdf document (include bookmark, colorlink)
%\documentclass[doctor,english,final,pdfdoc]{kaist-ucs}

% kaist.cls 에서는 기본으로 dhucs, ifpdf, graphicx 패키지가 로드됩니다.
% 추가로 필요한 패키지가 있다면 주석을 풀고 적어넣으십시오,
%\usepackage{...}
\usepackage{amsmath}
\usepackage{amssymb}
\usepackage[capitalise,nameinlink]{cleveref}
\usepackage[final]{listings}
\usepackage[T1]{fontenc}
\usepackage{lmodern}

\lstdefinestyle{dsl}{
  xleftmargin=\parindent,
  basicstyle=\ttfamily,
  keywordstyle=\bfseries,
  columns=fullflexible,
  language=ML,
  commentstyle=\itshape,
  escapechar=\@,
  keywords={syntax,relation,rule,def,grammar,var,if,otherwise,hint}
}

\lstdefinestyle{al}{
  xleftmargin=\parindent,
  basicstyle=\ttfamily,
  keywordstyle=\bfseries,
  columns=fullflexible,
  escapechar=\@,
  keywords={Assert, Pop, Let, Push}
}



\newcounter{example} % Define a counter for examples

% Define the "example" environment
\newenvironment{example}{%
    \refstepcounter{example} % Update the counter and make it referenceable
    \textbf{Example~\theexample}%
}{}

\crefname{example}{Example}{Examples} % Singular and plural forms
\Crefname{example}{Example}{Examples} % For capitalized references
\crefformat{example}{\textbf{#2Example~#1#3}}

\crefname{section}{\S\!\!}{\S\!\!\S}
\Crefname{section}{\S\!\!}{\S\!\!\S}
\crefname{chapter}{\S\!\!}{\S\!\!\S}
\Crefname{chapter}{\S\!\!}{\S\!\!\S}


\allowdisplaybreaks

\newcommand{\red}[1]{\textcolor{purple}{#1}}

%% font definitions

\renewcommand{\symbol}[1]{\textbf{#1}}
\newcommand{\helper}[1]{\textsf{#1}}
\newcommand{\premise}[1]{\textsf{\textbf{#1}}}
\newcommand{\tildecode}{\texttt{\raisebox{0.5ex}{\texttildelow}}}
\newcommand{\dollarcode}{\texttt{\textdollar}}


% Algorithm
\newcommand{\rel}{\symbol{Rel}}
\newcommand{\fun}{\symbol{Fun}}

% Continuation
\newcommand{\mt}{\symbol{Empty}}
\newcommand{\toplevelcall}{\symbol{TopLevelCall}}
\newcommand{\call}{\symbol{Call}}
\newcommand{\wasm}{\symbol{Wasm}}
\newcommand{\exe}{\symbol{Execute}}
\newcommand{\algo}{\symbol{Algo}}
\newcommand{\ret}{\symbol{Return}}

% Instruction
\newcommand{\ifi}{\symbol{IfI}}
\newcommand{\eitheri}{\symbol{EitherI}}
\newcommand{\enteri}{\symbol{EnterI}}
\newcommand{\pushctxi}{\symbol{PushCtxI}}
\newcommand{\pushi}{\symbol{PushI}}
\newcommand{\popctxi}{\symbol{PopCtxI}}
\newcommand{\popi}{\symbol{PopI}}
\newcommand{\popni}{\symbol{PopNI}}
\newcommand{\popalli}{\symbol{PopAllI}}
\newcommand{\leti}{\symbol{LetI}}
\newcommand{\trapi}{\symbol{TrapI}}
\newcommand{\nopi}{\symbol{NopI}}
\newcommand{\returnreli}{\symbol{ReturnRelI}}
\newcommand{\returnfuni}{\symbol{ReturnFunI}}
\newcommand{\executei}{\symbol{ExecuteI}}
\newcommand{\calli}{\symbol{CallI}}
\newcommand{\replaceframei}{\symbol{ReplaceFrameI}}
\newcommand{\replacestorei}{\symbol{ReplaceStoreI}}

% Expression
\newcommand{\vare}{\symbol{VarE}}
\newcommand{\nume}{\symbol{NumE}}
\newcommand{\boole}{\symbol{BoolE}}
\newcommand{\fnamee}{\symbol{FnameE}}
\newcommand{\une}{\symbol{UnE}}
\newcommand{\bine}{\symbol{BinE}}
\newcommand{\acce}{\symbol{AccE}}
\newcommand{\upde}{\symbol{UpdE}}
\newcommand{\stre}{\symbol{StrE}}
\newcommand{\compe}{\symbol{CompE}}
\newcommand{\cate}{\symbol{CatE}}
\newcommand{\meme}{\symbol{MemE}}
\newcommand{\lene}{\symbol{LenE}}
\newcommand{\tupe}{\symbol{TupE}}
\newcommand{\casee}{\symbol{CaseE}}
\newcommand{\itere}{\symbol{IterE}}
\newcommand{\liste}{\symbol{ListE}}
\newcommand{\getcurctxe}{\symbol{GetCurContextE}}
\newcommand{\choosee}{\symbol{ChooseE}}
\newcommand{\iscaseofe}{\symbol{IsCaseOfE}}
\newcommand{\ctxkinde}{\symbol{CtxKindE}}
\newcommand{\matche}{\symbol{MatchE}}
\newcommand{\hastypee}{\symbol{HasTypeE}}

% Unary operator
\newcommand{\notop}{\symbol{NotOp}}
\newcommand{\minusop}{\symbol{MinusOp}}

% Binary operator
\newcommand{\addop}{\symbol{AddOp}}
\newcommand{\subop}{\symbol{SubOp}}
\newcommand{\mulop}{\symbol{MulOp}}
\newcommand{\divop}{\symbol{DivOp}}
\newcommand{\modop}{\symbol{ModOp}}
\newcommand{\expop}{\symbol{ExpOp}}
\newcommand{\implop}{\symbol{ImplOp}}
\newcommand{\equivop}{\symbol{EquivOp}}
\newcommand{\andop}{\symbol{AndOp}}
\newcommand{\orop}{\symbol{OrOp}}
\newcommand{\eqop}{\symbol{EqOp}}
\newcommand{\neop}{\symbol{NeOp}}
\newcommand{\ltop}{\symbol{LtOp}}
\newcommand{\gtop}{\symbol{GtOp}}
\newcommand{\leop}{\symbol{LeOp}}
\newcommand{\geop}{\symbol{GeOp}}

% Path
\newcommand{\idxp}{\symbol{Idx}}
\newcommand{\slicep}{\symbol{Slice}}
\newcommand{\dotp}{\symbol{Dot}}

% Iter
\newcommand{\listiter}{\symbol{List}}
\newcommand{\listniter}{\symbol{ListN}}
\newcommand{\listidxiter}{\symbol{Index}}

% Value
\newcommand{\numv}{\symbol{NumV}}
\newcommand{\boolv}{\symbol{BoolV}}
\newcommand{\listv}{\symbol{ListV}}
\newcommand{\strv}{\symbol{StrV}}
\newcommand{\casev}{\symbol{CaseV}}
\newcommand{\tupv}{\symbol{TupV}}
\newcommand{\fnamev}{\symbol{FnameV}}
\newcommand{\trapv}{\symbol{TrapV}}
\newcommand{\storev}{\symbol{StoreV}}

% Helper function
\newcommand{\getalgoname}{\helper{get\_algo\_name}}
\newcommand{\lookup}{\helper{lookup}}
\newcommand{\assign}{\helper{assign}}
\newcommand{\createalgo}{\helper{create\_algo}}
\newcommand{\exit}{\helper{exit}}
\newcommand{\getenv}{\helper{get\_env}}
\newcommand{\setenv}{\helper{set\_env}}
\newcommand{\addctx}{\helper{add\_ctx}}
\newcommand{\getstore}{\helper{get\_store}}
\newcommand{\setstore}{\helper{set\_store}}
\newcommand{\prependinstr}{\helper{prepend\_instr}}
\newcommand{\popwasminstr}{\helper{pop\_wasm\_instr}}
\newcommand{\popwasmctx}{\helper{pop\_wasm\_ctx}}
\newcommand{\push}{\helper{push}}
\newcommand{\pop}{\helper{pop}}
\newcommand{\popn}{\helper{popn}}
\newcommand{\unop}{\helper{unop}}
\newcommand{\splitctx}{\helper{split\_ctx}}
\newcommand{\getcurctx}{\helper{get\_cur\_ctx}}
\newcommand{\getcurframe}{\helper{get\_cur\_frame}}
\newcommand{\setcurframe}{\helper{set\_cur\_frame}}
\newcommand{\access}{\helper{access}}
\newcommand{\update}{\helper{update}}
\newcommand{\updateidx}{\helper{update\_idx}}
\newcommand{\updateslice}{\helper{update\_slice}}
\newcommand{\updatedot}{\helper{update\_dot}}
\newcommand{\getendinstr}{\helper{get\_end\_instr}}
\newcommand{\isendinstr}{\helper{is\_end\_instr}}
\newcommand{\istrue}{\helper{is\_true}}
\newcommand{\isframe}{\helper{is\_frame}}
% Omit definition
\newcommand{\domain}{\helper{domain}}
\newcommand{\binop}{\helper{binop}}
% Reference interpreter
\newcommand{\match}{\helper{match}}
\newcommand{\hastype}{\helper{has\_type}}


% @command title 논문 제목(title of thesis)
% @options [default: (none)]
% - korean: 한글제목(korean title) | english: 영문제목(english title)
% TODO: Below is working titles
\title[korean] {실행가능한 알고리즘 명세 언어를 활용하여 올바른 WebAssembly 명세를 위한 기계화 명세 기술 연구}
\title[english]{Mechanizing Specification with An Executable Algorithmic Specification Language for Correct WebAssembly Specification}

% @note 표지에 출력되는 제목을 강제로 줄바꿈하려면 \linebreak 을 삽입.
%       \\ 나 \newline 등을 사용하면 안됩니다. (아래는 예시)
%
%\title[korean]{탄소 나노튜브의 물리적 특성에 대한\linebreak 이론 연구}
%\title[english]{Theoretical study on physical properties of\linebreak
%                carbon nanotubes}
%
% If you want to begin a new line in cover, use \linebreak .
% See examples above.
%


% @command author 저자 이름
% @param   family_name, given_name 성, 이름을 구분해서 입력
% @options [default: (none)]
% - korean: 한글이름 | chinese: 한문이름 | english: 영문이름
% 한문 이름이 없다면 빈 칸으로 두셔도 됩니다.
%
%
% If you are a foreigner , write your name in korean or your korean name.
% If you can't write native character, you can make the chinese blank empty 
% Write as follow
% \author[korean]{family name in korean}{given name in korean}
% \author[chinese]{family name in your native language}{given name in your native language}
% \author[english]{family name in english}{given name in english}
%
\author[korean] {신}{원 호}
\author[korean2] {신}{원호}    %이름을 붙여 써 주시기 바랍니다.
\author[chinese]{申}{元 鎬}
\author[english]{Shin}{Wonho}

% @command advisor 지도교수 이름 (복수가능)
% @usage   \advisor[options]{...한글이름...}{...영문이름...}{signed|nosign}
% @options [default: major]
% - major: 주 지도교수  | coopr: 공동 지도교수
% TODO: sign
\advisor[major]{류 석 영}{Sukyoung Ryu}{nosign}
\advisor[major2]{류석영}{Sukyoung Ryu}{nosign}    %한글 성과 한글 이름을 모두 붙여 써 주시기 바랍니다.

% [주의] 전산학부의 경우, 전공이름(Computer Science)을 적어주시기 바랍니다. 조직명(School of Computing) 적지 말아주세요!
\advisorinfo{Professor of Computer Science} %제출승인서에 들어가는 교수님 정보, advisor's information 

%\advisor[coopr]{홍 길 동}{Gil-Dong Hong}{nosign}
%\advisor[coopr2]{홍길동}{Gil-Dong Hong}{nosign}    %한글 성과 한글 이름을 모두 붙여 써 주시기 바랍니다.
%
% 지도교수 한글이름은 입력하지 않아도 됩니다.
% You may not input advisor's korean name
% like this \advisor[major]{}{Chang, Kee Joo}{signed}
%


% @command department {학과이름}{학위종류} - 아래 규칙에 따라 코드를 입력
% @command department {department code}{degree field}
%
% department code
% 2. 석박사학위논문 작성 및 제출요령 4쪽 ~ 5쪽 참고
% 또는 kaist-ucs.cls 의 % @command department 참고

% science: 이학 | engineering: 공학 | business : 경영학
% 박사논문의 경우는 학위종류를 입력하지 않아도 됩니다.
% If you write Ph.D. dissertation, you cannot input degree field.
% The third parameter : a | b | c
% a: 소속된 학과만 쓰는 옵션 (학과에만 소속되어 있는 경우에는 무조건 a를 선택해야 함)
% b: 학과 아래의, 프로그램이나 학제전공에 소속되어 있을 경우에 학과와 프로그램을 함께 쓰는 옵션
% c: 학과 아래의, 프로그램이나 학제전공에 소속되어 있을 경우에 학과를 쓰지 않고 프로그램이나 학제전공의 이름만 쓰는 옵션 
% 
% a: it represents only the name of department. (if you aren't in the program under the department, must choose a)
% b: it represents the names of department and the program that is under the department (consider this when you are in the program not only department)
% c: it represents only the name of program that is under the department (consider this when you are in the program not only department)
\department{CS}{engineering}{a}

% @command referee 심사위원 (석사과정 3인, 박사과정 5인)
\referee[1]{류 석 영}
\referee[2]{유 신}
\referee[3]{허 기 홍}
% Of course english name is available

% @command approvaldate 지도교수논문승인일
% @param   year,month,day 연,월,일 순으로 입력
% TODO: not approved!
\approvaldate{2020}{12}{5}

% @command refereedate 심사위원논문심사일
% @param   year,month,day 연,월,일 순으로 입력
% TODO: TBD
\refereedate{2020}{12}{5}

% @command gradyear 졸업년도
\gradyear{2025}

% 본문 시작
\begin{document}

  % 앞표지, 속표지, 학위논문 제출승인서, 학위논문 심사완료 검인서는
  % 클래스 옵션을 final로 지정해주면 자동으로 생성되며,
  % 반대로 옵션을 draft로 지정해주면 생성되지 않습니다.

  % 논문 서지, 초록, 핵심 낱말, 영문 초록, 영어 핵심 낱말 (Information of thesis, abstract in korean, keywords in korean, abstract in english, keywords in english)
 \thesisinfo
 %% Letters of abstract in korean must be less than 500 and words of abstract in english must be less than 300.
 %% Number of keywords must be less than 6.
 %% Don't write english letters in the abstract in korean.
  \begin{summary}
  % TODO: Korean abstract
  초록초록

  \end{summary}

  \begin{Korkeyword}
  % TODO: Korean keyword
  \end{Korkeyword}


  \begin{abstract}
  % TODO: English abstract
  Greengreen
  \end{abstract}

  \begin{Engkeyword}
  % TODO: English keyword
  \end{Engkeyword}


  \addtocounter{pagemarker}{1}                 % 백색별지분을 고려
  \newpage



  % 목차 (Table of Contents) 생성
  \tableofcontents

  % 표목차 (List of Tables) 생성
  \listoftables

  % 그림목차 (List of Figures) 생성
  \listoffigures

  % 위의 세 종류의 목차는 한꺼번에 다음 명령으로 생성할 수도 있습니다.
  %\makecontents

%% 이하의 본문은 LaTeX 표준 클래스 report 양식에 준하여 작성하시면 됩니다.
%% 하지만 part는 사용하지 못하도록 제거하였으므로, chapter가 문서 내의
%% 최상위 분류 단위가 됩니다.
%% You cannot use 'part'

% !TEX root = main.tex

\chapter{Introduction}
\label{ch:intro}
\noindent

% breif overview
WebAssembly (Wasm)~\cite{wasm} is a low-level bytecode language that is safe, fast,
portable, and compact.
It is widely used as a compilation target for web applications.
Beyond web development, Wasm's advantages are also deployed in areas such as
edge computing~\cite{wasm-edge}, IoT~\cite{wasm-iot}, and
blockchain~\cite{wasm-block}.
% risk of implementation divergence
There are numerous Wasm engines, with all major browsers implementing their own
multi-tiered versions~\cite{v8, spidermonkey, webkit}.
Additionally, specialized engines target specific domains, such as embedded
systems and edge computing.
However, ensuring portability across these diverse implementations introduces
the risk of divergence among them.


% rigorous standardization -> spec is rigorous: formal notation & prose notation
To address this challenge, Wasm has been rigorously standardized by the
W3C~\cite{wasm-w3c}.
The Wasm specification is particularly rigorous, describing its semantics in
two complementary forms: formal notation and prose notation.
\red{\cref{fig:testop} illustrates the specification document for $testop$
instruction, where the upper part contains the prose notation, and the lower
part contains the formal notation.}
The formal notation employs mathematical rules to succinctly define the
semantics, supporting proofs such as type soundness.
Conversely, the prose notation uses pseudocode-like algorithms to explain the
semantics through step-by-step instructions.
Since most Wasm users and engine developers are less familiar with mathematical
formalism, they primarily rely on the prose notation.

\begin{figure}[t]
    \centerline{\includegraphics[width=15cm]{fig/testop}}
    \caption[\texttt{testop} instruction in official specification document]
      {\texttt{testop} instruction in official specification document}
      \label{fig:testop}
    \centerline{\includegraphics[width=15cm]{fig/spectec-testop}}
    \caption[\texttt{testop} instruction in SpecTec specification document]
      {\texttt{testop} instruction in SpecTec specification document}
      \label{fig:spectec-testop}
\end{figure}


% challenging specification process
The demanding standardization process places a significant burden on
specification authors, \red{particularly in maintaining consistency and correctness.
Crafting the specification document is labor-intensive, which becomes
increasingly difficult to scale as Wasm evolves.
This challenge is compounded by the dual requirement to maintain both a formal
notation, written in LaTeX, and a prose notation, authored in reStructuredText.
The separate and user-unfriendly nature of these formats hinders collaborative
review and increases the risk of discrepancies and errors between the two
notations, further complicating the standardization process.}

% SpecTec
To mitigate this problem, we \red{propose} SpecTec, a framework for mechanizing
WebAssembly specification.
It provides domain specific language (DSL) that enables the declarative
definition of Wasm syntax and semantics, akin to the formal notation.
\red{
SpecTec performs type-checking on the DSL to prevent meta-level specification
errors and generates LaTeX for the formal notation.
Additionally, to capture the pseudocode-like structure of the prose notation,
SpecTec incorporates an imperative language, \textit{AL}, which stands for
Algorithmic Language.
SpecTec translates the DSL into AL and generate reStructuredText for the prose
notation.
\cref{fig:spectec-testop} is the specification document generated by the
SpecTec using the DSL and the AL.
The generated document closely resembles the official document except for some
minor notation changes and some missing hyperlinks.
One notable aspect of AL is that it is executable, meaning any specification
written in AL serves as a Wasm interpreter program.
By testing the Wasm interpreter program, we can check the correctness of the
specification.
This bridges the gap between formal specification and executable code.
Futhermore, SpecTec generates mechanized definitions for theorem provers.}


% challenge
\red{
One challenge of our approach is making AL executable.
Since the prose notation in the official specification is written in informal
pseudocode, the interpretation of the prose notation is not straightforward.
The naive implementation of AL interpreter produced incorrect results when
running Wasm official test suites, particularly those related to control flow.
}
Additionally, the absence of a formalized definition or formal semantics for AL
made it difficult to identify and address the fundamental issues in its design.
Since running tests is the primary method for assessing the correctness of the
specification, this limitation hindered the development of both the generated
interpreter and the specification itself.
This underscored the need for a more robust approach to formally define AL and
ensure its alignment with Wasm's complex semantics.


% solution
To address this, we present a formal model for how AL describes Wasm control flow.
Using this model, we identify the underlying issues and the necessary changes
to improve it.
Based on these insight, we formalize AL's syntax and semantics to accurately
capture Wasm control flow.
We then implement an AL interpreter according to this formalization and
evaluate its correctness.
The interpreter successfully passes all tests in the Wasm official test suite,
as well as tests from proposals.


% contribution
The contributions of this paper are as follows:
\begin{itemize}
  \item We propose a formal model for Wasm control flow in AL (\cref{ch:motivation})
  \item We formalize syntax and semantics of AL (\cref{ch:formal})
  \item We implement an AL interpreter based on the formalization (\cref{ch:eval})
\end{itemize}


% !TEX root = main.tex

\chapter{Background} \label{ch:background} \noindent

\section{WebAssembly} \label{sec:webassembly}

WebAssembly (Wasm) is a portable, binary instruction format designed for safe
and efficient execution across diverse environments.
Initially developed to enhance performance in web browsers, diverse ecosystems
leverages the advantages of Wasm, including cloud and edge
computing~\cite{lucet, cloudflare}, the Internet of Things (IoT)\cite{wasm-iot}, and
blockchain technologies\cite{wasm-blockchain}.
Wasm programs are compiled from high-level languages (e.g., C, C++, Rust) into
a compact, platform-independent binary format, enabling consistent execution
across different hardware and operating systems.

% Structure
A Wasm program is comprised of \textit{modules}.
A module is the unit of deployment, loading, and compilation, which contains
definitions such as functions, tables, memories, and globals.
Functions represent units of executable code.
Each function has a type specifying its parameter and result types, along with
a body composed of Wasm instructions.
Tables represent indexed collections of references, such as function
references.
Tables are commonly used for indirect function calls, enabling features like
dynamic dispatch.
Memories represent linear memory, a contiguous, resizable array of bytes that
serves as the module’s heap.
Memory is used for storing and accessing data such as arrays, strings, or other
data structures.
Globals represent mutable or immutable variables that hold single values.
Globals can store values such as integers or floating-point numbers and are
accessible throughout the module.


% Instantiation & store
Before executing the code in a module, the module must be instantiated to
validate its content and allocate its definitions into a runtime data structure
called the \textit{store}.
The store acts as the memory space where executable code for functions are
stored, where function references are placed in tables, where the linear memory
is allocated for data, and where global variables are maintained.
Once instantiated, functions in the module can be invoked.
Invoking a function triggers the execution of its body, a sequence of
Wasm instructions.
These instructions operate on the module’s internal state, including the values
stored in the store.


% execution
Wasm execution is based on a stack machine.
An instruction consumes values from the stack as operands, performs an
operation, and pushes the resulting values onto the stack.
For example, the \texttt{testop} instruction exemplifies how this execution
model operates in practice.
\cref{fig:testop} illustrates the Wasm specification document for the
\texttt{testop} instruction.
It describes its semantics using two complementary forms: prose notation and
formal notation.
The prose notation is written in the upper part, and the formal notation is
written in the lower grey box.
In the prose notation, the instruction pops a value from the stack, performs a
test operation, and pushes the result back onto the stack.
In the formal notation, this is compactly expressed as a reduction rule
consisting of the left-hand side (LHS), right-hand side (RHS), and premises:
$(t.const ~ c_1) ~ t.testop$, $(i32.const ~ c)$, and $c = testop_t(c_1)$,
respectively.
The LHS indicates that the value at the top of the stack is $(t.const ~ c_1)$,
and the instruction to execute is $t.testop$.
The RHS shows that the value at the top of the stack is updated to $(i32.const
~ c)$, which occurs if the premise $c = testop_t(c_1)$ holds true.
% Additionally, the stack also stores structures related to the control flow
% along with the input and output values of the instructions: A \textit{frame}
% for function calls and a \textit{label} for branching.


% !TEX root = main.tex

\chapter{Overview}
\label{ch:overview}
\noindent


\begin{figure}[t]
  \centerline{\includegraphics[width=15cm]{fig/overview}}
  \caption[An overview of the SpecTec architecture]
    {An overview of the SpecTec architecture}
    \label{fig:overview}
\end{figure}

% SpecTec: specification mechanization tool
SpecTec is a tool for \textit{mechanizing} WebAssembly specification
~\cite{spectec}.
Specification mechanization is a technique that treats a specification as data
that can be manipulated by a machine to automatically generate parser and
interpreter ~\cite{jiset}, to generate tests and perform differential tests
~\cite{jest}, to find meta-level errors in the specification ~\cite{jstar}, and
to perform meta-level static analysis for the static analysis of the defined
language ~\cite{jsaver}.
% Overview explanation
In this chapter, we explain the overall approach to generating multiple
artifacts from the specification and ensuring the correctness of the
specification.
\Cref{fig:overview} illustrates an overview of our approach.


% DSL
SpecTec offers a domain specific language (DSL) designed for writing
Wasm specifications.
It is a declarative language, providing a compact and user-friendly syntax.
This DSL can represent Wasm semantics in a manner closely resembling the formal
notation as shown in the following code for specification of testop
instruction.
The code below shows the specification for the \texttt{testop} instruction
written in the DSL:
\begin{lstlisting}[style=dsl]
rule Step_pure/testop:
(CONST nt c_1) (TESTOP nt testop)  ~>  (CONST I32 c)
-- if c = @\dollarcode@testop_(nt, testop, c_1)
\end{lstlisting}
The DSL is type checked to prevent meta-level errors such as notation misuses and
dimension mismatches.
Using the DSL, SpecTec generates LaTeX code and mechanized definitions for theorem provers.


% AL
Additionally, SpecTec introduces another imperative language, \textit{AL}, for
the prose notation.
In AL, semantics are expressed algorithmically, in the form of step-by-step
instructions that take inputs and perform operations, mirroring the prose notation.
SpecTec translates declarative definitions in the DSL into algorithmic
definitions in AL.
The following code is generated from the DSL, representing the algorithmic
specification for the \texttt{testop} instruction using AL:
\begin{lstlisting}[style=al]
@\textbf{Step\_pure/testop}@ nt testop {
    Assert (check_type_of_stack_top(nt))
    Pop (CONST nt c_1)
    Let c = @$\dollarcode$@testop_(nt, testop, c_1)
    Push (CONST I32 c)
}
\end{lstlisting}
Using AL, SpecTec generates reStructuredText code.
\cref{fig:spectec-testop} shows the generated document for \texttt{testop}
instruction.
The generated document closely resembles the official document except for some
minor notation changes and some missing hyperlinks.


% AL interpreter
A notable feature of AL is that it is interpretable.
SpecTec includes an AL interpreter that executes an AL program with given inputs
and produces results.
Since an AL program specify the behavior of a Wasm program, the input to an AL
program is a Wasm program, and the result is the execution outcome of the Wasm
program.
In other words, the generated AL program functions as a WebAssembly interpreter.
What is particularly intriguing is that if specification authors write a Wasm
specification using the DSL, a WebAssembly interpreter that adheres to the
defined semantics is automatically generated.
By testing this generated interpreter, we can assess the specification,
thereby ensuring its correctness.


% !TEX root = main.tex

\chapter{DSL}
\label{ch:dsl}
\noindent


% DSL high-level explanation
SpecTec offers a domain-specific language (DSL) tailored for specification
purposes.
The DSL is a declarative language, providing a compact and user-friendly syntax.
It includes definitions for type, relation, and functions.
Type definitions define the abstract syntax in a mannar analogous to the BNF
format, where terminals are denoted by capital letters, while non-terminals are
represented by lowercase letters.
The below code represents the abstract syntax for \texttt{testop},
\texttt{numtype} and \texttt{instr}.
\begin{lstlisting}[style=dsl]
syntax testop = EQZ
syntax numtype = I32 | I64 | F32 | F64 | V128
syntax instr =
  | CONST numtype number
  | TESTOP numtype testop
  | ...
\end{lstlisting}

With relation, user can define the formal semantics of Wasm.
The below code illustrates the execution semantics for \texttt{testop}
instruction described in a relation definition.
\begin{lstlisting}[style=dsl]
  relation Step_pure: instr* ~> instr*
  rule Step_pure/testop:
  (CONST numtype c_1) (TESTOP numtype testop)  ~>  (CONST I32 c)
  -- if c = @$\dollarcode$@testop_(numtype, testop, c_1)
\end{lstlisting}
The signature of the step relation $\tildecode$\texttt{>} is defined first.
It then specifies a step rule that describes the semantics of the
\texttt{testop} instruction with premises under which the relation holds.
SpecTec performs type checking to ensure that any incorrect symbols or
ill-typed relations like below code are rejected.
\begin{lstlisting}[style=dsl]
  (* Wrong type *)
  rule Step_pure/testop:
  (CONST numtype c_1) (TESTOP numtype testop)  ~>  c
  -- if c = @$\dollarcode$@testop_(numtype, testop, c_1)

  (* Notation misuse *)
  rule Step_pure/testop:
  (CONST numtype c_1) (TESTOP numtype testop)  =>  (CONST I32 c)
  -- if c = @$\dollarcode$@testop_(numtype, testop, c_1)
\end{lstlisting}


TODO: funciton definition
Function definitions can be utilized for other definitions such as
\texttt{\$testop\_} in the code.
In addition, they can be also used to describe semantics.
Especially, SpecTec uses function definitions to describe module instantiation
and function invocation.

TODO: latex and mechanized definitions


\chapter{Algorithmic Specification Language}
\label{ch:al}
\noindent


SpecTec introduces AL, an algorithmic specification language, for specifying
the prose specification.
Since the prose notation of the official specification is informal, it is
essential to rigorously define AL to ensure an correct specification, rather
than merely imitating the original.
We begin by outlining the informality of the prose specification in
\cref{sec:prose-informal}.
Next, we propose a formal model for the prose specification \cref{sec:model}.
We then define the syntax and the semantics of AL in \cref{sec:definition}.
Finally, we produce an executable Wasm specification using AL interpreter that
ensures the correctness of the specification in \cref{sec:exec-spec}.

\section{Informality in the prose specification}
\label{sec:prose-informal}

The prose specification describes Wasm semantics in an informal manner,
particularly when addressing control flow.
% Jump
Wasm control instructions \textit{jump} to certain code point for control
transfer.
While it does not explicitly explain the mechanics of jumping, the simplest
interpretation involves assuming a program counter.
\cref{fig:return} illustrates the prose specification for the \texttt{return}
instruction.
In the final step, it jumps to the instruction after the original call that
pushed the frame, which likely corresponds to updating the program counter to
the return address stored in the frame.


% Execute
On the other hand, some control instructions \textit{execute} another Wasm
instruction.
\cref{fig:if} represents the prose specification for the \texttt{if}
instruction, which executes a block instruction with instructions in the then
branch or the else branch based on the value in the stack.
However, the block instruction itself does not appeared in the actual code,
making its execution unclear in this context.
The situation become even more complicated when the block contains control
instructions, such as function call.
While the intended meaning can still be inferred, this informality leaves room
for wrong interpretations.


\section{A Formal Model for the prose specification}
\label{sec:model}


% TODO: Clever way to distinguish jargon rather then \textit
% Formal model
To formally specify the prose specification, we construct a formal model for
the prose specification.
% Execute
Rather than relying on a program counter, we employ an instruction stack to
represent execution progress.
Execution proceeds by popping an instruction from the stack and performing its
operations.
To avoid confusion with the stack that contains values interleaved with
contexts, we refer to the latter as the \textit{interleaved stack}.
This approach allows us to define the execute operation without introducing a
program counter.
We define a Wasm state as a tuple of the interleaved stack, the instruction
stack, and the store.
An operation, as described in a step of the prose specification, is then
formalized as a state transition.


% Jump
To define the jump formally, we split two distinct usages of the jump
operation: \textit{enter} and \textit{exit}.
Since Wasm supports structured control flow, the jump operation is used either
for entering or exiting control structures such as a function body and a block
of instructions.
% Enter
We define the enter as the operation of pushing the context associated with the
jump onto the interleaved stack and the instructions within the structure onto
the instruction stack.
Similarly, we define exit as the operation of popping the corresponding context
and instructions.


% State transition
We then formalize the enter operation as a state transition.
For brevity, we omit the store in this discussion.
Let the interleaved stack initially contains $n$ entries $we_1 ~ ... ~ we_n$,
and the instruction stack contains $m$ entries $wi_1 ~ ... ~ wi_m$.
Suppose $wi_1$ is a control instruction that enters a control structure
containing a sequence of k instrucitons $wi_{11} ~ ... ~ wi_{1k}$ with an
associated context $wc$.
The enter operation can be expressed as the following state transition:
\[
(we_1 ~ ... ~ we_n, ~ wi_1 ~ ... ~ wi_m) \quad\leadsto\quad
  (
    wc ~ we_1 ~ ...  ~ we_n, ~
    wi_{11} ~ ... ~ wi_{1k} ~ \textbf{end}_{wc} ~ wi_2 ~ ... ~ wi_m
  )
\]


% End
Note that we introduce an administrative instruction, \textbf{end}, to mark the
end of the structure in the instruction stack.
It models unique behavior described in the prose specification that must take
place when the end of a structure is reached without control transfer.
For example, if a function body executes without concluding with a
\texttt{return} instruction, a behavior that corresponds to the \texttt{return}
is performed.
\cref{fig:exiting-label} illustrates this behavior.
While not a Wasm instruction itself, it is represented in the same way as Wasm
instructions.
Therefore, we formalize this behavior as an administrative instruction,
\textbf{end}.


% Exit
In the same way, the exit operation can be described as a transition of Wasm
state.
Suppose $wi_1$ is a control instruction that exits a control structure, and $i$
values $wv_1 ~ ... wv_i$ are at the top of the interleaved stack, followed by a
context $wc$, and $j$ instructions $wi_1 ~ ... ~ wi_j$ are at the top of the
instruction stack followed by an \textbf{end} instruction.
The exit operation can be represented as the following state transition:
\[
(wv_1 ~ ... ~ wv_i ~ wc ~ we_{i+2} ~ ...  ~ we_n,
  wi_1 ~ ... ~ wi_j ~ \textbf{end}_{wc} ~ wi_{j+2} ~ ..  ~ wi_m)
  \quad\leadsto\quad
  (we_{i+1} ~ ... ~ we_n, wi_{j+1} ~ ... ~ wi_m)
\]
This transition model provides the foundation for the formal definition of AL.

% XXX: HACK
\newpage

\begin{figure}[h]
  \centerline{\includegraphics[width=15cm]{fig/return}}
  \caption[\texttt{return} instruction]{\texttt{return} instruction}
    \label{fig:return}
  \centerline{\includegraphics[width=15cm]{fig/if}}
  \caption[\texttt{if} instruction]{\texttt{if} instruction}
    \label{fig:if}
  \centerline{\includegraphics[width=15cm]{fig/returning}}
  \caption[A behavior for returning from a function without \texttt{return} instruction]
    {A behavior for returning from a function without \texttt{return} instruction}
    \label{fig:exiting-label}
\end{figure}





\section{Definition of Algorithmic Specification Language}
\label{sec:definition}

% TODO: remove seq command
\newcommand{\seq}[1]{#1^*}

%% Syntax
\subsection{Syntax}
\label{syntax}

{
\renewcommand{\arraystretch}{0.97}  % Decrease row spacing locally
\begin{align*}
\begin{array}{lcccrlr}
%
% Algorithm
  \text{Algorithm}\quad& A &\ni& a &::=& ~ \rel ~ (s, \seq e, \seq i) ~ | ~ \fun ~ (s, \seq e, \seq i) \\
%
% Instruction
  \text{Instruction}\quad& I &\ni& i &::=& ~ \ifi ~ (e, \seq i, \seq i) &\quad\text{(If)} \\
    &&&& | & ~ \eitheri ~ (\seq i, \seq i) &\quad\text{(Either)} \\
    &&&& | & ~ \enteri ~ (e, e) &\quad\text{(Enter)} \\
    &&&& | & ~ \pushctxi ~ e &\quad\text{(Push Context)} \\
    &&&& | & ~ \pushi ~ e &\quad\text{(Push)} \\
    &&&& | & ~ \exiti ~ &\quad\text{(Exit)} \\
    &&&& | & ~ \popctxi ~ e &\quad\text{(Pop Context)} \\
    &&&& | & ~ \popi ~ e &\quad\text{(Pop)} \\
    &&&& | & ~ \popni ~ (e, e) &\quad\text{(Pop N)} \\
    &&&& | & ~ \popalli ~ e &\quad\text{(Pop All)} \\
    &&&& | & ~ \leti ~ (e, e) &\quad\text{(Let)} \\
    &&&& | & ~ \trapi &\quad\text{(Trap)} \\
    &&&& | & ~ \returnreli &\quad\text{(Return Relation)} \\
    &&&& | & ~ \returnfuni ~ e &\quad\text{(Return Function)} \\
    &&&& | & ~ \executei ~ e &\quad\text{(Execute)} \\
    &&&& | & ~ \calli ~ (s, s, \seq{e}) &\quad\text{(Let Call)} \\
    &&&& | & ~ \replaceframei ~ (p^+, e) &\quad\text{(Replace Frame)} \\
    &&&& | & ~ \replacestorei ~ (p^+, e) &\quad\text{(Replace Store)} \\
%
% Expression
  \text{Expression}\quad& E &\ni& e &::=& ~ \vare ~ s &\quad\text{(Variable)} \\
    &&&& | & ~ \nume ~ n &\quad\text{(Number)} \\
    &&&& | & ~ \boole ~ b &\quad\text{(Boolean)} \\
    &&&& | & ~ \fnamee ~ s &\quad\text{(Function Name)} \\
    &&&& | & ~ \liste ~ \seq e &\quad\text{(List)} \\
    &&&& | & ~ \stre ~ \seq{(s, e)} &\quad\text{(Record)} \\
    &&&& | & ~ \tupe ~ \seq e &\quad\text{(Tuple)} \\
    &&&& | & ~ \casee ~ (s, \seq e) &\quad\text{(Tagged Tuple)} \\
    &&&& | & ~ \une ~ (unop, e) &\quad\text{(Unary Operation)} \\
    &&&& | & ~ \bine ~ (binop, e, e) &\quad\text{(Binary Operation)} \\
    &&&& | & ~ \acce ~ (e, p) &\quad\text{(Access)} \\
    &&&& | & ~ \upde ~ (e, p^+, e) &\quad\text{(Update)} \\
    &&&& | & ~ \cate ~ (e, e) &\quad\text{(Concatenation)} \\
    &&&& | & ~ \compe ~ (e, e) &\quad\text{(Composition)} \\
    &&&& | & ~ \meme ~ (e, e) &\quad\text{(Membership)} \\
    &&&& | & ~ \choosee ~ e &\quad\text{(Choose)} \\
    &&&& | & ~ \lene ~ e &\quad\text{(Length)} \\
    &&&& | & ~ \iscaseofe ~ (e, s) &\quad\text{(Check Tag)} \\
    &&&& | & ~ \getcurctxe  &\quad\text{(Get Current Context)} \\
    &&&& | & ~ \ctxkinde ~ s &\quad\text{(Context Kind)} \\
    &&&& | & ~ \itere ~ (e, iter, \seq{s}) &\quad\text{(Iteration)} \\
    &&&& | & ~ \matche ~ (e, e) &\quad\text{(Match)} \\
    &&&& | & ~ \hastypee ~ (e, s) &\quad\text{(Has Type)} \\
\end{array}
\end{align*}
}

\newpage
\begin{align*}
\begin{array}{lcccrlr}
%
% Path
  \text{Path}\quad& P &\ni& p &::=& ~ \idxp ~ e ~ | ~ \slicep ~ (e, e) ~ | ~ \dotp ~ s \\
%
% Iter
  \text{Iter}\quad& Iter &\ni& iter &::=& ~ \listiter ~ | ~ \listniter ~ e ~|~ \listidxiter ~ (s, e) \\
%
% Operator
  \text{Unary Operator}\quad& Unop &\ni& unop &::=& ~ \notop ~ | ~ \minusop \\
  \text{Binary Operator}\quad& Binop &\ni& binop &::=& ~ \addop ~ | ~ \subop ~ | ~ \mulop \\
    &&&& | & \divop ~ | ~ \modop ~ | ~ \expop \\
    &&&& | & \implop ~ | ~ \equivop ~ | ~ \andop \\
    &&&& | & \orop ~ | ~ \eqop ~ | ~ \neop ~ | ~ \ltop \\
    &&&& | & \gtop ~ | ~ \leop ~ | ~ \geop \\
% Primitives
  \text{String}\quad& \mathbb S &\ni& s \\
  \text{Integer}\quad& \mathbb Z &\ni& n \\
  \text{Boolean}\quad& \mathbb B &\ni& b \\
\end{array}
\end{align*}

% algorithm
An algorithm is either an AL relation or an AL function, corresponding to the
relation and the function definitions in the DSL.
% instruction
An AL instruction is \ifi{} for conditionally choosing a branch, \eitheri{} for
nondeterministically choosing a branch, \enteri{} for entering a control
structure, \pushctxi{} for pushing a Wasm context, \pushi{} for pushing a Wasm
value, \exiti{} for exiting a control structure, \popctxi{} for popping a Wasm
context, \popi{} for popping a Wasm value, \popni{} for popping $n$ Wasm
values, \popalli{} for popping all Wasm values within a Wasm context, \leti{}
for let binding, \trapi{} for trapping, \returnreli{} for returning from an AL
relation, \returnfuni{} for returning from an AL function, \executei{} for
executing given Wasm instructions, \calli{} for an AL function call with let
binding, \replaceframei{} for replacing a frame, or \replacestorei{} for
replacing a store.
% expression
An expression is a variable, number, boolean, AL function name, list, record
(or struct), tuple, tagged tuple, unary or binary operation, data structure
access or update, list or record concatenation, membership check, element
selection, length retrieval, tag check, Wasm context retrieval, Wasm context
kind check, iteration, match relation, or type relation.


\begin{align*}
\begin{array}{lcccrlr}
%
% State
  \text{State}\quad& \Sigma &\ni& \sigma & ::=& ~ \seq a, w, k \\
%
% Wasm state
  \text{Wasm State}\quad& W &\ni& w &::=& ~ \seq{we}, \seq{wi}, sto \\
%
% Wasm Entry
  \text{Wasm Entry}\quad& WE &\ni& we &::=& ~ wv ~ | ~ wc \\
%
% Wasm Value
  \text{Wasm Value}\quad& WV &\ni& wv &::=& ~ \casev ~ (s, \seq v) \\
%
% Wasm Context
  \text{Wasm Context}\quad& WC &\ni& wc &::=& ~ \casev ~ (\text{``Label"}, \seq v) ~ | ~ \casev ~ (\text{``Frame"}, \seq v) \\
%
% Wasm Instruction
  \text{Wasm Instruction}\quad& WI &\ni& wi &::=& ~ \casev ~ (s, \seq v) \\
%
% Store
  \text{Store}\quad& Store &\ni& sto &::=& ~ \seq{(s, v)} \\
%
% Continuation
  \text{Continuation}\quad & K &\ni& k &::=& ~ \mt &\quad\text{(Empty)} \\
    &&&& | & ~ \toplevelcall ~ (s, \seq v) &\quad\text{(Top-level Call)} \\
    &&&& | & ~ \call ~ (s, s, \seq v, k) &\quad\text{(Call)} \\
    &&&& | & ~ \exe ~ (\seq{wi}, k) &\quad\text{(Execute)} \\
    &&&& | & ~ \wasm ~ (\seq{wc}, k) &\quad\text{(Wasm)} \\
    &&&& | & ~ \algo ~ c &\quad\text{(Algorithm)} \\
    &&&& | & ~ \ret ~ (v, k) &\quad\text{(Return)} \\
%
% Context
  \text{Context}\quad& C &\ni& c &::=& ~ (\seq i, \mu, \seq{wc}, k) \\
%
% Environment
  \text{Environment}\quad& M &\ni& \mu &::=& ~ \seq{[s \mapsto v]} \\
\end{array}
\end{align*}

\newpage
\begin{align*}
\begin{array}{lcccrlr}
%
% Value
  \text{Value}\quad& V &\ni& v &::=& ~ \numv ~ n &\quad\text{(Number)} \\
    &&&& | & ~ \boolv ~ b &\quad\text{(Boolean)} \\
    &&&& | & ~ \fnamev ~ s &\quad\text{(Function Name)} \\
    &&&& | & ~ \listv ~ \seq v &\quad\text{(List)} \\
    &&&& | & ~ \strv ~ \seq{(s, v)} &\quad\text{(Record)} \\
    &&&& | & ~ \tupv ~ \seq v &\quad\text{(Tuple)} \\
    &&&& | & ~ \casev ~ (s, \seq v) &\quad\text{(Tagged Tuple)} \\
    &&&& | & ~ \trapv &\quad\text{(Trap)} \\
    &&&& | & ~ \storev &\quad\text{(Store)} \\
\end{array}
\end{align*}

% State
An AL state consists of a sequence of algorithms, a Wasm state, and a
continuation.
% Wasm State
A Wasm state consists of an interleaved stack, a Wasm instruction stack, and a
store.
% Wasm Entry
An interleaved stack contains two types of entries: a Wasm value and a Wasm
context.
% Wasm Value & Wasm Context & Wasm Instruction
Each of a Wasm value, a Wasm context, and a Wasm instruction is expressed as an
AL value $\casev$.
In particular, the tag of a Wasm context is either ``Label" or ``Frame".
% Store
A store is expressed as a sequence of pairs, each consisting of a field name
and an AL value.
% Continuation
A continuation is \mt{} for an empty continuation, \toplevelcall{} for
top-level function call, \call{} for a function call with let binding, \exe{}
for executing given Wasm instructions, \wasm{} for executing a Wasm instruction
in the instruction stack, \algo{} for interpreting AL instructions in an
algorithm, or \ret{} for returning from an AL function.
% Env
An environment is a finite mapping from variable names to AL values.
% Context
An AL context consists of a sequence of Wasm instruction, an environment, a
sequence of Wasm contexts, and a continuation.
% Value
An AL value is a number, boolean, function name, list, record, tuple, tagged tuple,
trap, or store.


% Terminology
In this section, AL-specific terms (state, context, value, function, and
instruction) are referred to simply as state, context, value, function, and
instruction, while Wasm terms remain unchanged.




\newpage
%% Semantics
\subsection{Semantics}
\label{semantics}

%% Continuation

\begin{gather*}
\boxed{\leadsto \subseteq \Sigma \times \Sigma} \\
%
% TopLevelCall
\newline \\
  \\
  \hline
  (\seq{a}, w, \toplevelcall ~ (s, \seq v)) \leadsto (\seq{a}, w, \createalgo(\seq a, s, \seq v, \mt)) \\
%
% Call
\newline \\
  \\
  \hline
  (\seq{a}, w, \call ~ (s_{var}, s_{name}, \seq v, k)) \leadsto
  (\seq{a}, w, \createalgo(\seq a, s_{name}, \seq v, \call ~ (s_{var}, s_{name}, \seq v, k))) \\
%
% Execute-empty
\newline \\
  \\
  \hline
  (\seq{a}, w, \exe ~ (\epsilon, k)) \leadsto (\seq{a}, w, k) \\
%
% Execute-instr
\newline \\
  \casev ~ (s, \seq v) = wi \\
  \hline
  (\seq{a}, w, \exe ~ (wi ~ \seq{wi}, k)) \leadsto
  (\seq{a}, w, \createalgo(\seq a, s, \seq v, \exe ~ (\seq{wi}, k))) \\
%
% Wasm-empty
\newline \\
  \\
  \hline
  (\seq{a}, w, \wasm ~ (\epsilon, k)) \leadsto (\seq{a}, w, k) \\
%
% Wasm-instr
\newline \\
  (w', \casev ~ (s, \seq v)) = \popwasminstr(w) \\
  \hline
  (\seq{a}, w, \wasm ~ (wc^+, k))
  \leadsto
  (\seq{a}, w', \createalgo(\seq a, s, \seq v, \wasm ~ (wc^+, k))) \\
%
% Al-empty
\newline \\
  \\
  \hline
  (\seq{a}, w, \algo ~ (\epsilon, \mu, \seq{wc}, k)) \leadsto (\seq{a}, w, k) \\
%
% Al-instr
\newline \\
  \seq{a}, w, (\seq{i_1}, \mu, \seq{wc}, k) \vdash i_0 \Rightarrow (w', k') \\
  \hline
  (\seq{a}, w, \algo ~ (i_0 ~ \seq{i_1}, \mu, \seq{wc}, k)) \leadsto (\seq{a}, w', k') \\
%
% Return
\newline \\
  \mu = \getenv(c) \qquad c' = \setenv(c, \mu[s_{var} \mapsto v]) \\
  \hline
  (\seq{a}, w, \ret ~ (v, \call ~ (s_{var}, s_{name}, \seq v, \algo ~ c)) \leadsto
  (\seq a, w, \algo ~ c') \\
\end{gather*}

The semantics of AL is defined by a state transition system.
% TopLevelCall
Given algorithms $\seq a$ generated from the DSL and a Wasm state $w$, a
top-level function $s$ can be invoked with arguments $\seq v$ to perform
either module instantiation or function invocation:
$(\seq a, w, \toplevelcall ~ (s, \seq v))$.
$\toplevelcall ~ (s, \seq v)${} transitions to \algo{} with the function name
$s$, the arguments $\seq v$, and \mt{}.
% Empty
\mt{} is an empty continuation, representing the end of the transition, so no
further transition occurs.
% Call
Similarly to the top-level call, $\call~ (s_{var}, s_{name}, \seq v, k)${}
transitionsto \algo{} with the function name $s_{name}$ and the arguments $\seq
v$, but with the current continuation nested within it.
% Execute
$\exe ~ (\seq{wi}, k)${} executes each Wasm instruction in the sequence
$\seq{wi}$ one by one, each containing a relation name and arguments.
The continuation transitions to \algo{} with the relation name, the arguments,
and the current continuation, with the Wasm instruction being popped.
% Wasm
In $\wasm ~ (\seq{wc}, k)${}, a Wasm instruction is popped from
the Wasm instruction stack and executed until the Wasm context sequence
$\seq{wc}$ is exhausted.
This execution transitions the continuation to \algo{}, with the relation name
and the arguments of the Wasm instruction, and the current continuation nested
within it.
% Algo
In $\algo ~ (\seq i, \mu, \seq{wc}, k)${}, the body instructions $\seq i$
execute sequentially, allowing transitions to \call{}, \exe{}, \wasm{}, \ret{},
or \algo{}.
% Return
The execution of a function call concludes with a return instruction, which
changes the continuation to $\ret ~ v${}.
If the function call originates from \call{}, the return value $v$ is assigned
to the variable $s_{var}$.
If it originates from \toplevelcall{}, the inner continuation $k$ is always
\mt, so the entire execution concludes with the return value $v$:
$
(\seq a, w, \toplevelcall ~ (s, \seq v))
\leadsto^*
(\seq a, w', \ret ~ (v, \mt))
$.




%% Instruction

\begin{gather*}
  \boxed{\seq{a}, ~ w, ~ c \vdash i \Rightarrow w, ~ k} \\
%
% If-true
\newline \\
  w, \getenv(c) \vdash e \Rightarrow v \qquad
  \istrue(v) \\
  \hline
  \seq{a}, w, c \vdash \ifi ~ (e, \seq{i_1}, \seq{i_2}) \Rightarrow
  (w, \algo ~ (\prependinstr(c, \seq{i_1}))) \\
%
% If-false
\newline \\
  w, \getenv(k) \vdash e \Rightarrow v \qquad
  \neg \istrue(v) \\
  \hline
  \seq{a}, w, c \vdash \ifi ~ (e, \seq{i_1}, \seq{i_2}) \Rightarrow
  (w, \algo ~ (\prependinstr(c, \seq{i_2}))) \\
%
% Either-1
\newline \\
  \\
  \hline
  \seq{a}, w, c \vdash \eitheri ~ (\seq{i_1}, \seq{i_2}) \Rightarrow
  (w, \algo ~ (\prependinstr(c, \seq{i_1}))) \\
%
% Either-2
\newline \\
  \\
  \hline
  \seq{a}, w, c \vdash \eitheri ~ (\seq{i_1}, \seq{i_2}) \Rightarrow
  (w, \algo ~ (\prependinstr(c, \seq{i_2}))) \\
%
% Enter
\newline \\
  (\seq{we}, \seq{wi}, sto) = w \qquad
  (i^*, \mu, wc_1 ~ ... ~ wc_n, k) = c \\
  w, \mu \vdash e_1 \Rightarrow wc \qquad
  w, \mu \vdash e_2 \Rightarrow \listv ~ \seq{wi_1} \\
  wi = \casev ~ (\text{"End"}, wc) \qquad
  wi_1 = \casev ~ (\text{"End"}, wc_1) \quad ... \quad wi_n = \casev ~ (\text{"End"}, wc_n) \\
  wi_2^* = wi_1^* ~ wi ~ wi_1 ~ ... ~ wi_n \\
  \hline
  \seq{a}, w, c \vdash \enteri ~ (e_1, e_2)
  \Rightarrow
  (
    (wc ~ \seq{we}, wi_2^* ~ \seq{wi}, sto),
    \wasm ~ (wc ~ wc_1 ~ ... ~ wc_n, \algo ~ (i^*, \mu, \epsilon, k))
  ) \\
%
% PushCtx
\newline \\
  w, \getenv(c) \vdash e \Rightarrow wc \\
  \hline
  \seq{a}, w, c \vdash \pushctxi ~ e
  \Rightarrow
  (\push(w, wc), \algo ~ (\addctx(c, wc))) \\
%
% Push
\newline \\
  w, \getenv(c) \vdash e \Rightarrow wv \\
  \hline
  \seq{a}, w, c \vdash \pushi ~ e \Rightarrow (\push(w, wv), \algo ~ c) \\
\end{gather*}
\newpage
\begin{gather*}
%
% Exit
\newline \\
  (\seq{we}, \seq{wi}, sto) = w \qquad
  (\seq{wv}, wc, \seq{we'}) = \splitctx(\seq{we}) \\
  \seq{wi'} = \exit(\seq{wi}) \qquad
  c' = \popwasmctx_{C}(c) \\
  \hline
  \seq{a}, w, c \vdash \exiti
  \Rightarrow
  ((\seq{we'}, \seq{wi'}, sto), \algo ~ c') \\
%
% PopCtx
\newline \\
  (\seq{we}, \seq{wi}, sto) = w \qquad
  (\seq{wv}, wc, \seq{we'}) = \splitctx(\seq{we}) \\
  \mu = \getenv(c) \cdot \assign(e, wc) \qquad
  c' = \popwasmctx_{C}(c) \\
  \hline
  \seq{a}, w, c \vdash \popctxi ~ e
  \Rightarrow
  ((\seq{wv} ~ \seq{we'}, \seq{wi}, sto), \algo ~ (\setenv(c', \mu))) \\
%
% Pop
\newline \\
  (w', wv) = \pop(w) \qquad
  \mu' = \getenv(c) \cdot \assign(e, wv) \\
  \hline
  \seq{a}, w, c \vdash \popi ~ e \Rightarrow (w', \algo ~ (\setenv(c, \mu'))) \\
%
% PopN
\newline \\
  \mu = \getenv(c) \qquad
  w, \mu \vdash e_2 \Rightarrow \numv ~ n \\
  (w', wv^n) = \popn(w, n) \qquad
  \mu' = \mu \cdot \assign(e_1, \listv ~ wv^n) \\
  \hline
  \seq{a}, w, c \vdash \popni ~ (e_1, e_2) \Rightarrow
  (w', \algo ~ (\setenv(c, \mu'))) \\
%
% PopAll
\newline \\
  (\seq{wv}, wc, \seq{we'}) = \splitctx(\seq{we}) \qquad
  \mu' = \getenv(c) \cdot \assign(e, \liste ~ \seq{wv}) \\
  \hline
  \seq{a}, (\seq{we}, \seq{wi}, sto), c \vdash \popalli ~ e
  \Rightarrow
  ((wc ~ \seq{we'}, \seq{wi}, sto), \algo ~ (\setenv(c, \mu'))) \\
%
% Let
\newline \\
  \mu = \getenv(c) \qquad
  w, \mu \vdash e_2 \Rightarrow v \qquad
  \mu' = \mu \cdot \assign(e_1, v) \\
  \hline
  \seq{a}, w, c \vdash \leti ~ (e_1, e_2)
  \Rightarrow
  (w, \setenv(c, \mu')) \\
%
% Trap
\newline \\
  \\
  \hline
  \seq{a}, w, c \vdash \trapi \Rightarrow (w, \ret ~ (\trapv, \mt)) \\
%
% Return-rule
\newline \\
  \\
  \hline
  \seq{a}, w, (\seq{i}, \mu, wc^*, k) \vdash \returnreli \Rightarrow (w, k) \\
%
% Return-func
\newline \\
  w, \mu \vdash e \Rightarrow v \\
  \hline
  \seq{a}, w, (\seq{i}, \mu, wc^*, k) \vdash \returnfuni ~ e \Rightarrow
  (w, \ret ~ (v, k)) \\
%
% Execute
\newline \\
  w, \getenv(c) \vdash e \Rightarrow \listv ~ wi^* \\
  \hline
  \seq{a}, w, c \vdash \executei ~ e \Rightarrow
  (w, \exe ~ (wi^*, \algo ~ c)) \\
%
% Call-fname
\newline \\
  \mu = \getenv(c) \qquad
  \fnamev ~ s_3 = \mu(s_2) \qquad
  w, \mu \vdash e_1 \Rightarrow v_1 \quad ... \quad
  w, \mu \vdash e_n \Rightarrow v_n \\
  \hline
  \seq{a}, w, c \vdash \calli ~ (s_1, s_2, e_1 ~ ... ~ e_n) \Rightarrow
  (w, \call ~ (s_1, s_3, v_1 ~ ... ~ v_n, \algo ~ c)) \\
%
% Call-fname
\newline \\
  \mu = \getenv(c) \qquad
  s_2 \not\in \domain(\mu) \qquad
  w, \mu \vdash e_1 \Rightarrow v_1 \quad ... \quad
  w, \mu \vdash e_n \Rightarrow v_n \\
  \hline
  \seq{a}, w, c \vdash \calli ~ (s_1, s_2, \seq e) \Rightarrow
  (w, \call ~ (s_1, s_2, v_1 ~ ... ~ v_n, \algo ~ c)) \\
%
% Replace-frame
\newline \\
  \mu = \getenv(c) \qquad
  \casev ~ (\text{``Frame"}, v_1 ~ v_2) = \getcurframe(w) \qquad
  w, \mu \vdash e \Rightarrow v \\
  wc = \casev ~ (\text{``Frame"}, v_1 ~ \update(w, \mu, v_2, p^+, v)) \\
  \hline
  \seq{a}, w, c \vdash \replaceframei ~ (p^+, e) \Rightarrow
  (\setcurframe(w, wc), \algo ~ c) \\
%
% Replace-store
\newline \\
  \mu = \getenv(c) \qquad
  w, \mu \vdash e \Rightarrow v \\
  \strv ~ sto = \update(w, \mu, \strv ~ (\getstore(w)), p^+, v) \\
  \hline
  \seq{a}, w, c \vdash \replacestorei ~ (p^+, e)
  \Rightarrow (\setstore(w, sto), c) \\
\end{gather*}

Given algorithms $\seq a$, a Wasm state $w$, and a context $c$, executing an
instruction $i$ produces an updated Wasm state and a continuation.
% If
$\ifi ~ (e, i_1^*, i_2^*)${} evaluates $e$ and branches to $\seq i_1$ if the
result is true, or $\seq i_2$ otherwise.
% Either
$\eitheri ~ (i_1^*, i_2^*)${} nondeterministically selects either $i_1^*$ or
$i_2^*$.
% Enter
$\enteri ~ (e_1, e_2)${} evaluates $e_1$, pushing the resulting Wasm context
$wc$ to the interleaved stack.
It then evaluates $e_2$ to obtain the Wasm instruction sequence $wi_1^*$.
Then, the end instructions of the Wasm context $wc$ and the Wasm contexts
$wc_1 ~ ... ~ wc_n$ in the context $c$ are appended to $wi_1^*$ and pushed to
the Wasm instruction stack.
Finally, it changes the continuation to
$\wasm ~ (wc ~ wc_1 ~ ... ~ wc_n, \algo ~ (i^*, \mu, \epsilon, k))${}.
% PushCtx
$\pushctxi ~ e${} evaluates $e$, pushing the resulting Wasm context $wc$ to the
interleaved stack and the context $c$.
% Push
$\pushi ~ e${} evaluates $e$, pushing the resulting value $wv$ to the
interleaved stack.
% Exit
$\exiti${} pops all Wasm values up to and including a Wasm context $wc$ from
the interleaved stack and all Wasm instructions up to and including an end
instruction from the Wasm instruction stack.
It also pops a Wasm context from the context $c$.
% PopCtx
$\popctxi ~ e${} pops a Wasm context $wc$ from the interleaved stack.
It also pops a Wasm context from the context $c$.
Then, it assign the Wasm context $wc$ to $e$.
% Pop
$\popi ~ e${} pops a Wasm value $wv$ from the interleaved stack and assign the
value to $e$.
% PopN
$\popni ~ (e_1, e_2)${} evaluates $e_2$ to obtain a number $n$, pops $n$ Wasm
values from the interleaved stack, and assigns the values to $e_1$.
% PopAll
$\popalli ~ e${} pops all the Wasm values up to, but not including, a
Wasm context from the interleaved stack and assign the values to $e$.
% Let
$\leti ~ (e_1, e_2)${} evaluates $e_2$ and assigns the resulting value to
$e_1$.
% Trap
\trapi{} results in $\ret ~ (\trapv{}, \mt{})$ to terminate the execution.
% ReturnRel
\returnreli{} results in the continuation $k$ in the context $c$, indicating a
return from a relation algorithm.
% ReturnFunc
$\returnfuni ~ e${} evaluates $e$ to $v$, resulting in $\ret ~ (v, k)${} where
$k$ is the continuation in the context $c$.
% Execute
\executei{} evaluates $e$ to obtain a Wasm instruction sequence $wi^*$ and
result in $\exe ~ (wi^*, \algo ~ c)${}.
% Call
$\calli ~ (s_1, s_2, e_1 ~ ... ~ e_n)${} checks if $s_2$ is bound in the
environment.
If it is, it retrieves the function name $s_3$ from the environment $\mu$;
otherwise, it uses $s_2$ as the function name.
It evaluates $\seq e$ to $v^*$, resulting in $\call ~ (s_1, s_{name}, v_1 ~ ...
~ v_n)${} where $s_{name}$ is the function name.
% ReplaceFrame
\replaceframei{} retrieves the current frame $\casev ~ (\text{``Frame"}, v_1 ~
_2)${} in the Wasm context $w$.
It then evaluates $e$ to $v$, updating the value at location $v_2$ along paths
$p^+$ to $v$.
% ReplaceFrame
\replacestorei{} evaluates $e$ to $v$, updating the value at location $\strv ~
sto$ along paths $p^+$ to v.



\newpage

%% Expression

\begin{gather*}
  \boxed{w, \mu \vdash e \Rightarrow v} \\
%
% Var
\newline \\
  v = \mu(s) \\
  \hline
  w, \mu \vdash \vare ~ s \Rightarrow v \\
%
% Num
\newline \\
  \hline
  w, \mu \vdash \nume ~ n \Rightarrow \numv ~ n \\
%
% Bool
\newline \\
  \hline
  w, \mu \vdash \boole ~ b \Rightarrow \boolv ~ b \\
%
% Fname
\newline \\
  \hline
  w, \mu \vdash \fnamee ~ s \Rightarrow \fnamev ~ s \\
%
% List
\newline \\
  w, \mu \vdash e_1 \Rightarrow v_1 \quad ... \quad
  w, \mu \vdash e_n \Rightarrow v_n \\
  \hline
  w, \mu \vdash \liste ~ e_1 ~ ... ~ e_n \Rightarrow \listv ~ v_1 ~ ... ~ v_n \\
%
% Str
\newline \\
  w, \mu \vdash e_1 \Rightarrow v_1 \quad ... \quad
  w, \mu \vdash e_n \Rightarrow v_n \\
  \hline
  w, \mu \vdash \stre ~ (s_1, e_1) ~ ... ~ (s_n, e_n) \Rightarrow
  \strv ~ (s_1, v_1) ~ ... ~ (s_n, v_n) \\
%
% Tup
\newline \\
  w, \mu \vdash e_1 \Rightarrow v_1 \quad ... \quad
  w, \mu \vdash e_n \Rightarrow v_n \\
  \hline
  w, \mu \vdash \tupe ~ e_1 ~ ... ~ e_n \Rightarrow \tupv ~ v_1 ~ ... ~ v_n \\
%
% Case
\newline \\
  w, \mu \vdash e_1 \Rightarrow v_1 \quad ... \quad
  w, \mu \vdash e_n \Rightarrow v_n \\
  \hline
  w, \mu \vdash \casee ~ (s, e_1 ~ ... ~ e_n) \Rightarrow \casev ~ (s, v_1 ~ ... ~ v_n) \\
%
% Un
\newline \\
  w, \mu \vdash e \Rightarrow v \\
  \hline
  w, \mu \vdash \une ~ (unop, e) \Rightarrow \unop(unop, v) \\
%
% Bin
\newline \\
  w, \mu \vdash e_1 \Rightarrow v_1 \qquad w, \mu \vdash e_2 \Rightarrow v_2 \\
  \hline
  w, \mu \vdash \bine ~ (binop, e_1, e_2) \Rightarrow \binop(binop, v_1, v_2) \\
%
% Acc
\newline \\
  w, \mu \vdash e \Rightarrow v \\
  \hline
  w, \mu \vdash \acce ~ (e, p) \Rightarrow \access(w, \mu, v, p) \\
%
% Upd
\newline \\
  w, \mu \vdash e_1 \Rightarrow v_1 \qquad w, \mu \vdash e_2 \Rightarrow v_2 \\
  \hline
  w, \mu \vdash \upde ~ (e_1, p^+, e_2) \Rightarrow \update(w, \mu, v_1, p^+, v_2) \\
%
% Cat
\newline \\
   w, \mu \vdash e_1 \Rightarrow \listv ~ \seq{v_1} \qquad
   w, \mu \vdash e_2 \Rightarrow \listv ~ \seq{v_2} \\
  \hline
  w, \mu \vdash \cate ~ (e_1, e_2) \Rightarrow \listv ~ (\seq{v_1} ~ \seq{v_1}) \\
%
% Comp
\newline \\
   w, \mu \vdash e_1 \Rightarrow \strv ~ \seq{(s_1, v_1)} \qquad
   w, \mu \vdash e_2 \Rightarrow \strv ~ \seq{(s_2, v_2)} \\
  \hline
  w, \mu \vdash \compe ~ (e_1, e_2) \Rightarrow \strv ~ (\seq{(s_1, v_1)} ~ \seq{(s_2, v_2)}) \\
%
% Mem
\newline \\
  w, \mu \vdash e_1 \Rightarrow v_1 \qquad
  w, \mu \vdash e_2 \Rightarrow \listv ~ \seq{v_2} \\
  \hline
  w, \mu \vdash \meme ~ (e_1, e_2) \Rightarrow \boolv ~ (v_1 \in \seq{v_2}) \\
%
% Choose
\newline \\
  w, \mu \vdash e \Rightarrow \listv ~ \seq{v} \qquad
  v \in \seq{v} \\
  \hline
  w, \mu \vdash \choosee ~ e \Rightarrow v \\
%
% Len
\newline \\
  w, \mu \vdash e \Rightarrow \listv ~ \seq{v} \\
  \hline
  w, \mu \vdash \lene ~ e \Rightarrow \numv ~ |\seq v| \\
%
% IsCaseOf
\newline \\
  w, \mu \vdash e \Rightarrow \casev ~ (s', \seq{v}) \\
  \hline
  w, \mu \vdash \iscaseofe ~ (e, s) \Rightarrow \boolv ~ (s = s') \\
%
% GetCurCtx
\newline \\
  wc = \getcurctx(w) \\
  \hline
  w, \mu \vdash \getcurctxe \Rightarrow wc \\
%
% CtxKind
\newline \\
  \casev (s', \seq v) = \getcurctx(w) \\
  \hline
  w, \mu \vdash \ctxkinde ~ s \Rightarrow \boolv ~ (s = s') \\
%
% Iter-
\newline \\
  \,~\mu(s_1) = \listv ~ (v_{11} ~ ... ~ v_{1n}) \quad ... \quad
  \mu(s_m) = \listv ~ (v_{m1} ~ ... ~ v_{mn}) \\
  \mu_1 = \mu[s_1 \mapsto v_{11}, ~ ... ~ s_m \mapsto v_{m1}] \quad ... \quad
  \mu_n = \mu[s_1 \mapsto v_{1n}, ~ ... ~ s_m \mapsto v_{mn}] \\
  w, \mu_1 \vdash e \Rightarrow v_1 \quad ... \quad
  w, \mu_n \vdash e \Rightarrow v_n \\
  \hline
  w, \mu \vdash \itere ~ (e, \listiter, s_1 ~ ... ~ s_m) \Rightarrow
  \listv ~ v_1 ~ ... ~ v_n \\
%
% Iter-n
\newline \\
  w, \mu \vdash e_2 \Rightarrow \numv ~ n \\
  \,~\mu(s_1) = \listv ~ (v_{11} ~ ... ~ v_{1n}) \quad ... \quad
  \mu(s_m) = \listv ~ (v_{m1} ~ ... ~ v_{mn}) \\
  \mu_1 = \mu[s_1 \mapsto v_{11}, ~ ... ~ s_m \mapsto v_{m1}] \quad ... \quad
  \mu_n = \mu[s_1 \mapsto v_{1n}, ~ ... ~ s_m \mapsto v_{mn}] \\
  w, \mu_1 \vdash e \Rightarrow v_1 \quad ... \quad
  w, \mu_n \vdash e \Rightarrow v_n \\
  \hline
  w, \mu \vdash \itere ~ (e_1, \listniter ~ e_2, s_1 ~ ... s_m) \Rightarrow
  \listv ~ v_1 ~ ... ~ v_n \\
%
% Iter-idx
\newline \\
  w, \mu \vdash e_2 \Rightarrow \numv ~ n \\
  \mu_1 = \mu[s_1 \mapsto \numv ~ 0, ~ ... ~ s_m \mapsto \numv ~ 0] \quad ... \quad
  \mu_n = \mu[s_1 \mapsto \numv ~ n, ~ ... ~ s_m \mapsto \numv ~ n] \\
  w, \mu_1 \vdash e \Rightarrow v_1 \quad ... \quad
  w, \mu_n \vdash e \Rightarrow v_n \\
  \hline
  w, \mu \vdash \itere ~ (e_1, \listidxiter ~ e_2, s_1 ~ ... ~ s_m) \Rightarrow
  \listv ~ v_1 ~ ... ~ v_n \\
\end{gather*}
\newpage
\begin{gather*}
%
% Match
\newline \\
  w, \mu \vdash e_1 \Rightarrow v_1 \qquad
  w, \mu \vdash e_2 \Rightarrow v_2 \\
  \hline
  w, \mu \vdash \matche ~ (e_1, e_2) \Rightarrow \boolv ~ (\match(v_1, v_2)) \\
%
% HasType
\newline \\
  w, \mu \vdash e \Rightarrow v \\
  \hline
  w, \mu \vdash \hastypee ~ (e, s) \Rightarrow \boolv ~ (\hastype(v, s)) \\
\end{gather*}

Given a Wasm state $w$ and an environment $\mu$, an expression $e$ evaluates
to a value.
% Var
$\vare ~ s${} looks up the environment $\mu$ and gets the value.
% Num Bool Fname
$\nume ~ n${}, $\boole ~ b${}, and $\fnamee ~ s${} directly reduce to $\numv ~
n${}, $\boolv ~ b${}, and $\fnamev ~ s${}, respectively.
% Tuple Case List Struct
$\liste ~ e_1 ~ ... ~ e_n${}, $\stre ~ (s_1, e_1) ~ ... ~ (s_n, e_n)${}, $\tupe
~ e_1 ~ ... ~ e_n${}, and $\casee ~ (s, e_1 ~ ... ~ e_n)${} evaluate their
sub-expressions and reduce to $\listv ~ v_1 ~ ... ~ v_n${}, $\strv ~ (s_1, v_1)
~ ... ~ (s_n, v_n)${}, $\tupv ~ v_1 ~ ... ~ v_n${}, and $\casev ~ (s, v_1 ~ ...
~ v_n)${}, respectively.
% Un Bin
$\une ~ (unop, e)${} and $\bine ~ (binop, e_1, e_2)${} evaluate their
sub-expressions and perform \unop{} and \binop{}, respectively.
% Acc
$\acce ~ (e, p)${} evaluates $e$ and accesses the resulting value $v$ using a
path $p$.
% Upd
$\upde ~ (e_1, p^+, e_2)${} evaluates $e_1$ and $e_2$ to $v_1$ and $v_2$,
updating the value at location $v_1$ along paths $p^+$ to the value $v_2$.
% Cat Comp
$\cate ~ (e_1, e_2)${} and $\compe ~ (e_1, e_2)${} each evaluate two
sub-expressions $e_1$ and $e_2$, concatenating the resulting lists and
resulting records, respectively.
% Mem
$\meme ~ (e_1, e_2)${} evaluates $e_1$ and $e_2$ to $v_1$ and $\listv ~ \seq
v_2$, and checks whether $v_1$ is a member of $\seq v_2$.
% Choose Len
$\choosee ~ e$ and $\lene ~ e${} each evaluate $e$ to the list value $\listv ~
\seq v$, choosing an element from the list and calculating the length of the
list, respectively.
% IsCaseOf
$\iscaseofe ~ (e, s)${} evaluates $e$ to the tagged tuple $\casev ~ (s', \seq
v)$ and checks whether the tag $s'$ matches $s$.
% GetCurCtx
\getcurctxe{} retrieves the current Wasm context from the interleaved stack.
% CtxKind
$\ctxkinde ~ s${} retrieves the current Wasm context $\casev ~ (s', \seq v)$
and checks whether the kind $s'$ matches $s$.
% Iter
$\itere ~ (e, iter, s^*)${} behaves differently depending on $iter$.
\listiter{} and $\listniter ~ e_2${} are essentially equivalent.
They assume that all the values mapped to the variables $\seq s$ by the
environment $\mu${} are list values with the same length $n$.
$\listniter ~ e_2${} checks whether the number $n$ matches the result of
evaluating $e_2$, while \listiter{} allows any number.
It generates a sequence of environments $\mu_1 ~ ... ~ \mu_n$ where the k-th
environment maps each variable to the k-th elements of each list.
Each environment is used to evaluate $e_1$, producing a sequence of values $v_1
~ ... ~ v_n$.
Finally, it results in $\listv ~ v_1 ~ ... ~ v_n$.
$\listidxiter ~ e_2${} behaves similarly except that the variables $\seq s$ are
unbound.
It evaluates $e_2$ to obtain $n$ and then behaves like $\listniter ~ e_2${} as
if a list of values [0 ... n-1] were bound to the variables $s^*$.
% IsValid Match HasType
$\matche ~ (e_1, e_2)${} and $\hastypee ~ (e, s)${} evaluate their
sub-expressions and perform \match{} and \hastype{}, respectively.
Note that \match{} and \hastype{} represent the match relation and the type
relation in the Wasm static semantics.
Their formalization is omitted as they are highly specific to Wasm and
excessively verbose.




%% Helper functions

\begin{align*}
%
% get_algo_name
\newline \\
  &\getalgoname(\rel ~ (s, \seq e, \seq i)) = s \\
  &\getalgoname(\fun ~ (s, \seq e, \seq i)) = s \\
%
% lookup
\newline \\
  &\lookup(\epsilon, s) = \epsilon \\
  &\lookup(a ~ \seq{a}, s) =
    \begin{cases}
      a &\quad\quad \premise{if}\quad ~ s = \getalgoname(a) \\
      \lookup(\seq{a}, s) &\quad\quad \premise{otherwise}
    \end{cases}
  \\
%
% assign
\newline \\
  &\assign(\vare ~ s, v) = [s \mapsto v] \\
  &\assign(\itere ~ (e, iter, s^*), v) = \assign(e, v) \\
  &\assign(\tupe ~ e_1 ~ ... ~ e_n, \tupv ~ v_1 ~ ... ~ v_n) =
    \assign(e_1, v_1) \cdot ~ ... ~ \cdot \assign(e_n, v_n) \\
  &\assign(\liste ~ e_1 ~ ... ~ e_n, \listv ~ v_1 ~ ... ~ v_n) =
    \assign(e_1, v_1) \cdot ~ ... ~ \cdot \assign(e_n, v_n) \\
  &\assign(\casee ~ (s, e_1 ~ ... ~ e_n), \casev ~ (s, v_1 ~ ... ~ v_n)) =
    \assign(e_1, v_1) \cdot ~ ... ~ \cdot \assign(e_n, v_n) \\
  &\assign(
    w,
    \mu,
    \stre ~ (s_1, e_1) ~ ... ~ (s_2, e_n),
    \strv ~ (s_1, v_1) ~ ... ~ (s_n, v_n)
  ) =
    \assign(e_1, v_1) \cdot ~ ... ~ \cdot \assign(e_n, v_n) \\
  &\assign(\cate ~ (\liste ~ e_1 ~ ... ~ e_m, e), \listv v_1 ~ ... ~ v_n) =
    \assign(e_1, v_1) \cdot ~ ... ~ \cdot \assign(e_m, v_m) \cdot
    \assign(e, \listv ~ v_{m+1} ~ ... ~ v_n) \\
  &\assign(\cate ~ (e, \liste ~ e_1 ~ ... ~ e_m), \listv v_1 ~ ... ~ v_n) =
    \assign(e, \listv ~ v_1 ~ ... ~ v_{n-m}) \cdot
    \assign(e_1, v_{n-m+1}) \cdot ~ ... ~ \cdot \assign(e_m, v_n) \\
%
% create_algo
\newline \\
  &\createalgo(\seq a, s, \seq v, k) =
  \algo(\seq i, \mu, \epsilon, k) \\
  &\qquad\qquad\qquad\qquad\premise{if}\quad
  (\rel ~ (s, \seq e, \seq i) \quad\lor\quad \fun ~ (s, \seq e, \seq i)) = \lookup(\seq a, s)
  \quad\land\quad
  \mu = \assign(w, [\text{``s"} \mapsto \storev], \seq e, \seq v) \\
%
% split_ctx
\newline \\
  &\splitctx(\epsilon) = (\epsilon, \epsilon, \epsilon) \\
  &\splitctx(we ~ \seq{we}) =
    \begin{cases}
      (\epsilon, wc, \seq{we})
      &\quad\quad \premise{if}\quad
      we = wc \\
      (wv ~ \seq{wv}, wc, \seq{we'})
      &\quad\quad \premise{if}\quad
      we = wv \quad\land\quad
      (\seq{wv}, wc, \seq{we'}) = \splitctx(\seq{we}) \\
    \end{cases}
  \\
%
% pop-wasm-ctx_context
\newline \\
  &\popwasmctx_{C}((\seq i, \mu, epsilon, k)) = (\seq i, \mu, \epsilon, \popwasmctx_K(k)) \\
  &\popwasmctx_{C}((\seq i, \mu, wc ~ \seq{wc}, k)) = (\seq i, \mu, wc^*, k) \\
%
\newline \\
% pop-wasm-ctx_cont-mt
  &\popwasmctx_{K}(\mt) = \epsilon \\
% pop-wasm-ctx_cont-toplevelcall
  &\popwasmctx_{K}(\toplevelcall ~ (s, v^*)) = \epsilon \\
% pop-wasm-ctx_cont-call
  &\popwasmctx_{K}(\call ~ (s_1, s_2, \seq v, k)) =
    \call ~ (s_1, s_2, \seq v, (\popwasmctx_K(k)) \\
% pop-wasm-ctx_cont-execute
  &\popwasmctx_{K}(\exe ~ (v, k)) = \exe ~ (v, (\popwasmctx_K(k)) \\
% pop-wasm-ctx_cont-wasm_0
  &\popwasmctx_{K}(\wasm ~ (\epsilon, k)) = \wasm ~ (\epsilon, \popwasmctx_{K}(k)) \\
% pop-wasm-ctx_cont-wasm_nonzero
  &\popwasmctx_{K}(\wasm ~ (wc ~ \seq{wc}, k)) = \wasm ~ (\seq{wc}, k) \\
% pop-wasm-ctx_cont-al
  &\popwasmctx_{K}(\algo ~ c) = \algo ~ (\popwasmctx_C(c)) \\
% pop-wasm-ctx_cont-ret
  &\popwasmctx_{K}(\ret ~ (v, k)) = \ret ~ (v, \popwasmctx_K(k)) \\
%
% exit
\newline \\
  &\exit(wi ~ \seq{wi}) =
    \begin{cases}
      \seq{wi} &\quad\quad \premise{if}\quad ~ wi = \casev ~ (\text{"End"}, wc) \\
      \exit(\seq{wi}) &\quad\quad \premise{otherwise} \\
    \end{cases}
  \\
%
% get_env
\newline \\
  &\getenv((\seq i, \mu, wc^*, k)) = \mu \\
%
% set_env
\newline \\
  &\setenv((\seq i, \mu, wc^*, k), \mu') = (\seq i, \mu', wc^*, k) \\
%
% prepend_instr
\newline \\
  &\prependinstr((\seq i, \mu, wc^*, k), \seq{i'}) = (\seq{i'} ~ \seq i, \mu, wc^*, k) \\
%
% add_ctx
\newline \\
  &\addctx((\seq i, \mu, wc^*, k), wc') = (\seq i, \mu, wc' ~ wc^*, k) \\
%
% get_store
\newline \\
  &\getstore((\seq{we}, \seq{wi}, sto)) = sto \\
%
% set_store
\newline \\
  &\setstore((\seq{we}, \seq{wi}, sto), sto') = (\seq{we}, \seq{wi}, sto') \\
%
% pop_wasm_instr
\newline \\
  &\popwasminstr((\seq{we}, wi ~ \seq{wi}, sto)) = ((\seq{we}, \seq{wi}, sto), wi) \\
%
% push
\newline \\
  &\push((\seq{we}, \seq{wi}, sto), we) = (we ~ \seq{we}, \seq{wi}, sto) \\
%
% pop
\newline \\
  &\pop((we ~ \seq{we}, \seq{wi}, sto)) = ((\seq{we}, \seq{wi}, sto), we) \\
%
% popn
\newline \\
  &\popn((we^n ~ \seq{we}, \seq{wi}, sto), n) = ((\seq{we}, \seq{wi}, sto), we^n) \\
%
% unop
\newline \\
  &\unop(\notop, v) =
    \begin{cases}
      \boolv ~ true &\quad\quad\premise{if}\quad ~ \istrue(v) \\
      \boolv ~ false &\quad\quad\premise{otherwise} \\
    \end{cases}
  \\
  &\unop(\minusop, \numv ~ n) = \numv ~ (-n) \\
%
% get_cur_ctx
\newline \\
  &\getcurctx((\seq{we}, \seq{wi}, sto)) =
  wc \quad\quad\premise{if}\quad ~ (\seq{wv}, wc, \seq{we'}) = \splitctx(\seq{we}) \\
%
% get_cur_frame-frame
\newline \\
  &\getcurframe((\seq{we}, \seq{wi}, sto)) =
    \begin{cases}
      wc &\quad\quad\premise{if}\quad
      (\seq{wv}, wc, \seq{we'}) = \splitctx(\seq{we}) \quad\land\quad \isframe(wc)\\
      \getcurframe(\seq{we'}) &\quad\quad\premise{if}\quad
      (\seq{wv}, wc, \seq{we'}) = \splitctx(\seq{we}) \quad\land\quad \neg \isframe(wc) \\
    \end{cases}
  \\
%
% set_cur_frame-frame
\newline \\
  &\setcurframe((\seq{we}, \seq{wi}, sto), wc_{frame})
  =
  \begin{cases}
    (\seq{wv} ~ wc_{frame} ~ \seq{we'}, \seq{wi}, sto) \\
    \qquad\qquad\qquad\quad\premise{if}\quad
    (\seq{wv}, wc, \seq{we'}) = \splitctx(\seq{we}) \quad \isframe(wc) \\
    (\seq{wv} ~ wc ~ \setcurframe(\seq{we'}, wc_{frame}), \seq{wi}, sto) \\
    \qquad\qquad\qquad\quad\premise{if}\quad
    (\seq{wv}, wc, \seq{we'}) = \splitctx(\seq{we}) \quad \neg \isframe(wc) \\
  \end{cases}
  \\
%
% access-idx
\newline \\
  &\access(w, \mu, \listv ~ \seq{v}, \idxp ~ e) = \seq{v}[n]
  \qquad\qquad\qquad~\,\quad\premise{if}\quad w, \mu \vdash e \Rightarrow \numv ~ n \\
% access-slice
  &\access(w, \mu, \listv ~ \seq{v}, \slicep ~ (e_1, e_2)) = \seq{v}[n_1: n_2]
  \quad\quad\premise{if}\quad
  w, \mu \vdash e_1 \Rightarrow \numv ~ n_1 \quad\land\quad
  w, \mu \vdash e_2 \Rightarrow \numv ~ n_2 \\
% access-dot
  &\access(w, \mu, \strv ~ ((s_0, v_0) ~ \seq{(s_1, v_1)}), \dotp ~ s) =
  \begin{cases}
    v_0 &\quad\quad\premise{if}\quad s_0 = s \\
    \access(w, \mu, \strv ~ \seq{(s_1, v_1)}, \dotp ~ s) &\quad\quad\premise{otherwise} \\
  \end{cases}
  \\
% access-store
  &\access(w, \mu, \storev, p) =\access(w, \mu, \strv ~ sto, p)
  \quad\quad\premise{if}\quad w = (\seq{we}, \seq{wi}, sto)\\
%
% update
\newline \\
  % idx
  &\update(w, \mu, \listv ~ \seq{v}, (\idxp ~ e) ~ \seq{p}, v) =
  \listv ~ (\updateidx(\seq v, n, \update(w, \mu, \seq{v}[n], \seq{p}, v))) \\
  &\qquad\qquad\qquad\qquad\qquad\qquad\qquad\qquad\qquad\qquad\qquad\qquad\qquad\qquad\qquad\qquad\qquad\qquad\qquad\qquad\premise{if}\quad
  w, \mu \vdash e \Rightarrow \numv ~ n \\
  % slice
  &\update(w, \mu, \listv ~ \seq{v}, (\slicep ~ e) ~ \seq{p}, v) =
  \listv ~ (\updateslice(\seq v, n_1, n_2, \update(w, \mu, \seq{v_i}[n_1: n_2], \seq{p}, v))) \\
  &\qquad\qquad\qquad\qquad\qquad\qquad\qquad\qquad\qquad\qquad\qquad\qquad\qquad\premise{if}\quad
  w, \mu \vdash e_1 \Rightarrow \numv ~ n_1 \quad\land\quad
  w, \mu \vdash e_2 \Rightarrow \numv ~ n_2 \\
  % dot
  &\update(w, \mu, \strv ~ \seq{(s, v)}, (\dotp ~ s') ~ \seq{p}, v)
  =
  \strv ~ (\updatedot(\seq{(s, v)}, s', \update(w, \mu, \access(v, \dotp ~ s'), \seq{p}, v))) \\
%
% update_idx
\newline \\
  &\updateidx(v_0 ~ \seq{v_1}, n, v_2) =
  \begin{cases}
    v_2 ~ \seq{v_1}
    &\quad\quad\premise{if}\quad n = 0 \\
    v_0 ~ \updateidx(\seq{v_1}, n-1, v_2)
    &\quad\quad\premise{if}\quad n > 0 \\
  \end{cases} \\
%
% update_slice
\newline \\
  &\updateslice(v_0^n ~ \seq{v_1}, 0, n, v_2^n) = v_2^n ~ \seq{v_1} \\
  &\updateslice(v_0 ~ \seq{v_1}, m, n, v_2^n) =  v_0 ~ \updateslice(\seq{v_1}, m-1, n, v_2)
  \quad\quad\premise{if}\quad m > 0 \\
%
% update_dot
\newline \\
  &\updatedot((s_0, v_0) ~ \seq{(s_1, v_1)}, s_2, v_2) =
  \begin{cases}
    (s_2, v_2) ~ \seq{(s_1, v_1)}
    &\quad\quad\premise{if}\quad s_0 = s_2 \\
    (s_0, v_0) ~ \updatedot(\seq{(s_1, v_1)}, s_2, v_2)
    &\quad\quad\premise{otherwise}
  \end{cases}
  \\
%
% is_true
\newline \\
  &\istrue(v) =
  \begin{cases}
  b &\quad\quad\premise{if}\quad v = \boolv ~ b \\
    \istrue(v_1) ~ \land ~ ... ~ \land ~ \istrue(v_n) &\quad\quad\premise{if}\quad
    v = \listv ~ v_1 ~ ... ~ v_n \\
  \end{cases} \\
%
% is_frame
\newline \\
  &\isframe(v) =
  \begin{cases}
    true
    &\quad\quad\premise{if}\quad v = \casev ~ (\text{"FRAME"}, \seq v) \\
    false &\quad\quad\premise{otherwise}\\
  \end{cases}
  \\
\end{align*}




\section{An Executable WebAssembly Specification}
\label{sec:exec-spec}

% AL spec
SpecTec translates declarative definitions in the DSL into algorithms in AL.
\cref{ex:al-testop} is the generated algorithm from \cref{ex:dsl-testop},
representing the algorithmic specification for the \texttt{testop} instruction:
\\
\begin{example}
\begin{lstlisting}[style=al]
@\textbf{Step\_pure/testop}@ nt testop {
    Assert (check_type_of_stack_top(nt))
    Pop (CONST nt c_1)
    Let c = @$\dollarcode$@testop_(nt, testop, c_1)
    Push (CONST I32 c)
}
\end{lstlisting}
  \label{ex:al-testop}
\end{example}
SpecTec generates reStructuredText code from the AL algorithms.
The upper part of \cref{fig:spectec-testop} shows the rendered result of
reStructuredText code for the \texttt{testop} instruction.


% Executable
A notable feature of AL is executability, supported by SpecTec's AL interpreter
following the syntax and the semantics of AL.
This executability allows the specification itself to be executed, enabling the
execution of the specified language, Wasm.
The execution of a Wasm program begins with two entry points: module
instantiation and function invocation, both defined within the specification as
algorithms.
Executing these algorithms performs operations for executing the Wasm program,
ultimately producing its execution result.
In this way, the execution of a Wasm program is realized by the execution of
the specification itself.


% Correctness
That is to say, the prose specification written in AL, together with the AL
interpreter, functions as a Wasm interpreter which faithfully follows the
written specification.
By testing this Wasm interpreter, we can test the specification, thereby
ensuring its correctness.


\chapter{Interpreter for Alogrithmic Specification Language}
\label{ch:interpreter}
\noindent



% !TEX root = main.tex

\chapter{Evaluation}
\label{ch:eval}
\noindent

We evaluate our approach with the following three research questions.
\begin{itemize}
  \item RQ1. (Correctness) Does the formalization correctly capture the Wasm
    control flow? (\S 5.1)
  \item RQ2. (Generality) Does the formalization capture the fundamental
    control flow semantics rather than being tailored to the current version of
    the specification? (\S 5.2)
  \item RQ3. \red{(Efficiency) Does our approach efficient? (or scalable?) (\S
    5.3)}
\end{itemize}

\section{RQ1: Correctness}
To demonstrate that our formalization correctly solves the problem, we revisit
\cref{ex:bug}.
The code below is a simplified version of Wasm interpreter program written in AL.
\begin{verbatim}
(* Top-level function: invoke *)
invoke funidx:
  PushCtxI Frame
  EnterI (Label(\epsilon), get_function_body(funidx))
  ReturnFunI get_return_values(funidx)

(* Wasm instructions *)
Br 0:
  PopCtxI Label(instr*)
  ExecuteI instr*

Unreachable:
  TrapI

(* Admin instructions *)
Exiting:
  PopCtxI Label(instr*)

Returning:
  PopCtxI Frame

\end{verbatim}
Using the semantics defined in \cref{ch:formal} together with the AL program,
\cref{ex:bug} produces the same as the officialp{}; invoking the function has
no effect.
We adopt the simplified notation $wc \vdash wi^*$ for a Wasm state, $L$ for a
label and $F$ for a frame, as in the notation from \cref{sec:spectecp}.
Additionally, we omit the environment argument in \algo{} and the arguments of AL
instructions.

\[
\begin{array}{l}
  \epsilon \vdash \epsilon; ~ \toplevelcall(\text{``invoke"}, \numv ~ (\helper{funidx\_of}(\text{``\$brAndReturning"})))
  \\
  \leadsto \epsilon \vdash \epsilon; ~ \algo ~ (\pushctxi ~ \enteri ~ \returnfuni, ~ \epsilon, ~ \mt)
  \\
  \leadsto F \vdash \epsilon; ~ \algo ~ (\enteri ~ \returnfuni, ~ F, ~ \mt)
  \\
  \leadsto L ~ F \vdash (br ~ \epsilon) ~ (unreachable) ~ (exiting) ~ (returning); \wasm ~ (L ~ F, ~ \algo ~ (\returnfuni, ~ \epsilon, ~ \mt))
  \\
  \leadsto L ~ F \vdash (unreachable) ~ (exiting) ~ (returning); \\
  \quad\qquad\qquad\qquad\qquad \algo ~ (\popctxi ~ \executei, ~ \epsilon, ~ \wasm ~ (L ~ F, ~ \algo ~ (\returnfuni, ~ \epsilon, ~ \mt)))
  \\
  \leadsto F \vdash (returning); ~
    \algo ~ (\executei, ~ \epsilon, ~ \wasm ~ (F, ~ \algo ~ (\returnfuni, ~ \epsilon, ~ \mt)))
  \\
  \leadsto F \vdash (returning); ~
    \algo ~ (\epsilon, ~ \epsilon, ~ \wasm ~ (F, ~ \algo ~ (\returnfuni, ~ \epsilon, ~ \mt)))
  \\
  \leadsto F \vdash (returning); ~ \wasm ~ (F, ~ \algo ~ (\returnfuni, ~ \epsilon, ~ \mt))
  \\
  \leadsto F \vdash \epsilon; ~
    \algo ~ (\popctxi, ~ \epsilon, ~ \wasm ~ (\epsilon, ~ \algo ~ (\returnfuni, ~ \epsilon, ~ \mt)))
  \\
  \leadsto \epsilon \vdash \epsilon; ~
    \algo ~ (\epsilon, ~ \epsilon, ~ \wasm ~ (\epsilon, ~ \algo ~ (\returnfuni, ~ \epsilon, ~ \mt)))
  \\
  \leadsto \epsilon \vdash \epsilon; ~ \wasm ~ (\epsilon, ~ \algo ~ (\returnfuni, ~ \epsilon, ~ \mt))
  \\
  \leadsto \epsilon \vdash \epsilon; ~ \algo ~ (\returnfuni, ~ \epsilon, ~ \mt)
  \\
  \leadsto \epsilon \vdash \epsilon; ~ \ret ~ (\epsilon, ~ \mt)
  \\
\end{array}
\]

Furthermore, we implement AL interpreter according to the formalization to
evaluate its correctness.
We write the Wasm specification using the DSL and generat a Wasm interpreter
program written in AL.
By executing the program with the AL interpreter against the official Wasm test
suite, we indirectly test the AL interpreter.
We test it using execution tests from official Wasm test suite.
Tests related to inifinite loop detection are excluded, as the specification
does not define this behavior.
The result shows that the interpreter successfully pass all 51,549 tests in 70
seconds on \red{(lambda 9)}.


\section{RQ2: Generality}
The Wasm exception handler proposal introduces a new control structure,
\textit{handler}, along with new control instructions.
In our formalization, we initially assumed that a context is either a label or
a frame.
By simply extending the formalization to include a handler case, as shown
below, AL can effectively describe the exception handler semantics:

\begin{align*}
%
% Wasm Context
  \text{Wasm Context}\quad ~ WC ~\ni~ wc ~::=~
    ~ \casev ~ (\text{``Label"}, \seq v)
    ~ | ~ \casev ~ (\text{``Frame"}, \seq v)
    ~ | ~ \casev ~ (\text{``Handler"}, \seq v) \\
\end{align*}
% \begin{align*}
% %
% % get_end_algo
% \newline \\
%   &\getendinstr(wc) =
%   \begin{cases}
%     \casev ~ (\text{``Exiting"}, \epsilon) &\quad\quad\premise{if}\quad wc = \casev ~ (\text{``Label"}, \seq v) \\
%     \casev ~ (\text{``Returning"}, \epsilon) &\quad\quad\premise{if}\quad wc = \casev ~ (\text{``Frame"}, \seq v) \\
%     \casev ~ (\text{``Exiting-handler"}, \epsilon) &\quad\quad\premise{if}\quad wc = \casev ~ (\text{``Handler"}, \seq v) \\
%   \end{cases} \\
% %
% % is_end_instr
% \newline \\
%   &\isendinstr(wi) =
%   \begin{cases}
%     true &\quad\quad\premise{if}\quad wi = \casev ~ (\text{``Exiting"}, \epsilon) \\
%     &\quad\quad\lor\quad
%     wi = \casev ~ (\text{``Returning"}, \epsilon) \\
%     &\quad\quad\lor\quad
%     wi = \casev ~ (\text{``Exiting-handler"}, \epsilon) \\
%     false &\quad\quad\premise{otherwise}
%   \end{cases} \\
% \end{align*}

To evaluate the extended formalization, we updated the AL interpreter
accordingly.
We incorporated the exception handler feature into the specification and
generated a Wasm interpreter.
The interpreter successfully passed all 51,614 execution tests in the exception handlers
proposal, excluding infinite loop tests, in 116 seconds.

\section{\red{RQ3: Efficiency}}
The Wasm tail call proposal introduces tail call optimization so that it
contains tests with heavy recursive calls.
As a result, running the test with previous AL interpreter results in a call
stack overflow.
To evaluate the \red{efficiency} of our approach, we similarly generate a Wasm
interpreter including tail call features.
It successfully passes all 51,639 execution tests in the tail call proposal,
excluding infinite loop tests, in 1,284 seconds.


% !TEX root = main.tex

\chapter{Related Work}
\label{ch:related}
\noindent


% TODO: Fact check

\textbf{General purpose language framework.}
% Lem
Lem~\cite{lem} is a language and tool for developing large-scale, rigorous
semantics; it supports multiple output formats (e.g., LaTeX, Coq, and OCaml) to
integrate formal specifications with proof tools and implementations.
While Lem itself does not directly generate interpreters, its OCaml backend can
be used to implement one.
% PLT
PLT Redex~\cite{plt} is a Racket-based domain-specific language for specifying,
experimenting with, and testing operational semantics, making it particularly
suited for exploring small-step semantics and reduction systems.
It provides an interactive environment for evaluating terms, which is akin to
generating an interpreter.
% K
K Framework~\cite{k} is a formal language for defining programming languages
and formal analysis tools; it enables the generation of interpreters,
compilers, and formal verification frameworks directly from the semantics,
making it a powerful tool for executable formal semantics.
% Sail
Sail~\cite{sail} is a domain-specific language designed for the formal
specification of hardware architectures, particularly processors, using an
abstract, executable semantics.
While it is primarily used for hardware modeling, its general framework can
also be applied to other systems for language.
However, none of these tools are capable of specifying an algorithmic
specification.


\textbf{Specialized language framework.}
% ESMeta
ESMeta~\cite{esmeta} is a language framework specialized for JavaScript.
It parses the JavaScript specification, ECMAScript, to automatically generate
parser and interpreter ~\cite{jiset}, to generate tests and perform
differential tests ~\cite{jest}, to find meta-level errors in the specification
~\cite{jstar}, and to perform meta-level static analysis for the static
analysis of the defined language ~\cite{jsaver}.
However, it is not capable of specifying declarative specification.
% ASL
ASL~\cite{asl} is a specialized language framework designed for specifying the
behavior and formal semantics of ARM instruction sets.
The tool also parses the algorithmic specification of the ARM architecture,
enabling verification of ARM processors through bounded model checking.
Additionally, it refines the specification by leveraging methods such as
testing.
% ASL to Sail
ASL to Sail project~\cite{asl2sail} translates ASL into Sail to enable formal
verification, analysis, and multi-platform compatibility of ARM architecture
semantics.
Hence, ASL can specify both declarative and algorithmic specifications;
however, it takes an opposite approach compared to SpecTec, translating the
algorithmic specification into a declarative form.


% !TEX root = main.tex

\chapter{Conclusion}
\label{ch:conclusion}
\noindent


This thesis introduces SpecTec, a framework designed to mechanize the
specification of Wasm.
It offers a domain specific language that facilitates the declarative
definition of Wasm's syntax and semantics, closely resembling formal notation.
SpecTec incorporates type-checking within the DSL to prevent errors in
meta-level specifications and translates them into prose specifications.

To formally define the semantics of Wasm in the prose specification, we propose
a rigorous model of Wasm’s control flow, define the syntax and semantics of the
AL language for specifying the prose, and implement an AL interpreter.
By leveraging this interpreter, specifications written in AL are executable,
functioning as a Wasm interpreter.

Executing these specifications enables the direct evaluation of the Wasm
specification.
This framework bridges the gap between formal specification and executable
code, ensuring their correctness and fidelity.




\include{bib}




%%
%% 감사의 글 시작
%% Acknowledgement
%%
% @command acknowledgement 감사의글
% @options [1 | 2 | 3 |4 ]
% - 1 : 본문과 감사의 글이 둘 다 한글일 때  | 2 : 본문은 한글인데 감사의 글이 영어일 때 | 3 :  본문과 감사의 글이 둘 다 영어일 때  | 4 : 본문은 영어인데 감사의 글이 % 한글일 때 
%% It is optional.

\acknowledgment[4]
% TODO: ack
악

%%
%% 약력 시작
%% Curriculum Vitae
%%
% @command curriculumvitae 이력서
% @options [1 | 2 | 3 |4 ]
% - 1 : 본문과 약력이 둘 다 한글일 때  | 2 : 본문은 한글인데 약력이 영어일 때 | 3 :  본문과 약력이 둘 다 영어일 때  | 4 : 본문은 영어인데 약력이 한글일 때 
%% It is optional and you can change form of this in the class file if you want.
\curriculumvitae[4]

  % @environment personaldata 개인정보
  % @command     name         이름
  %              dateofbirth  생년월일
  %              birthplace   출생지
  %              domicile     본적지
  %              address      주소지
  %              email        E-mail 주소
  % - 위 6개의 기본 필드 중에 이력서에 적고 싶은 정보를 입력
  % input data only you want
  \begin{personaldata}
    \name       {신 원 호}
    \dateofbirth{1999}{06}{05}
  \end{personaldata}

  % @environment education 학력
  % @options [default: (none)] - 수학기간을 입력
  \begin{education}
    \item[2015. 3.\ --\ 2018. 2.] 고등학교 (3년 졸업)
    \item[2018. 2.\ --\ 2023. 2.] 한국과학기술원 전산학부 (학사)
    \item[2023. 3.\ --\ 2025. 2.] 한국과학기술원 전산학부 (석사)
  \end{education}

  % @environment career 경력
  % @options [default: (none)] - 해당기간을 입력
  \begin{career}
    % TODO
    \item[2023. 3.\ --\ 2025. 2.] 한국과학기술원 전산학부 조교
    \item[2024. 10.\ --\ 2024. 10.] 한국과학기술원 홍재민 박사 결혼식 참석
  \end{career}

  % @environment activity 학회활동
  % @options [default: (none)] - 활동내용을 입력
%%  \begin{activity}
%%    \item J. Choi, \textbf{Yong-Hyun Kim}, K.J. Chang, and D. Tomanek,
%%      \textit{Occurrence of itinerant ferromagnetism in C/BN superlattice
%%      nanotubes}, 5th Asian Workshop on First-Principles Electronic
%%      Structure Calculations, Seoul (Korea), October., 2002.
%%  \end{activity}

  % @environment publication 연구업적
  % @options [default: (none)] - 출판내용을 입력
  \begin{publication}
    % TODO
    %\item \textbf{Yong-Hyun Kim}, J. Choi, K.J. Chang, and D. Tomanek,
    %   \textit{Magnetic instability in partly opened C$_{60}$ isomers},
    %   in preparation.
    \item H.-K. Min, Y. Hou, {\bf S. Park}, and I. Song,
``A computationally efficient scheme for feature extraction with kernel discriminant analysis,"
\textit{Patt. Recogn.}, vol.~50, no.~2, pp.~45-55, Feb. 2016 (to be published).
  \end{publication}

  \label{paperlastpagelabel}     % <-- 추가 부분: 마지막 페이지 위치 지정	
%% 본문 끝
\end{document}
