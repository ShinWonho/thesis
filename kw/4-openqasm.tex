% !TEX root = main.tex

\chapter{Desugaring from OpenQASM to OpenQASMCore}
\label{ch:openqasm}

\noindent
In this section, we present the syntax of OpenQASM 2.0
(\cref{ch:openqasm:syntax}) and define desugaring from OpenQASM to
OpenQASMCore (\cref{ch:openqasm:desugaring}).
%
This enables us to indirectly define the executable formal semantics of
OpenQASM and naturally achieve the physical consistency of OpenQASM.

\section{Syntax of OpenQASM}
\label{ch:openqasm:syntax}

\begin{figure}[t]
	\[
		\begin{array}{@{}rrrll@{}}
			\text{Identifier}        & x                            & \multicolumn{1}{c}{\in} & \embox{Id}                                                                \\
			\text{Length/Index}      & n                            & \multicolumn{1}{c}{\in} & \mathbb{N}                                                                \\
			\text{Program}           & p                            & ::=                     & \rep{d};\rep{S}                                                           \\
			\text{Declaration}       & d                            & ::=                     & \dcreg{x_c}{n}                             & \text{Classical register}    \\
			                         &                              & |                       & \dqreg{x_q}{n}                             & \text{Quantum register}      \\
			                         &                              & |                       & \dgate{x_g}{\rep{x_r}}{\rep{x_q}}{\rep{g}} & \text{Custom gate}           \\
			\text{Gate operation}    & g                            & ::=                     & \irotate{e}{e}{e}{a_q}                     & \text{Rotation}              \\
			                         &                              & |                       & \icnot{a_q}{a_q}                           & \text{Cnot}                  \\
			                         &                              & |                       & \icustom{x_g}{\rep{e}}{\rep{a_q}}          & \text{Custom gate}           \\
			\text{Expression}        & e                            & ::=                     & r                                          & \text{Real number}           \\
			                         &                              & |                       & x_r                                        & \text{Real number parameter} \\
			                         &                              & |                       & e+e\quad|\quad\cdots                       & \text{Math operations}       \\
			\text{Argument}          & a                            & ::=                     & x[n]
			                         & \text{Quantum/classical bit}                                                                                                       \\
			                         &                              & |                       & x                                          & \text{Register}              \\
			\text{Statement}         & S                            & ::=                     & Q                                          & \text{Quantum operation}     \\
			                         &                              & |                       & \iif{x_c}{n}{Q}                            & \text{Conditional}           \\
			\text{Quantum operation} & Q                            & ::=                     & g                                          & \text{Gate operation}        \\
			                         &                              & |                       & \imeasure{a_q}{a_c}                        & \text{Measurement}           \\
			                         &                              & |                       & \ireset{a_q}                               & \text{Reset}                 \\
		\end{array}
	\]
	\caption{Syntax of OpenQASM}
	\label{fig:openqasm-syntax}
\end{figure}

\noindent
The syntax of OpenQASM is outlined in \cref{fig:openqasm-syntax}.
%
The metavariable $x$ ranges over identifiers, which serve as names for
classical registers, quantum registers, custom gates, and real-number
parameters of custom gates.
%
Although these various kinds of names are not syntactically differentiated, we
denote them with $x_c$, $x_q$, $x_g$, and $x_r$ in the syntax to provide
clarity regarding the role of an identifier in each context.
%
The metavariable $n$ ranges over natural numbers, which are used as the lengths
of registers and indices to access specific bits within registers.
%
The metavariable $p$ ranges over programs, which consist of declarations and
statements.

A declaration $d$ declares either a classical register, a quantum register, or
a custom gate.
%
$\dcreg{x_c}{n}$ declares a classical register named $x_c$ of length $n$.
%
Similarly, $\dqreg{x_q}{n}$ declares a quantum register named $x_q$ of length
$n$.
%
$\dgate{x_g}{\rep{x_r}}{\rep{x_q}}{\rep{g}}$ defines a custom gate named $x_g$
that takes zero or more real numbers $\rep{x_r}$ and one or more qubits
$\rep{x_q}$.
%
The behavior of a custom gate is specified by a sequence of gate operations
$\rep{g}$, which apply gates to qubits.
%
The scope of a custom gate begins after its definition.
%
Therefore, each custom gate can only refer to built-in or already-defined gates
and cannot be recursive.

A gate operation is either a rotation gate $\irotate{e_1}{e_2}{e_3}{a_q}$, a
CNOT gate $\icnot{a_q}{a_q'}$, or a custom gate
$\icustom{x_g}{\rep{e}}{\rep{a_q}}$.
%
A rotation gate requires three expressions, namely $e_1$, $e_2$, and $e_3$,
which evaluate to real numbers.
%
These expressions can either be a real number $r$, a real-number parameter
$x_r$ (only used in the body of a custom gate), or mathematical operations
between expressions (e.g., $e+e$).
%
Additionally, a rotation gate takes a quantum argument $a_q$.
%
Note that an argument $a$ is either a single bit $x[n]$ or a register $x$;
%
we use $a_q$ for a quantum argument $x_q[n]$ or $x_q$, and $a_c$ for a
classical argument $x_c[n]$ or $x_c$.
%
When a rotation gate operates on a register, it independently applies the
rotation of the same degree to every bit in the register.
%
CNOT and custom gates are similar to rotation gates but can have multiple
quantum arguments.
%
If multiple registers are provided to a gate, their lengths must be equal, and
the gate is independently applied to every index of the registers.
%
For example, $\icnot{x}{y}$ means $\icnot{x[i]}{y[i]}$ for all $i$ where $x$
and $y$ have the same length, and $\icnot{x[0]}{y}$ means $\icnot{x[0]}{y[i]}$
for all $i$.

A statement $S$ is either a quantum operation $Q$ or a conditional quantum
operation $\iif{x_c}{n}{Q}$.
%
Unlike the conditional statement in OpenQASMCore, which checks the value of a
single bit, the conditional statement in OpenQASM compares the value of a
classical register, interpreted as a binary number using the bit at index zero
as the low order bit, with a natural number.

A quantum operation $Q$ is either a gate operation $g$, a measurement
$\imeasure{a_q}{a_c}$, or a reset $\ireset{a_q}$.
%
A measurement operation can measure either a single qubit or a quantum
register.
%
In the latter case, the outcome should be assigned to a classical register of
the same length.
%
A reset operation assigns a new state $\ket{0}$ to a single qubit or each qubit
of a quantum register.

In fact, for the sake of presentation, the syntax defined in
\cref{fig:openqasm-syntax} differs from the syntax of real OpenQASM 2.0 in a
few aspects:
%
\begin{itemize}
	%
	\item We omit \emph{include} statements, which are used to include code from another
	      file in the current code.
	      %
	      They can be replaced with proper code through preprocessing.
	      %
	\item We omit \emph{opaque gate} declarations, which declare gates that can be
	      physically implemented but cannot be described in OpenQASM.
	      %
	      It is impossible to determine the behavior of opaque gates, and they do not
	      accord with our goal of defining executable formal semantics.
	      %
	\item We omit \emph{barrier} statements, which disable optimization across
	      themselves.
	      %
	      They do not affect the semantics and can be simply replaced by $\inop$ during
	      desugaring.
	      %
	\item We require all declarations to precede statements, whereas the real OpenQASM
	      syntax allows the interleaving of declarations and statements.
	      %
	      They can be easily reordered to follow our restriction through preprocessing.
	      %
\end{itemize}

\section{Desugaring}
\label{ch:openqasm:desugaring}

\noindent
The desugaring process from OpenQASM to OpenQASMCore involves five steps:
%
unrolling operations to prevent registers from being used as arguments
(\cref{ch:openqasm:desugaring:unrolling});
%
inlining custom gates to remove their definitions and usages
(\cref{ch:openqasm:desugaring:inlining});
%
decomposing each conditional into nested conditionals to handle comparisons
with natural numbers (\cref{ch:openqasm:desugaring:conditional});
%
representing qubits or classical bits with natural numbers instead of
identifiers to remove register declarations
(\cref{ch:openqasm:desugaring:register});
%
evaluating expressions to real numbers in rotation gate operations
(\cref{ch:openqasm:desugaring:expr}).

\subsection{Unrolling operations}
\label{ch:openqasm:desugaring:unrolling}

\noindent
We unroll each quantum operation that takes a quantum register as an argument
into a sequence of operations, each operating on a single qubit.
%
This process resembles loop unrolling, which transforms a loop with a finite
number of iterations into a sequence of individual statements.
%
For instance, $\icnot{x}{y}$ is desugared as
$\icnot{x[0]}{y[0]};\icnot{x[1]}{y[1]}$, assuming both $x$ and $y$ have lengths
of 2.
%
It is a precondition in OpenQASM that the involved registers must be of equal
length; otherwise, the desugaring fails.
%
This step addresses the disparity between OpenQASM and OpenQASMCore regarding
the acceptance of registers as arguments;
%
while every operation in OpenQASM allows registers, a statement in OpenQASMCore
does not.

\subsection{Inlining custom gates}
\label{ch:openqasm:desugaring:inlining}

\noindent
We inline custom gates to eliminate their definitions and usages because
OpenQASMCore lacks the notion of a custom gate.
%
The inlining process is straightforward as none of the custom gates are
recursive.
%
If the inlining process encounters a reference to a custom gate that is not
defined, it will result in failure.

\subsection{Decomposing conditionals}
\label{ch:openqasm:desugaring:conditional}

\noindent
We desugar conditional statements, which allow comparison with a natural
number, whereas OpenQASMCore only allows comparison with a bit.
%
We decompose each conditional statement into nested conditionals that compare
each bit of the register with the corresponding digit in the binary
representation of the natural number.
%
For instance, $\iif{x}{6}{\cdots}$ is desugared as
$\iif{x[2]}{1}{\iif{x[1]}{1}{\iif{x[0]}{0}{\cdots}}}$, where the length of $x$
is $3$.

\subsection{Representing qubits and classical bits with natural numbers}
\label{ch:openqasm:desugaring:register}

\noindent
We represent qubits and classical bits using natural numbers instead of
identifiers because OpenQASMCore prohibits the declaration of registers and
requires each qubit or classical bit to be represented by a natural number.
%
Conceptually, we merge all quantum registers into a single register and
substitute each qubit access with the corresponding index within the merged
register;
%
this process is repeated for classical registers.

\subsection{Evaluating expressions}
\label{ch:openqasm:desugaring:expr}

\noindent
We substitute every expression, which is used as an argument for a rotation
gate, with the actual real number denoted by the expression by evaluating it.
%
Note that rotation gates in OpenQASMCore accept only real-number constants as
arguments, whereas OpenQASM allows arbitrary expressions.
%
This substitution is possible because all real-number parameters have been
replaced by arguments through inlining, and, at this stage, every expression
contains no variables but solely consists of constants and operations involving
them.

