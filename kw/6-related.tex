% !TEX root = main.tex

\chapter{Related Work}
\label{ch:related}

\paragraph{Formal Semantics for Programming Languages}

Researchers have defined formal semantics for various real-world programming
languages.
%
\Cref{ch:intro} already provides the list of studies of formal
semantics for each language.
%
Therefore, this section emphasizes their methodologies and compares them with
the approach proposed in this work.

While numerous studies have directly defined the formal semantics of a complete
real-world language, this approach necessitates significant manual effort.
%
To circumvent this issue, some studies have chosen to define formal semantics
by defining a small core language and then desugaring the surface language to
the core
language~\cite{guha2010essence,politz2012tested,memarian2016depths,politz2013python}.
%
This approach aligns with our work, where we introduce OpenQASMCore and perform
desugaring from OpenQASM to OpenQASMCore.

Another way to reduce the burden of defining formal semantics is to utilize
semantic frameworks, such as K~\cite{rosu2010overview}, PLT
Redex~\cite{felleisen2009semantics}, and Lem~\cite{mulligan2014lem}.
%
These semantic frameworks are also beneficial because they allow the semantics
to be executable without any additional work.
%
K is a rewrite-based semantic framework, in which the semantics of many
languages have been
formalized~\cite{park2015kjs,ellison2012executable,bogdanas2015kjava,filaretti2014executable,jiao2020semantic}.
%
PLT Redex is a domain-specific language for specifying operational semantics
and has been used to describe the semantics of various
languages~\cite{guha2010essence,politz2013python}.

Researchers have defined formal semantics using proof assistants as well.
%
Coq~\cite{barras1997coq} has been widely employed to mechanize formal
semantics~\cite{bodin2014trusted,blazy2009mechanized,jung2017rustbelt,watt2021two}.
%
Isabelle/HOL~\cite{nipkow2002isabelle} is another popular proof assistant used
for defining formal semantics~\cite{klein2006machine,watt2018mechanising}.
%
Semantics defined with proof assistants are often executable, such as shown by
\citet{bodin2014trusted}, who extract OCaml code from Coq.
%
In this work, we also mechanize the semantics in Coq and extract OCaml code.

One notable approach to obtaining formal semantics with low manual effort is to
automatically extract mechanized semantics from a structured natural language
specification, as proposed and adopted in JavaScript by \citet{park2021jiset}.
%
However, this approach is not applicable to OpenQASM due to its lack of
sufficient structure in the specification.

When formal semantics is executable, it is a standard approach to validate its
conformance using existing test suites.
%
Most traditional programming languages provide official conformance test
suites, such as Test262~\cite{test262} for JavaScript and the Java
Compatibility Kit~\cite{jck} for Java.
%
However, quantum computing is still in its early stages, and OpenQASM 2.0 does
not have an official test suite.
%
Consequently, we assess the conformance of the semantics using example circuits
from the OpenQASM 2.0 specification~\cite{openqasm2} and benchmark circuits
from QASMBench~\cite{li2023qasmbench}.
%
Several studies have demonstrated the practical utility of executable semantics
in identifying bugs in language
implementations~\cite{watt2023wasmref,schumi2021spectest,park2021jest} and
automatically generating test suites of high coverage~\cite{park2023feature}.

\paragraph{Quantum Programming Languages}

Researchers have proposed high-level quantum programming languages, which are
complementary to quantum circuit description languages, such as OpenQASM, the
main focus of this work.
%
While high-level languages allow programmers to express complex quantum
algorithms using abstractions, circuit description languages are tightly
coupled with real quantum hardware and often serve as the compilation targets
for high-level languages.

In the early days of quantum computing, quantum programming languages were
defined by incorporating quantum computations into classical languages.
%
For instance, QCL~\cite{omer2003structured} and qGCL~\cite{sanders2000quantum}
introduce imperative features alongside quantum computations, while quantum
lambda calculus~\cite{maymin1997extending},
QML~\cite{altenkirch2005functional}, and Quipper~\cite{green2013quipper} enable
quantum computations within functional languages.

Recently, researchers have proposed quantum programming languages to address
specific difficulties in quantum programming.
%
Here, we discuss a few notable examples.
%
Silq~\cite{bichsel2020silq} enables programmers to safely and concisely perform
\emph{uncomputation}, which means ignoring temporary qubits during computation
without any side effects.
%
Twist~\cite{yuan2022twist} aids in reasoning about \emph{purity}, i.e., the
absence of entanglement, through its novel type system.
%
Tower~\cite{yuan2022tower} allows the use of random-access memory for
implementing pointer-based data structures.
%
Qunity~\cite{voichick2023qunity} generalizes language constructs in classical
programming, thereby establishing a unified approach to quantum and classical
computing.
%
\citet{Feng2021Hoare} propose a Hoare logic for quantum programs by defining
classical-quantum states, which are capable of describing states of dynamic
circuits.
%
\textsc{voqc}~\cite{Hietala2021voqc} is the first verified optimizer for quantum
circuits, utilizing a language called \textsc{sqir}, compatible with OpenQASM.
%
Embedded in Coq, \textsc{sqir} facilitates the verification of \textsc{voqc}'s
correctness.

A notable exception is the work of \citet{paykin2017qwire}, which introduces
\textsc{Qwire}, a quantum circuit description language that can be manipulated
using a classical language.
%
Like our semantics, \textsc{Qwire}'s semantics is defined in terms of density
matrices.
%
Its type system guarantees the preservation of the trace of the density matrix,
ensuring that any well-typed circuit maintains a trace of 1 during execution.
%
This aligns with the last condition specified by \cref{pos:density}.
%
However, \textsc{Qwire} does not enforce the other properties required by the
density matrix postulate, whereas in this work, we prove the physical
consistency of the proposed semantics.

\paragraph{Testing Quantum Programming Platforms}

As mentioned in \Cref{ch:evaluation:rq2}, \textsc{QDiff}~\cite{wang2022qdiff}
and MorphQ~\cite{paltenghi2023morphq} are differential testing and metamorphic
testing techniques for quantum programming platforms, respectively.
%
Although these techniques are effective in identifying bugs in quantum
programming platforms, all the identified bugs are due to software crashes, but
not due to outcomes not following the correct probability distributions.
%
These approaches do not utilize a testing oracle to compute the correct
probability distributions, thereby limiting their bug detection capabilities.