% !TEX root = main.tex

\chapter{Introduction}
\label{ch:intro}
\noindent
Quantum computing is attracting significant attention from academia and
industry, offering distinct advantages over classical computing by enabling the
development of substantially faster algorithms.
%
A prime example is \citet{shor1994algorithms}'s algorithm, which efficiently
factorizes integers in polynomial time.

When developing quantum programs, \emph{quantum circuit simulators} are
essential.
%
Quantum programs, written in high-level languages, are typically compiled to
quantum circuits described by specific \emph{quantum circuit description
	languages}~\cite{wang2022qdiff, paltenghi2023morphq}.
%
These circuits are executed either on real quantum computers or simulators
running on classical devices.
%
Owing to the high cost of quantum computing resources, developers predominantly
rely on simulators.

Unfortunately, the correctness of existing quantum circuit simulators has not
been explored, primarily due to the lack of formal semantics for quantum
circuit description languages.
%
Since formal semantics are crucial for verifying language implementations,
researchers have defined formal semantics for numerous real-world languages,
such as
%
JavaScript~\cite{maffeis2008operational,guha2010essence,politz2012tested,bodin2014trusted,park2015kjs,park2021jiset},
%
C~\cite{blazy2009mechanized,leroy2009formal,ellison2012executable,krebbers2014formal,memarian2016depths},
%
Java~\cite{farzan2004formal,klein2006machine,bogdanas2015kjava},
%
Standard ML~\cite{kumar2014cakeml,lee2007towards},
%
WebAssembly~\cite{watt2018mechanising,watt2021two},
%
Rust~\cite{jung2017rustbelt},
%
Python~\cite{politz2013python},
%
PHP~\cite{filaretti2014executable},
%
Solidity~\cite{jiao2020semantic},
%
and even the POSIX shell~\cite{greenberg2019executable},
%
often using proof assistants or semantic frameworks like
K~\cite{rosu2010overview}.
%
Particularly, \emph{executable} formal semantics has been beneficial for
testing, serving as a testing oracle to aid in bug
detection~\cite{watt2023wasmref,schumi2021spectest,park2021jest} and test suite
generation~\cite{park2023feature}.
%
However, the semantics of quantum circuit description languages has not been
formalized yet.
%
Notably, OpenQASM~\cite{cross2017open}, a representative language supported by
various major quantum programming platforms~\cite{qiskit, cirq, q-sharp},
defines its syntax using BNF but explains its semantics in prose, lacking
formal rigor.
%
This gap hinders the systematic verification and testing of quantum circuit
simulators.

In this work, we present the first executable formal semantics for OpenQASM
2.0.
%
We chose version 2.0 over the newer OpenQASM 3.0 due to its established
stability and extensive software support.
%
Our formalization employs Coq, from which we extract OCaml code to achieve
executability.
%
The unique characteristics of quantum circuits pose distinct challenges in
developing formal semantics, compared to traditional programming languages.
%
Specifically, we address two challenges:
%
(1) taming \emph{nondeterminism} and (2) ensuring \emph{physical consistency}.

The first challenge stems from the nondeterminism in quantum circuit execution,
a fundamental aspect of quantum mechanics.
%
Quantum circuit simulators mimic this nondeterministic behavior through
\emph{sampling-based} executions.
%
While outcomes may vary between executions, they collectively conform to a
specific probability distribution over multiple runs.
%
Since executable formal semantics should have deterministic results, we adopt
the notion of \emph{exact inference}, i.e., calculating the probability
distribution of possible outcomes, from probabilistic programming.
%
For \emph{static quantum circuits}, featuring end-of-circuit measurements,
exact inference is relatively straightforward, as probability distributions can
be computed by multiplying matrices corresponding to circuits' gates.
%
However, applications like quantum teleportation and iterative phase
estimation~\cite{dobsicek2007arbitrary} involve \emph{dynamic quantum
	circuits}~\cite{dynamic-circuit}.
%
These circuits modify the control flow based on mid-circuit measurements, and
their effects cannot be captured through matrix multiplication.

To enable exact inference for dynamic circuits, we adopt physicist
\citet{everett1957relative}'s \emph{many-worlds interpretation} (MWI), which
posits that each nondeterministic event generates multiple parallel
\emph{worlds} with varying outcomes.
%
Our semantics maintains a list of worlds, starting with one world and splitting
it as execution progresses.
%
A potential concern with MWI is that large circuits may create an excessive
number of worlds, leading to unreasonably long execution times and hindering
the use as a testing oracle.
%
To address this, we introduce two optimizations:
%
\emph{world merging} and \emph{efficient reset}, both aimed at reducing the
number of worlds.

The second challenge relates to the physical laws governing quantum circuits.
%
Unlike traditional programming languages for high-level computations, quantum
circuit description languages specify behavior of circuits running on quantum
hardware, bound to physical principles.
%
Physicists use \emph{density matrices} to model quantum states, and these
matrices and their transformations must fulfill specific constraints, stated
through five postulates~\cite{scherer2019mathematics}.
%
Formal semantics must align with this physical model because deviating from
these constraints would yield probability distributions unrealizable in actual
quantum devices, failing as a testing oracle for simulators.
%
Therefore, verifying the \emph{physical consistency}, i.e., adherence to these
postulates, of our semantics is critical.

To prove the physical consistency of our semantics, we introduce OpenQASMCore,
a core language of OpenQASM.
%
The simplified nature of OpenQASMCore facilitates a streamlined proof of
physical consistency.
%
We define OpenQASMCore's big-step operational semantics and prove its physical
consistency in Coq.
%
We then indirectly define OpenQASM's semantics through desugaring from OpenQASM
to OpenQASMCore.
%
As OpenQASM programs are essentially OpenQASMCore programs with syntactic
sugar, OpenQASM's physical consistency naturally follows.

To demonstrate the effectiveness of our formal executable semantics as a
testing oracle, we tested real-world quantum programming platforms,
Qiskit~\cite{qiskit} and staq~\cite{staq}, specifically a simulator and its
optimization passes in Qiskit and optimizers in staq, and identified five
previously unknown bugs with distinct root causes.
%
For testing, we collected three OpenQASM benchmark suites~\cite{cross2017open,
	li2023qasmbench, tomesh2022supermarq} and generated random circuits, some of
them designed to target Qiskit's optimization passes.
%
Of the five bugs, one was in Qiskit's transpiler, three in Qiskit's
optimization passes, and one in staq's optimizer.
%
All the bugs were confirmed by the developers and two have been already fixed.

Overall, our contributions are as follows:
%
\begin{itemize}
	%
	\item We introduce OpenQASMCore, a core language of OpenQASM
	      (\cref{ch:openqasmcore}).
	      %
	\item We define the executable formal semantics of OpenQASMCore, propose its
	      optimizations, and prove its physical consistency using Coq
	      (\cref{ch:consistency}).
	      %
	\item We indirectly define the semantics of OpenQASM by introducing desugaring from
	      OpenQASM to OpenQASMCore (\cref{ch:openqasm}).
	      %
	\item We show the effectiveness of our optimizations in reducing execution times and
	      validate the utility of our executable semantics as a testing oracle by
	      identifying five previously unknown bugs in real-world quantum programming
	      platforms (\cref{ch:evaluation}).
	      %
\end{itemize}
%
We then discuss related work (\cref{ch:related}) and conclude
(\cref{ch:conclusion}).