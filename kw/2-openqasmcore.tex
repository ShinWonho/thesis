% !TEX root = paper.tex

\chapter{Core Language: OpenQASMCore}
\label{ch:openqasmcore}
\noindent
In this chapter, we define the syntax of OpenQASMCore
(\cref{ch:openqasmcore:syntax}) and provide examples of quantum circuits
written in OpenQASMCore (\cref{ch:openqasmcore:example}).
%
These examples serve to offer a concise introduction to quantum computing and
aid in the intuitive understanding of OpenQASMCore's semantics.

\section{Syntax}
\label{ch:openqasmcore:syntax}

\begin{figure}[t]
	\[
		\begin{array}{@{}l}
			\text{Bit value} \quad b \in \{0,1\} \qquad
			\text{Real number} \quad r \in \mathbb{R} \qquad
			\text{Classical bit} \quad c \in \mathbb{N} \qquad
			\text{Qubit} \quad q \in \mathbb{N} \\
			\text{Statement} \quad s ::= \inop
			\ |\ \iseq{s}{s}
			\ |\ \iif{c}{b}{s}
			\ |\ \irotate{r}{r}{r}{q}
			\ |\ \icnot{q}{q}
			\ |\ \imeasure{q}{c}
			\ |\ \ireset{q}
		\end{array}
	\]
	\caption{Syntax of OpenQASMCore}
	\label{fig:syntax-openqasmcore}
\end{figure}

\noindent
\cref{fig:syntax-openqasmcore} defines the syntax of OpenQASMCore.
%
The metavariable $b$ ranges over (classical) bit values, each of which is
either $0$ or $1$.
%
The metavariable $r$ ranges over real numbers.
%
The metavariable $c$ ranges over natural numbers when used to label classical
bits.
%
Similarly, the metavariable $q$ ranges over natural numbers when used to label
qubits (quantum bits).
%
The numbers of classical bits and qubits used by each program (i.e., circuit)
are finite.
%
In a program with $n$ classical bits and $m$ qubits, classical bits are
numbered from $0$ to $n-1$, and qubits are numbered from $0$ to $m-1$.
%
Since classical bits and qubits are used in different syntactic contexts, it is
always apparent whether a specific natural number corresponds to a classical
bit or a qubit.
%
Finally, the metavariable $s$ ranges over statements, and a program consists of
a single statement.

The language provides seven kinds of statements.
%
The first three kinds are commonly used in classical computing.
%
$\inop$ is a no-op, and $\iseq{s_1}{s_2}$ is a sequencing statement that
sequentially executes $s_1$ and $s_2$.
%
$\iif{c}{b}{s}$ is a conditional statement that executes $s$ if the value of
classical bit $c$ equals $b$.
%
The next four kinds are specific to quantum computing.
%
$\irotate{r_1}{r_2}{r_3}{q}$ is a rotation gate that alters the state of qubit
$q$ by rotating it by $r_1$, $r_2$, and $r_3$ radians around certain axes.
%
$\icnot{q_1}{q_2}$ is a controlled NOT (CNOT) gate that flips qubit $q_2$ if
qubit $q_1$ equals $1$.
%
$\imeasure{q}{c}$ is a measurement statement that measures qubit $q$, yielding a
non-deterministic result, which is then assigned to classical bit $c$.
%
$\ireset{q}$ is a reset statement that prepares qubit $q$ in $\ket{0}$.
%
Further details on these four quantum-specific statements are provided in
\cref{ch:openqasmcore:example}.

\section{Examples}
\label{ch:openqasmcore:example}

\noindent
We now provide example quantum circuits while introducing basic concepts in
quantum programming.
%
These concepts are standard in the physics
literature~\cite{hayashi2014introduction}.
%
First, we explain the execution of a single-qubit circuit, which is simpler
than a multi-qubit circuit.
%
We present the same single-qubit circuit twice: once using state vectors
(\cref{ch:openqasmcore:example:vector}) and once using density matrices
(\cref{ch:openqasmcore:example:matrix}).
%
State vectors offer an intuitive explanation of quantum circuits, while density
matrices are closely related the postulates to which our formal semantics
adheres.
%
Next, we show a multi-qubit circuit to illustrate how the concepts from a
single-qubit circuit can be generalized (\cref{ch:openqasmcore:example:multi}).

\subsection{Single-qubit circuit (with state vector)}
\label{ch:openqasmcore:example:vector}

\noindent
A \emph{qubit} is the base unit of quantum computing.
%
Unlike a classical bit that can take only values 0 or 1, a qubit exists in a
\emph{superposition} state, represented as
$\ket{\psi}=\alpha\ket{0}+\beta\ket{1}$, where $\alpha$ and $\beta$ are complex
numbers, and $\ket{0}$ and $\ket{1}$ represent states 0 and 1, respectively.
%
Measurement of a qubit yields either 0 or 1, with probabilities $|\alpha|^2$
and $|\beta|^2$, respectively.
%
The sum of these probabilities must be equal to 1, satisfying
$|\alpha|^2+|\beta|^2=1$.
%
We refer to $\ket{\psi}$ as the \emph{state vector} of the qubit since it
denotes a column vector as follows:
%
\[
	%
	\ket{0}=\mtrx{1\\0}
	%
	\qquad\qquad
	%
	\ket{1}=\mtrx{0\\1}
	%
	\qquad\qquad
	%
	\ket{\psi}=\alpha\ket{0}+\beta\ket{1}=\mtrx{\alpha\\\beta}
	%
\]

\cref{fig:example:code} presents an example quantum circuit implemented in
OpenQASMCore, while \cref{fig:example:circuit} provides a visual representation
of the circuit.
%
It is a dynamic circuit because it incorporates a measurement of a qubit in the
middle of execution (line 2), changing the control flow based on the
measurement outcome (line 3).
%
The circuit employs one qubit, $q_0$, and two classical bits, $c_0$ and $c_1$.
%
By convention, the initial state of the qubit is $\ket{0}$, and the initial
value of each classical bit is $0$.
%
We now explain the circuit's execution line by line.

\paragraph{Line 1}

We apply the $\ugate{\frac{\pi}{2}}{0}{\pi}$ gate to qubit $q_0$.
%
This gate is known as the \emph{Hadamard gate} and denoted as $H$ in the
diagram.
%
Its purpose is to create a superposition.
%
The effect of any rotation gate ($\mathsf{U}$ gate) is represented by matrix
multiplication with the state vector.
%
The matrix representing the Hadamard gate is
$\frac{1}{\sqrt{2}}\smtrx{1&1\\1&-1}$.
%
(A general formula for computing the matrix representing an arbitrary rotation
gate is discussed in \cref{ch:consistency:semantics}.)
%
Consequently, the state vector changes as follows after the gate:
%
\[
	%
	\frac{1}{\sqrt{2}}\mtrx{1&1\\1&-1}\ket{0}
	%
	=\frac{1}{\sqrt{2}}\mtrx{1&1\\1&-1}\mtrx{1\\0}
	%
	=\frac{1}{\sqrt{2}}\mtrx{1\\1}
	%
	=\frac{1}{\sqrt{2}}\ket{0}+\frac{1}{\sqrt{2}}\ket{1}
	%
\]

\begin{figure}[t]
	\centering
	\begin{subfigure}{0.3\textwidth}
		$
			\begin{array}{@{}r@{~}|l@{}}
				\iseq
				{{\scriptsize\texttt{1}} & \irotate{\frac{\pi}{2}}{0}{\pi}{0}}
				{                                                                          \\ \iseq
				{{\scriptsize\texttt{2}} & \imeasure{0}{0}}
				{                                                                          \\ \iseq
				{{\scriptsize\texttt{3}} & \iif{0}{0}{\irotate{\frac{\pi}{2}}{0}{\pi}{0}}}
				{                                                                          \\
				{\scriptsize\texttt{4}}  & \imeasure{0}{1}
				} } }
			\end{array}
		$
		\caption{OpenQASMCore}
		\label{fig:example:code}
	\end{subfigure}
	\hspace{2em}
	\begin{subfigure}{0.4\textwidth}
		\centering
		\begin{quantikz}[wire types={q,c,c},row sep={0.5cm,between origins}]
			\lstick{$q_0$} & \gate{H} & \meter{} \wire[d][1]{c} & \gate{H} & \meter{} \wire[d][2]{c}&\\
			\lstick{$c_0$} & & & \octrl[vertical wire=c]{-1} & &\\
			\lstick{$c_1$} & & & & &\\
		\end{quantikz}
		\caption{Diagram}
		\label{fig:example:circuit}
	\end{subfigure}
	\caption{An example single-qubit dynamic quantum circuit}
	\label{fig:example}
\end{figure}

\paragraph{Line 2}

We measure qubit $q_0$ and assign the outcome to classical bit $c_0$.
%
According to the many-worlds interpretation, this measurement splits the world
into two branches: one where $0$ is obtained and another where $1$ is obtained.
%
In the first branch, $0$ is assigned to $c_0$, while the superposition of $q_0$
collapses to $\ket{0}$.
%
Similarly, in the second branch, $1$ is assigned to $c_0$, while the
superposition of $q_0$ collapses to $\ket{1}$.
%
Each branch is reachable from the previous world with a probability of $1/2$.
%
We represent a world with a tuple consisting of a state vector, classical bits,
and the corresponding probability of reaching that branch.
%
Therefore, we get $(\smtrx{1\\0},[0,0],\frac{1}{2})$ and
$(\smtrx{0\\1},[1,0],\frac{1}{2})$.

\paragraph{Line 3}

If the value of $c_0$ equals $0$, we apply the Hadamard gate to qubit $q_0$
once again.
%
In the first world, $(\smtrx{1\\0},[0,0],\frac{1}{2})$, $c_0$ holds $0$, and
the gate is applied to $q_0$, resulting in
$(\frac{1}{\sqrt{2}}\smtrx{1\\1},[0,0],\frac{1}{2})$.
%
In the second world, $(\smtrx{0\\1},[1,0],\frac{1}{2})$, $c_0$ contains $1$, so
no changes occur.
%
Consequently, we obtain two worlds:
$(\frac{1}{\sqrt{2}}\smtrx{1\\1},[0,0],\frac{1}{2})$ and
$(\smtrx{0\\1},[1,0],\frac{1}{2})$.

\paragraph{Line 4}

We measure qubit $q_0$ and assign the outcome to classical bit $c_1$.
%
In the first world, the measurement yields $0$ or $1$ with probabilities of
$1/2$.
%
Consequently, the world splits into two: $(\smtrx{1\\0},[0,0],\frac{1}{4})$ and
$(\smtrx{0\\1},[0,1],\frac{1}{4})$.
%
In the second world, the measurement always yields $1$.
%
Therefore, the world does not split and only updates the value of $c_1$,
resulting in $(\smtrx{0\\1},[1,1],\frac{1}{2})$.
%
After the measurement, we have three worlds:
$(\smtrx{1\\0},[0,0],\frac{1}{4})$, $(\smtrx{0\\1},[0,1],\frac{1}{4})$, and
$(\smtrx{0\\1},[1,1],\frac{1}{2})$.
%
From this, we can conclude that the circuit outputs, i.e., the values in the
classical bits, are either $00$, $01$, or $11$, with probabilities of $1/4$,
$1/4$, and $1/2$, respectively.

\subsection{Single-qubit circuit (with density matrix)}
\label{ch:openqasmcore:example:matrix}

\noindent
The \emph{density matrix}, which is the linear combination of outer products of
a state vector with itself, is used to represent the state of a qubit as a
matrix instead of a vector. For example, consider a density matrix with a
single outer product:
%
\[
	%
	\rho
	%
	=\mtrx{\alpha\\\beta}\mtrx{\alpha\\\beta}^*
	%
	=\mtrx{\alpha\\\beta}\mtrx{\alpha^*&\beta^*}
	%
	=\mtrx{\alpha\alpha^*&\alpha\beta^*\\\beta\alpha^*&\beta\beta^*}
	%
\]
%
Note that $\alpha^*$ is the complex conjugate of a complex number $\alpha$, and
$M^*$ is the conjugate transpose of a matrix $M$.
%
The diagonal elements of a density matrix correspond to the probabilities of
each measurement outcome because $\alpha\alpha^*$ equals $|\alpha|^2$ for every
complex number $\alpha$.
%
We now explain the execution of \cref{fig:example}'s circuit using density
matrices.
%
The initial density matrix is $\smtrx{1\\0}\smtrx{1&0}=\smtrx{1&0\\0&0}$.

\paragraph{Line 1}

We apply the Hadamard gate to $q_0$.
%
The effect of a rotation gate is represented by matrix multiplication with the
density matrix.
%
Specifically, the updated density matrix is given by $M\rho M^*$, where $M$ is
the matrix representing the gate and $\rho$ is the current density matrix.
%
Since the Hadamard gate is represented as
$\frac{1}{\sqrt{2}}\smtrx{1&1\\1&-1}$, the density matrix is updated as
follows:
%
\[
	%
	\frac{1}{\sqrt{2}}\mtrx{1&1\\1&-1}\cdot\mtrx{1&0\\0&0}\cdot\frac{1}{\sqrt{2}}\mtrx{1&1\\1&-1}
	%
	=\frac{1}{2}\mtrx{1&1\\1&1}
	%
\]
%
This result matches the result obtained using the state vector:
%
\[
	\frac{1}{2}\mtrx{1&1\\1&1}=\frac{1}{\sqrt{2}}\mtrx{1\\1}\cdot\frac{1}{\sqrt{2}}\mtrx{1&1}.
\]

\paragraph{Line 2}

We measure $q_0$ and assign the outcome to $c_0$.
%
For each possible measurement outcome, there exists a corresponding matrix,
which is used to compute the probability of the outcome and the density matrix
after the measurement.
%
The matrices for outcomes $0$ and $1$ are $\smtrx{1&0\\0&0}$ and
$\smtrx{0&0\\0&1}$, respectively.
%
The probability of the outcome is $P=\trace{\rho M}$, and the density matrix
after the measurement is $\frac{1}{P}M\rho M$, where $M$ is the matrix
corresponding to an outcome and $\rho$ is the current density matrix.
%
Note that $\trace{M}$ is the \emph{trace} of $M$, i.e., the sum of $M$'s
diagonal elements.
%
We compute the probability of $0$ being the outcome and the updated density
matrix as follows:
%
\[
	%
	\trace{\frac{1}{2}\mtrx{1&1\\1&1}\mtrx{1&0\\0&0}}
	%
	=\trace{\frac{1}{2}\mtrx{1&0\\1&0}}
	%
	=\frac{1}{2}
	%
	\qquad
	%
	\frac{1}{\sfrac{1}{2}}\cdot\mtrx{1&0\\0&0}\cdot\frac{1}{2}\mtrx{1&1\\1&1}\cdot\mtrx{1&0\\0&0}
	%
	=\mtrx{1&0\\0&0}
	%
\]
%
Similarly, we compute the probability of $1$ being the outcome and the updated
density matrix:
%
\[
	%
	\trace{\frac{1}{2}\mtrx{1&1\\1&1}\mtrx{0&0\\0&1}}
	%
	=\trace{\frac{1}{2}\mtrx{0&1\\0&1}}
	%
	=\frac{1}{2}
	%
	\qquad
	%
	\frac{1}{\sfrac{1}{2}}\cdot\mtrx{0&0\\0&1}\cdot\frac{1}{2}\mtrx{1&1\\1&1}\cdot\mtrx{0&0\\0&1}
	%
	=\mtrx{0&0\\0&1}
	%
\]
%
As a result, we obtain two worlds: $(\smtrx{1&0\\0&0},[0,0],\frac{1}{2})$ and
$(\smtrx{0&0\\0&1},[1,0],\frac{1}{2})$.
%
This result is also consistent with the result obtained using the state vector:
%
$\smtrx{1&0\\0&0}=\smtrx{1\\0}\smtrx{1&0}$ and
$\smtrx{0&0\\0&1}=\smtrx{0\\1}\smtrx{0&1}$.

\paragraph{Lines 3--4}

Similar to the process using state vectors, we end up with three worlds in the
end: $(\smtrx{1&0\\0&0},[0,0],\frac{1}{4})$,
$(\smtrx{0&0\\0&1},[0,1],\frac{1}{4})$, and
$(\smtrx{0&0\\0&1},[1,1],\frac{1}{2})$.

\subsection{Multi-qubit circuit}
\label{ch:openqasmcore:example:multi}

\noindent
Both state vectors and density matrices easily extend to multi-qubit systems.
%
The state of an $n$-qubit system is represented by a state vector of $2^n$
entries or a density matrix of dimension $2^n \times 2^n$.
%
For example, in a $2$-qubit system, a measurement can produce the outcomes
$00$, $01$, $10$, or $11$.
%
A system that produces these outcomes with corresponding probabilities
$|\alpha|^2$, $|\beta|^2$, $|\gamma|^2$, and $|\delta|^2$ is represented by the
following state vector and density matrix:
%
\[
	%
	\ket{\psi}
	%
	=\alpha\ket{00}+\beta\ket{01}+\gamma\ket{10}+\delta\ket{11}
	%
	=\alpha\mtrx{1\\0\\0\\0}+\beta\mtrx{0\\1\\0\\0}+\gamma\mtrx{0\\0\\1\\0}+\delta\mtrx{0\\0\\0\\1}
	%
	=\mtrx{\alpha\\\beta\\\gamma\\\delta}
	%
\]
\[
	%
	\rho
	%
	=\mtrx{\alpha\\\beta\\\gamma\\\delta}\mtrx{\alpha\\\beta\\\gamma\\\delta}^*
	%
	=\mtrx{\alpha\\\beta\\\gamma\\\delta}\mtrx{\alpha^*&\beta^*&\gamma^*&\delta^*}
	%
	=\mtrx{
		%
		\alpha\alpha^*&\alpha\beta^*&\alpha\gamma^*&\alpha\delta^*\\
		%
		\beta\alpha^*&\beta\beta^*&\beta\gamma^*&\beta\delta^*\\
		%
		\gamma\alpha^*&\gamma\beta^*&\gamma\gamma^*&\gamma\delta^*\\
		%
		\delta\alpha^*&\delta\beta^*&\delta\gamma^*&\delta\delta^*
		%
	}
	%
\]

\begin{figure}[t]
	\centering
	\begin{subfigure}{0.3\textwidth}
		$
			\begin{array}{@{}r@{~}|l@{}}
				\iseq
				{{\scriptsize\texttt{1}}                          & \irotate{\frac{\pi}{2}}{0}{\pi}{0}}
				{                                                                                       \\
				\iseq
				{{\scriptsize\texttt{2}}                          & \icnot{0}{1}}
				{                                                                                       \\
				\iseq
				{{\scriptsize\texttt{3}}                          & \imeasure{0}{0}}
				{ \\{\scriptsize\texttt{4}} & \imeasure{1}{1}}
				}
				}
			\end{array}
		$
		\caption{OpenQASMCore}
		\label{fig:example-multi:code}
	\end{subfigure}
	\hspace{2em}
	\begin{subfigure}{0.4\textwidth}
		\centering
		\begin{quantikz}[wire types={q,q,c,c},row sep={0.4cm,between origins}]
			\lstick{$q_0$} & \gate{H} & \ctrl{1} & \meter{} \wire[d][2]{c} & & \\
			\lstick{$q_1$} & & \targ{} & & \meter{} \wire[d][2]{c} & \\
			\lstick{$c_0$} & & & & & \\
			\lstick{$c_1$} & & & & &
		\end{quantikz}
		\caption{Diagram}
		\label{fig:example-multi:circuit}
	\end{subfigure}
	\caption{An example multi-qubit quantum circuit}
	\label{fig:example-multi}
\end{figure}

\cref{fig:example-multi} illustrates an example multi-qubit quantum circuit.
%
The circuit utilizes two qubits, $q_0$ and $q_1$, as well as two classical
bits, $c_0$ and $c_1$.
%
Each qubit is initially set to $\ket{0}$, resulting in the following initial
state vector and density matrix of the whole system:
%
\[
	%
	\ket{00}=
	%
	\mtrx{1\\0\\0\\0}
	%
	\qquad\qquad\qquad\qquad
	%
	\mtrx{1\\0\\0\\0}\mtrx{1\\0\\0\\0}^*
	%
	=\mtrx{1&0&0&0\\0&0&0&0\\0&0&0&0\\0&0&0&0}
	%
\]
%
We now explain the circuit's execution line by line.

\paragraph{Line 1}

We apply the Hadamard gate to $q_0$.
%
Since the density matrix is of dimension $4\times4$, the matrix representing
the gate also needs to be of dimension $4\times4$.
%
This matrix can be obtained by taking the tensor product of the original
$2\times2$ matrix representing the Hadamard gate and an identity matrix.
%
The matrix representing the application of the Hadamard gate to $q_0$ is as
follows:
%
\[
	%
	\frac{1}{\sqrt{2}}\mtrx{1&1\\1&-1}\otimes\mtrx{1&0\\0&1}
	%
	=\frac{1}{\sqrt{2}}\mtrx{1&0&1&0\\0&1&0&1\\1&0&-1&0\\0&1&0&-1}
	%
\]
%
A general formula for computing a $2^n\times2^n$ matrix from a $2\times2$
matrix is discussed in \cref{ch:consistency:semantics}.
%
Using the above matrix, we compute the density matrix after the gate as
follows:
%
\[
	%
	\frac{1}{\sqrt{2}}\mtrx{1&0&1&0\\0&1&0&1\\1&0&-1&0\\0&1&0&-1}
	%
	\cdot
	%
	\mtrx{1&0&0&0\\0&0&0&0\\0&0&0&0\\0&0&0&0}
	%
	\cdot
	%
	\frac{1}{\sqrt{2}}\mtrx{1&0&1&0\\0&1&0&1\\1&0&-1&0\\0&1&0&-1}
	%
	= \frac{1}{2}\mtrx{1&0&1&0\\0&0&0&0\\1&0&1&0\\0&0&0&0}
	%
\]
%
The diagonal elements imply that the measurement outcome of the system is
either $00$ or $10$ with probabilities $1/2$.
%
In other words, the measurement outcome of $q_0$ has a probability of $1/2$ of
being $0$ or $1$, while the measurement outcome of $q_1$ is always $0$.
%
This aligns with the intuition that the Hadamard gate applied to $q_0$ should
create a superposition in $q_0$ only, leaving $q_1$ unaffected.

\paragraph{Line 2}

We apply the CNOT gate to $q_0$ and $q_1$, with $q_0$ serving as the
\emph{control} and $q_1$ as the \emph{target}.
%
Intuitively, it flips the target if the control is $1$.
%
The purpose of this gate is to \emph{entangle} the qubits, establishing a
quantum mechanical dependency between them.
%
The CNOT gate operates on two distinct qubits and is represented by a
$4\times4$ matrix: $\Smtrx{1&0&0&0\\0&1&0&0\\0&0&0&1\\0&0&1&0}$.
%
We generalize this matrix for a system with more than two qubits in
\cref{ch:consistency:semantics}.
%
We compute the density matrix after the gate as follows:
%
\[
	%
	\mtrx{1&0&0&0\\0&1&0&0\\0&0&0&1\\0&0&1&0}
	%
	\cdot
	%
	\frac{1}{2}\mtrx{1&0&1&0\\0&0&0&0\\1&0&1&0\\0&0&0&0}
	%
	\cdot
	%
	\mtrx{1&0&0&0\\0&1&0&0\\0&0&0&1\\0&0&1&0}
	%
	=\frac{1}{2}\mtrx{1&0&0&1\\0&0&0&0\\0&0&0&0\\1&0&0&1}
	%
\]
%
The diagonal elements indicate that the measurement outcome of the system can
be either $00$ or $11$ with probabilities of $1/2$.
%
This implies that measuring one qubit allows determining the measurement
outcome of the other qubit, demonstrating the entanglement between $q_0$ and
$q_1$.

\paragraph{Line 3}

We measure $q_0$ and assign the outcome to $c_0$.
%
The matrices corresponding to each measurement outcome are generalized using
tensor product, similar to the matrices representing rotation gates.
%
The matrices corresponding to outcomes $0$ and $1$ from the measurement of
$q_0$ are as follows:
%
\[
	%
	\mtrx{1&0\\0&0}\otimes\mtrx{1&0\\0&1}
	%
	=\mtrx{1&0&0&0\\0&1&0&0\\0&0&0&0\\0&0&0&0}
	%
	\qquad\qquad
	%
	\mtrx{0&0\\0&1}\otimes\mtrx{1&0\\0&1}
	%
	=\mtrx{0&0&0&0\\0&0&0&0\\0&0&1&0\\0&0&0&1}
	%
\]
%
We compute the probability of $0$ being the outcome and the updated density
matrix as follows:
%
\[
	%
	\btrace{\frac{1}{2}\mtrx{1&0&0&1\\0&0&0&0\\0&0&0&0\\1&0&0&1}\mtrx{1&0&0&0\\0&1&0&0\\0&0&0&0\\0&0&0&0}}
	%
	=\btrace{\frac{1}{2}\mtrx{1&0&0&0\\0&0&0&0\\0&0&0&0\\1&0&0&0}}
	%
	=\frac{1}{2}
	%
\]
\[
	%
	\frac{1}{\sfrac{1}{2}}
	%
	\cdot\mtrx{1&0&0&0\\0&1&0&0\\0&0&0&0\\0&0&0&0}
	%
	\cdot\frac{1}{2}\mtrx{1&0&0&1\\0&0&0&0\\0&0&0&0\\1&0&0&1}
	%
	\cdot\mtrx{1&0&0&0\\0&1&0&0\\0&0&0&0\\0&0&0&0}
	%
	=\mtrx{1&0&0&0\\0&0&0&0\\0&0&0&0\\0&0&0&0}
	%
\]
%
We can compute the probability of $1$ being the outcome and the updated density
matrix in a similar fashion.
%
Consequently, we obtain two worlds:
$(\Smtrx{1&0&0&0\\0&0&0&0\\0&0&0&0\\0&0&0&0},[0,0],\frac{1}{2})$ and
$(\Smtrx{0&0&0&0\\0&0&0&0\\0&0&0&0\\0&0&0&1},[1,0],\frac{1}{2})$.

\paragraph{Line 4}

We measure $q_1$ and assign the outcome to $c_1$.
%
The matrices corresponding to the outcomes $0$ and $1$ from the measurement of
$q_1$ are as follows:
%
\[
	%
	\mtrx{1&0\\0&1}\otimes\mtrx{1&0\\0&0}
	%
	=\mtrx{1&0&0&0\\0&0&0&0\\0&0&1&0\\0&0&0&0}
	%
	\qquad\qquad
	%
	\mtrx{1&0\\0&1}\otimes\mtrx{0&0\\0&1}
	%
	=\mtrx{0&0&0&0\\0&1&0&0\\0&0&0&0\\0&0&0&1}
	%
\]
%
Using these matrices, we can compute that the probability of the outcome being
$0$ is $1$ in the first world, and the probability of the outcome being $1$ is
$1$ in the second world.
%
Therefore, the execution terminates with two worlds:
%
$(\Smtrx{1&0&0&0\\0&0&0&0\\0&0&0&0\\0&0&0&0},[0,0],\frac{1}{2})$ and
$(\Smtrx{0&0&0&0\\0&0&0&0\\0&0&0&0\\0&0&0&1},[1,1],\frac{1}{2})$.