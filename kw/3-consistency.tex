% !TEX root = main.tex
\chapter{Executable Formal Semantics and Its Physical Consistency}
\label{ch:consistency}

\noindent
In this section, we define the executable formal semantics of OpenQASMCore
(\cref{ch:consistency:semantics}) and introduce two optimization techniques
(\cref{ch:consistency:optimization}).
%
We then prove the physical consistency of the semantics
(\cref{ch:consistency:consistency}).
%
The semantics and the proof are mechanized in Coq and publicly
available~\cite{artifact}.

\section{Semantics}
\label{ch:consistency:semantics}

\begin{figure}[t]
	\[
		\begin{array}{@{}rr@{~}c@{~}l@{\qquad\qquad}r@{~}c@{~}l@{}}
			\text{Density matrix}  & \rho   & \in                                                    & \bigcup_{n\in\mathbb{N}}\mathcal{M}_{2^n,2^n}(\mathbb{C}) &
			\sunitary{\rho}{M}     & =      & M\rho M^*                                                                                                            \\
			\text{Classical state} & \sigma & \in                                                    & \bigcup_{n\in\mathbb{N}}\mathbb{Z}_n\rightarrow \{0,1\}   &
			\sprob{\rho}{M}        & =      & \trace{\rho M}                                                                                                       \\
			\text{Probability}     & P      & \in                                                    & \mathbb{R}_{[0,1]}                                        &
			\smeasure{\rho}{M}     & =      & \frac{1}{\sprob{\rho}{M}}M\rho M                                                                                     \\
			\text{World}           & w      & ::=                                                    & (\rho,\sigma,P)                                           &
			\qubits{\rho}          & =      & n\ \text{if}\ \rho\in\mathcal{M}_{2^n,2^n}(\mathbb{C})
		\end{array}
	\]
	\caption{Semantic elements and functions}
	\label{fig:semantic-elements}
\end{figure}

\noindent
\cref{fig:semantic-elements} shows the definitions related to the semantics of
OpenQASMCore.
%
The metavariable $\rho$ ranges over density matrices, which are complex
matrices of dimensions $2^n\times2^n$, where $n$ is a natural number.
%
The metavariable $\sigma$ ranges over classical states, which are mappings from
classical bits to their values.
%
The metavariable $P$ ranges over probabilities of a world, which are real
numbers inclusively between $0$ and $1$.
%
The metavariable $w$ ranges over worlds, which are triples consisting of a
density matrix, a classical state, and a probability.
%
The functions $\embox{gate}$, $\embox{prob}$, and $\embox{msr}$ compute the
updated density matrix after a gate, the probability of a measurement outcome,
and the updated density matrix after a measurement, respectively.
%
Each function takes a density matrix and a matrix representing a gate/outcome
as inputs, and the result is defined only when the two matrices are of the same
dimension.
%
In addition, $\smeasure{\rho}{M}$ is also undefined when $\sprob{\rho}{M}$ is
$0$.
%
The function $\embox{qbts}$ returns the number of qubits in the system
represented by a given density matrix.

\begin{figure}[t]
	\[
		\small
		\begin{array}{@{}c@{}}
			\fbox{$\eval{\rep{w}}{s}{\rep{w}}$}
			\qquad\qquad
			\inferrule
			{ \rep{\eval{w}{s}{\rep{w'}}} }
			{ \eval{\rep{w}}{s}{\rep{\rep{w'}}} }
			\quad[\textsc{Eval-Many}]
			\qquad\qquad
			\eval{w}{\inop}{w}
			\quad[\textsc{Eval-Nop}]
			\\[1.5em]
			\inferrule
			{ \eval{w}{s_1}{\rep{w'}}                                                                 \\ \eval{\rep{w'}}{s_2}{\rep{w''}} }
			{ \eval{w}{\iseq{s_1}{s_2}}{\rep{w''}} }
			\quad[\textsc{Eval-Seq}]
			\qquad
			\inferrule
			{
			w=(\rho,\sigma,P)                                                                         \\
				\sigma(c)\not=b
			}
			{ \eval{w}{\iif{c}{b}{s}}{w} }
			\quad[\textsc{Eval-If-False}]
			\\[1.5em]
			\inferrule
			{
			w=(\rho,\sigma,P)                                                                         \\
			\sigma(c)=b                                                                               \\
				\eval{w}{s}{\rep{w'}}
			}
			{ \eval{w}{\iif{c}{b}{s}}{\rep{w'}} }
			\quad[\textsc{Eval-If-True}]
			\\[1.5em]
			\inferrule
			{ \qubits{\rho}=n }
			{
				\eval{(\rho,\sigma,P)}{\irotate{r_1}{r_2}{r_3}{q}}{(\sunitary{\rho}{\mrotate{n,q,r_1,r_2,r_3}},\sigma,P)}
			}
			\quad[\textsc{Eval-Rotate}]
			\\[1.5em]
			\inferrule
			{ \qubits{\rho}=n }
			{
				\eval{(\rho,\sigma,P)}{\icnot{q_1}{q_2}}{(\sunitary{\rho}{\mcnot{n,q_1,q_2}},\sigma,P)}
			}
			\quad[\textsc{Eval-Cnot}]
			\\[1.5em]
			\inferrule
			{
			\qubits{\rho}=n                                                                           \\
			\sprob{\rho}{\mproj{n,q,0}} \neq 0                                                        \\
			\sprob{\rho}{\mproj{n,q,1}} \neq 0                                                        \\\\
			w_0=(\smeasure{\rho}{\mproj{n,q,0}},\sigma[c\mapsto0],P\times\sprob{\rho}{\mproj{n,q,0}}) \\
				w_1=(\smeasure{\rho}{\mproj{n,q,1}},\sigma[c\mapsto1],P\times\sprob{\rho}{\mproj{n,q,1}})
			}
			{ \eval{(\rho,\sigma,P)}{\imeasure{q}{c}}{w_0,w_1} }
			\quad[\textsc{Eval-Measure}]
			\\[1.5em]
			\inferrule
			{
			\qubits{\rho}=n                                                                           \\
			\sprob{\rho}{\mproj{n,q,1}}=0                                                             \\\\
				w_0=(\smeasure{\rho}{\mproj{n,q,0}},\sigma[c\mapsto0],P\times\sprob{\rho}{\mproj{n,q,0}})
			}
			{ \eval{(\rho,\sigma,P)}{\imeasure{q}{c}}{w_0} }
			\quad[\textsc{Eval-Measure-0}]
			\\[1.5em]
			\inferrule
			{
			\qubits{\rho}=n                                                                           \\
			\sprob{\rho}{\mproj{n,q,0}}=0                                                             \\\\
				w_1=(\smeasure{\rho}{\mproj{n,q,1}},\sigma[c\mapsto1],P\times\sprob{\rho}{\mproj{n,q,1}})
			}
			{ \eval{(\rho,\sigma,P)}{\imeasure{q}{c}}{w_1} }
			\quad[\textsc{Eval-Measure-1}]
			\\[1.5em]
			\inferrule
			{
			w=(\rho,\sigma,P)                                                                         \\
			c_{\embox{\tiny reset}} \notin \dom{\sigma}                                               \\\\
				\eval {w} {
					\iseq{
						\imeasure{q}{c_{\embox{\tiny reset}}}
					}{
						\iif{c_{\embox{\tiny reset}}}{1}{\irotate{\pi}{0}{\pi}{q}}
					}} {\rep{w'}}
			} { \eval {w} {\ireset{q}} {\rep{w'}}}
			\quad[\textsc{Eval-Reset}]
		\end{array}
	\]
	\caption{Semantics of OpenQASMCore}
	\label{fig:semantics}
\end{figure}

\cref{fig:semantics} defines the big-step operational semantics of OpenQASMCore
in the form of \fbox{$\eval{\rep{w}}{s}{\rep{w}}$}.
%
Following the notation used by \citet{vytiniotis2013halo} and
\citet{coughlin2014fissile}, an overline above material represents zero or more
repetitions of the material, separated by commas, with a subscript added to
each metavariable in the material.
%
Thus, $\eval{\rep{w}}{s}{\rep{w'}}$ denotes
$\eval{w_1,w_2,\ldots,w_n}{s}{w_1',w_2',\ldots,w_m'}$, indicating that the
worlds $w_1$, $w_2$, \ldots, $w_n$ transform into $w_1'$, $w_2'$, \ldots,
$w_m'$ when $s$ is executed.

Most rules handle the case where a single world is provided as input, and they
are generalized to accommodate multiple worlds through rule
	[\textsc{Eval-Many}].
%
This rule states that when multiple worlds are given, the statement runs in
each world, and the output is determined by collecting the outputs from each
individual run.
%
Note that when overlines are nested, the outermost one expands first, while the
copies of each inner overline expand independently and may have different
lengths.

The semantics of classical statements are conventional, as defined by rules
	[\textsc{Eval-Nop}], [\textsc{Eval-Seq}], [\textsc{Eval-If-False}], and
	[\textsc{Eval-If-True}].
%
$\inop$ does not alter the world state.
%
$\iseq{s_1}{s_2}$ executes $s_1$ first in the given world and subsequently
executes $s_2$ in every world generated by $s_1$.
%
$\iif{c}{b}{s}$ executes $s$ if the value of $c$ equals $b$.

Rule [\textsc{Eval-Rotate}] defines the semantics of a rotation gate, which
updates only the density matrix.
%
The definition of the matrix $\mrotate{n,q,r_1,r_2,r_3}$ can be found in
\cref{fig:gen22}, which illustrates the generalization of $2\times2$ matrices
representing gates/measurements operating on a single qubit.
%
The function $\embox{gen}_{n,q}$ returns a $2^n\times2^n$ matrix that affects
the $q$-th qubit (indexed from zero) in an $n$-qubit system by generalizing a
given $2\times2$ matrix.
%
Note that $\identity{m}$ represents the identity matrix of dimension
$2^m\times2^m$.
%
The effect of the $\ugate{r_1}{r_2}{r_3}$ gate on a single qubit is denoted as
$\mrotate{r_1,r_2,r_3}$ and is generalized by $\embox{gen}_{n,q}$ to form
$\mrotate{n,q,r_1,r_2,r_3}$.

Rule [\textsc{Eval-Cnot}] defines the semantics of the CNOT gate, which also
updates only the density matrix.
%
\cref{fig:cnot} defines the matrix $\mcnot{n,q_1,q_2}$, which flips the
$q_2$-th qubit using the $q_1$-th qubit as control in an $n$-qubit system, by
generalizing the $4\times4$ matrix $\mcnot{}$.
%
Unlike rotation matrices, $\mcnot{}$ is not simply generalized by tensor
products with identity matrices due to its ability to operate on two qubits,
which may be non-adjacent.
%
To address this, we introduce the matrix $\mswap{n,q_1,q_2}$, which swaps the
$q_1$-th and $q_2$-th qubits in an $n$-qubit system, and employ this matrix to
define $\mcnot{n,q_1,q_2}$.
%
Since $\mswap{n,q_1,q_2}$ is meaningful only when $q_1$ differs from $q_2$, it
is an identity matrix when $q_1$ equals $q_2$.
%
For $q_1\not=q_2$, $\mswap{n,q_1,q_2}$ is recursively defined, beginning with
the base case $\mswap{}$, which swaps the qubits in a $2$-qubit system.
%
We then define $\mswap{n,0,n-1}$, which swaps the $0$-th and $(n-1)$-th qubits
by swapping the $0$-th and $1$-st, then the $1$-st and $(n-1)$-th, and finally
$0$-th and $1$-st.
%
Subsequently, we define $\mswap{n,q_1,q_2}$ using tensor products.
%
Finally, $\mcnot{n,q_1,q_2}$ is defined by swapping the $0$-th and $q_1$-th
qubits, then the $1$-st and $q_2$-th qubits, flipping the $1$-st qubit using
the $0$-th qubit as control, and finally swapping back the $0$-th and $q_1$-th
qubits as well as the $1$-st and $q_2$-th qubits.

Rules [\textsc{Eval-Measure}], [\textsc{Eval-Measure-0}], and
	[\textsc{Eval-Measure-1}] define the semantics of a measurement statement.
%
Rule [\textsc{Eval-Measure}] handles the case where both $0$ and $1$ are
possible outcomes by splitting the world into two, while rules
	[\textsc{Eval-Measure-0}] and [\textsc{Eval-Measure-1}] handle the cases where
only one of them is a possible outcome.
%
The matrices $\mproj{n,q,0}$ and $\mproj{n,q,1}$ are defined by generalizing
$\mproj{0}$ and $\mproj{1}$ using tensor products, as shown in
\cref{fig:gen22}.

Rule [\textsc{Eval-Reset}] defines the semantics of a reset statement, which
collapses a qubit's superposition and sets its state to $\ket{0}$.
%
To achieve this, we introduce a fresh classical bit that is not used by the
circuit and assign the measurement result to the bit.
%
If the measurement outcome is $0$, the qubit is already in the state $\ket{0}$.
%
If the outcome is $1$, the qubit is in the state $\ket{1}$, and we flip it by
applying the $\ugate{\pi}{0}{\pi}$ gate.
%
Due to the measurement, the world splits into two unless the qubit is already
in the states $\ket{0}$ or $\ket{1}$ before the measurement.

When executing a statement as a single program, the input is a single world
where every qubit is $\ket{0}$ and every classical bit is $0$. The probability
of the world is $1$.

\begin{figure}[t]
	\[
		\begin{array}{@{}r@{~}c@{~}l@{\quad}r@{~}c@{~}l@{}}
			\multicolumn{6}{c}{
				\generalize{n,q}{M}=\identity{q}\otimes{M}\otimes\identity{n-q-1}
			}                                                                          \\[1em]
			\mrotate{r_1,r_2,r_3}        & = &
			\mtrx{
			e^{-i(r_2+r_3)/2}\cos(r_1/2) &
			-e^{-i(r_2-r_3)/2}\sin(r_1/2)                                              \\
			e^{i(r_2-r_3)/2}\sin(r_1/2)  &
			e^{i(r_2+r_3)/2}\cos(r_1/2)
			}
			                             &
			\mrotate{n,q,r_1,r_2,r_3}    & = & \generalize{n,q}{\mrotate{r_1,r_2,r_3}}
			\\[1em]
			\multicolumn{6}{c}{
			\mproj{0}=\smtrx{1           & 0                                           \\0                  & 0} \qquad
				  \mproj{n,q,0}=\generalize{n,q}{\mproj{0}} \qquad
			\mproj{1}=\smtrx{0           & 0                                           \\0 & 1} \qquad
				  \mproj{n,q,1}=\generalize{n,q}{\mproj{1}}
			}
		\end{array}
	\]
	\caption{Generalization of $2\times2$ matrices}
	\label{fig:gen22}
\end{figure}

\begin{figure}[t]
	\[
		\begin{array}{@{}r@{~}c@{~}l@{\quad}r@{~}c@{~}l@{}}
			\mswap{}          & = & \Smtrx{1                                                                                    & 0 & 0 & 0 \\0&0&1&0\\0&1&0&0\\0                 & 0 & 0                         & 1}
			                  &
			\mswap{n,q_1,q_2} & = & \begin{cases}
				                        I_n\qquad\qquad\qquad\qquad\qquad\text{if}\ q_1=q_2                      \\
				                        \mswap{}\ \ \qquad\qquad\qquad\qquad\text{if}\ n=2\land q_1=0\land q_2=1 \\
				                        (\mswap{}\otimes\identity{n-2})
				                        (\identity{1}\otimes\mswap{n-1,0,n-2})
				                        (\mswap{}\otimes\identity{n-2})                                          \\
				                        \quad\qquad\qquad\qquad\qquad
				                        \text{if}\ n>2\land q_1=0\land q_2=n-1                                   \\
				                        \identity{q_1}\otimes\mswap{q_2-q_1+1,0,q_2-q_1}\otimes\identity{n-q_2-1}
				                        \ \qquad\quad\text{otherwise}
			                        \end{cases}
			\\[2em]
			\mcnot{}          & = & \Smtrx{1                                                                                    & 0 & 0 & 0 \\0&1&0&0\\0&0&0&1\\0                 & 0 & 1                                                                                           & 0}
			                  &
			\mcnot{n,q_1,q_2} & = & \mswap{n,0,q_1}\mswap{n,1,q_2}(\mcnot{}\otimes\identity{n-2})\mswap{n,1,q_2}\mswap{n,0,q_1}
		\end{array}
	\]
	{\footnotesize$(0\le q_1<n,0\le q_2<n)$}
	\caption{Generalization of CNOT}
	\label{fig:cnot}
\end{figure}

\section{Optimizations}
\label{ch:consistency:optimization}

\noindent
We propose optimization techniques for our executable formal semantics.
%
Since we split worlds at each measurement following MWI, the number of worlds
exponentially increases as the number of measurements grows.
%
This prevents the execution of large quantum circuits in a reasonable time.
%
To address this issue, we introduce optimizations that reduce the number of
worlds.

The key idea behind these optimizations is the notion of a \emph{mixed state},
which represents different quantum states and their probabilities using a
single density matrix.
%
For example, consider the density matrix $\rho=\tfrac{1}{2}\smtrx{1&1\\1&1}$
for a superposed state of
$\tfrac{1}{\sqrt{2}}\ket{0}+\tfrac{1}{\sqrt{2}}\ket{1}$.
%
When measuring the qubit, it yields either $0$ or $1$, each with a probability
of $\frac{1}{2}$.
%
The corresponding density matrices are $\rho_0=\smtrx{1&0\\0&0}$ and
$\rho_1=\smtrx{0&0\\0&1}$.
%
Instead of maintaining two matrices with their respective probabilities, a
single matrix can represent both possibilities through a weighted sum:
%
$\rho_{\embox{\tiny mixed}} = \tfrac{1}{2}\rho_0+\tfrac{1}{2}\rho_1 =
	\tfrac{1}{2}\smtrx{1&0\\0&1}$.

Unfortunately, we cannot always use mixed states to avoid splitting worlds
because we need to consider not only qubits and probabilities but also
classical bits.
%
Without using mixed states, we have two post-measurement worlds,
$(\rho_0,[0],\frac{1}{2})$ and $(\rho_1,[1],\frac{1}{2})$.
%
We cannot simply replace them with a single world containing
$\rho_{\embox{\tiny mixed}}$ because we cannot retain the classical bit
information.
%
Neither $(\rho_{\embox{\tiny mixed}},[0],1)$ nor $(\rho_{\embox{\tiny
			mixed}},[1],1)$ is correct.
%
To have only a single world, we would need to expand the density matrix to
incorporate the classical bit as if it were a qubit, necessitating a $4\times
	4$ matrix for a 2-qubit system, with sixteen entries.
%
However, this is less efficient than simply splitting the world into two,
requiring two $2\times 2$ matrices, which have eight entries in total.
%
Therefore, we should utilize mixed states only when we do not need to add
classical bit information to the density matrix.

As a solution, we identify two cases where mixed states can be used without
concerning classical bits and optimize each case by replacing two worlds with a
single world containing a mixed state.
%
This leads to two optimizations: \emph{world merging} and \emph{efficient
	reset}.
%
\cref{fig:optimization} defines the operational semantics for optimized
executions.
%
We now describe each optimization in detail.

\paragraph{World Merging}

Rule [\textsc{Eval-Merge}] defines the world-merging optimization, which can be
applied to the evaluation of any statements.
%
During execution, when two distinct worlds with identical classical states but
different quantum states emerge, we merge their quantum states into a single
density matrix, creating a mixed state while preserving the identical classical
state.
%
This rule guarantees that the number of worlds during the execution of a
circuit with $n$ different classical bits is upper-bounded by $2^n$ because the
system can result in a maximum of $2^n$ different classical states.

\paragraph{Efficient Reset}

Rule [\textsc{Eval-Reset-Opt}] enhances the efficiency of evaluating reset
statements by replacing Rule [\textsc{Eval-Reset}].
%
In the case of Rule [\textsc{Eval-Reset}], a reset statement assigns a
measurement outcome from a qubit to a classical bit.
%
This classical bit is solely utilized for conditionally flipping the qubit and
is never used thereafter.
%
Therefore, we can simply remove the classical bit from the world and employ a
mixed state instead of splitting the world into two.
%
Adopting this idea, Rule [\textsc{Eval-Reset-Opt}] updates the density matrix
but does not split the world.

\begin{figure}
	\[
		\small
		\begin{array}{@{}c@{}}
			\inferrule
			{
			\eval {\rep{w}} {s} {w'_0,\ldots,w'_i,\ldots,w'_j,\ldots,w'_n}      \\\\
			w'_i = (\rho_i, \sigma, P_i)                                        \\
			w'_j = (\rho_j, \sigma, P_j)                                        \\
			\rho_{ij} = \tfrac{P_i}{P_i+P_j}\rho_i + \tfrac{P_j}{P_i+P_j}\rho_j \\
			} {
			\eval {\rep{w}} {s} {w'_0,\ldots,w'_{i-1},w'_{i+1},\ldots,w'_{j-1},w'_{j+1},\ldots,w'_n,(\rho_{ij}, \sigma, P_i + P_j)}			}
			\quad[\textsc{Eval-Merge}]
			\\[3em]
			\inferrule
			{
			\qubits{\rho}=n                                                     \\
			\rho_0 = \smeasure{\rho}{\mproj{n,q,0}}                             \\
			p_0 = \sprob{\rho}{\mproj{n,q,0}}                                   \\\\
			\rho_1 = \smeasure{\rho}{\mproj{n,q,1}}                             \\
			p_1 = \sprob{\rho}{\mproj{n,q,1}}                                   \\\\
				\rho' = p_0\rho_0 + p_1\sunitary{\rho_1}{U_{\pi,0,\pi}}
			} {
				\eval {(\rho,\sigma,P)} {\ireset{q}} {(\rho',\sigma,P)}
			}
			\quad[\textsc{Eval-Reset-Opt}]
		\end{array}
	\]
	\caption{Optimized semantics}
	\label{fig:optimization}
\end{figure}

\section{Physical Consistency}
\label{ch:consistency:consistency}

\noindent
We introduce the postulates of quantum computing and prove that the optimized
version of the proposed formal semantics adheres to the postulates.
%
\citet{scherer2019mathematics} summarizes the laws of quantum physics within
five postulates.
%
The exact quote of the postulates can be found in our companion
report~\cite{supp}.
%
To ensure clarity and consistency with the terminologies defined in this paper,
we rephrase the original postulates by presenting three postulates:
\cref{pos:gate} (corresponding to postulate 4), \cref{pos:measurement}
(corresponding to postulates 2 and 3), and \cref{pos:density} (corresponding to
postulate 5).
%
We ignore postulate 1 due to its irrelevance to the functionality of
OpenQASMCore.

The first postulate describes the operation of gates:
%
\begin{postulate}[Gate]
	%
	\label{pos:gate}
	%
	The change of a density matrix from $\rho$ to $\rho'$ by a gate is described by
	$\rho'=M\rho M^*$ where $M$ is a unitary matrix, i.e., $M^*M=I$.
	%
\end{postulate}
%
\noindent
%
Since OpenQASMCore has two kinds of gate statements, we prove that each
satisfies the postulate.
%
\begin{theorem}[Rotation Gate]
	%
	\label{thm:rotation}
	%
	Suppose that the following holds:
	%
	$\eval{(\rho,\sigma,P)}{\irotate{r_1}{r_2}{r_3}{q}}{(\rho',\sigma',P')}$.
	%
	Then, there exists a unitary matrix $M$ such that $\rho'=M\rho M^*$.
	%
\end{theorem}
%
\begin{proof}
	%
	By rule [\textsc{Eval-Rotate}], $\rho'=M\rho M^*$ where
	$M=\mrotate{\qubits{\rho},q,r_1,r_2,r_3}$. Therefore, it is enough to show that
	$M$ is unitary. We can easily show that a $2\times2$ matrix
	$\mrotate{r_1,r_2,r_3}$ and the identity matrix of any dimension are unitary.
	Since the tensor product of unitary matrices is also unitary, $M$ is unitary.
	%
\end{proof}
%
\begin{theorem}[Cnot Gate]
	%
	\label{thm:cnot}
	%
	Suppose that the following holds:
	%
	$\eval{(\rho,\sigma,P)}{\icnot{q_1}{q_2}}{(\rho',\sigma',P')}$.
	%
	Then, there exists a unitary matrix $M$ such that $\rho'=M\rho M^*$.
	%
\end{theorem}
%
\begin{proof}
	%
	By rule [\textsc{Eval-Cnot}], $\rho'=M\rho M^*$ where
	$M=\mcnot{\qubits{\rho},q_1,q_2}$, it is enough to show that $M$ is unitary. By
	induction on $n$, we can show that $\mswap{n,0,n-1}$ is unitary for any $n$.
	Since the tensor product of unitary matrices is also unitary,
	$\mswap{n,q_1,q_2}$ is unitary. It is easy to show that the $4\times4$ matrix
	$\mcnot{}$ is unitary. Since the product of unitary matrices is also unitary,
	$M$ is unitary.
	%
\end{proof}

The second postulate describes the results of measurements:
%
\begin{postulate}[Measurement]
	%
	\label{pos:measurement}
	%
	For a quantum system described by a density matrix $\rho$, a physical
	observable to be measured is represented by a self-adjoint matrix $A$, whose
	dimension is the same as $\rho$.
	%
	Each measurement outcome corresponds to $\lambda$, an eigenvalue of $A$.
	%
	The probability of the outcome corresponding to $\lambda$ is $\trace{\rho M}$
	where $M$ is the projection onto the eigenspace of $\lambda$.
	%
	The change of a density matrix from $\rho$ to $\rho'$ by the measurement is
	described by $\rho'=\frac{1}{\trace{\rho M}}M\rho M$.
	%
\end{postulate}
%
\noindent
%
OpenQASMCore has only one kind of measurement statement, and we show that the
semantics of the statement follows the postulate:
%
\begin{theorem}[Measurement]
	%
	\label{thm:measure}
	%
	Suppose that the following holds:
	%
	$\eval{(\rho,\sigma,P)}{\imeasure{q}{c}}{\rep{w'}}$.
	%
	Then, $\exists A.\forall w'\in\{\rep{w'}\}.\exists\lambda$ such that the
	followings hold:
	%
	(1) $A$ is self-adjoint;
	%
	(2) $\lambda$ is an eigenvalue of $A$;
	%
	(3) $M$ is the projection operator onto the eigenspace of $\lambda$;
	%
	(4) $w'=(\rho',\sigma',P')$;
	%
	(5) $\rho'=\frac{1}{\trace{\rho M}}M\rho M$;
	%
	(6) $\frac{P'}{P}=\trace{\rho M}$.
	%
\end{theorem}
%
\begin{proof}
	%
	By rules [\textsc{Eval-Measure}], [\textsc{Eval-Measure-0}], and
		[\textsc{Eval-Measure-1}], $w'\in\{\rep{w'}\}$ is either
	$(\smeasure{\rho}{\mproj{n,q,0}},\sigma[c\mapsto0],P\times\sprob{\rho}{\mproj{n,q,0}})$
	or
	$(\smeasure{\rho}{\mproj{n,q,1}},\sigma[c\mapsto1],P\times\sprob{\rho}{\mproj{n,q,1}})$
	where $n=\qubits{\rho}$.
	%
	Let $A=\identity{q}\otimes\smtrx{1&0\\0&-1}\otimes\identity{n-q-1}$.
	%
	We can easily show that $1$ and $-1$ are eigenvalues of $A$, and
	$\mproj{n,q,0}$ and $\mproj{n,q,1}$ are their corresponding projections.
	%
\end{proof}

The last postulate describes the properties of density matrices:
%
\begin{postulate}[Density Matrix]
	%
	\label{pos:density}
	%
	A density matrix $\rho$ satisfies the following:
	%
	(1) $\rho$ is self-adjoint: $\rho^*=\rho$;
	%
	(2) $\rho$ is positive semi-definite: $\rho\ge0$, i.e., $z^*\rho z\ge 0$ for every
	column vector $z$;
	%
	(3) $\trace{\rho}=1$.
	%
\end{postulate}
%
\noindent
%
We show that for every OpenQASMCore program, if its execution starts with a
valid density matrix, then the execution ends with valid density matrices.
%
\begin{theorem}[Density Matrix]
	%
	Suppose that the following holds:
	%
	$\eval{(\rho,\sigma,P)}{s}{\rep{w'}}$.
	%
	If $\rho$ is self-adjoint and positive semi-definite and $\trace{\rho}=1$, then
	for every $(\rho',\sigma',P')\in\{\rep{w'}\}$, $\rho'$ is self-adjoint and
	positive semi-definite and $\trace{\rho'}=1$.
	%
\end{theorem}
%
\begin{proof} We prove it by structural induction on $s$.
	%
	\begin{itemize}
		%
		\item $s=\inop$.
		      %
		      By rule [\textsc{Eval-Nop}], $\rho'=\rho$.
		      %
		\item $s=\iseq{s_1}{s_2}$.
		      %
		      We prove it using rule [\textsc{Eval-Seq}] and the induction hypothesis.
		      %
		\item $s=\iif{c}{b}{s'}$.
		      %
		      If $\sigma(c)\not=b$, $\rho'=\rho$ by rule [\textsc{Eval-If-False}].
		      %
		      If $\sigma(c)=b$, we prove it using rule [\textsc{Eval-If-True}] and the
		      induction hypothesis.
		      %
		\item $s=\irotate{r_1}{r_2}{r_3}{q}$ or $s=\icnot{q_1}{q_2}$.
		      %
		      By \cref{thm:rotation,thm:cnot}, $\rho'=M\rho M^*$ where $M$ is unitary.
		      %
		      Basic linear algebra allows us to prove it using that $M$ is unitary.
		      %
		\item $s=\imeasure{q}{c}$.
		      %
		      By \cref{thm:measure}, $\rho'=\frac{1}{\trace{\rho M}}M\rho M$ where $M$ is a
		      projection.
		      %
		      Basic linear algebra allows us to prove it using that $M$ is a projection.
		      %
		\item $s=\ireset{q}$.
		      %
		      Rule [\textsc{Eval-Reset-Opt}] yields $\rho'=p_0\rho_0 +
			      p_1\sunitary{\rho_1}{X}$ where $p_0$ and $p_1$ correspond to measurement
		      outcomes of $q$ being 0 and 1, respectively, and $X$ representing a unitary
		      operator.
		      %
		      By proving that probabilities are non-negative and their sum is unity, and that
		      positive semi-definiteness and self-adjointness are preserved under convex
		      combinations, $\rho'$ retains the required properties.
	\end{itemize}
	%
	Finally, the application of [\textsc{Eval-Merge}] preserves the three
	properties of density matrices.
	%
	If $\rho' = \frac{P_0}{P_0+P_1}\rho_1 + \frac{P_1}{P_0+P_1}\rho_2$, which is a
	convex combination of $\rho_1$ and $\rho_2$, all properties are preserved,
	analogous to the $s=\ireset{q}$ case.
	%
\end{proof}

\noindent
%
The initial state in OpenQASMCore is a state where every qubit is $\ket{0}$,
which is straightforward to prove that it is a valid density matrix.
%
Therefore, we ensure that the execution of every OpenQASMCore program ends with
valid density matrices.
